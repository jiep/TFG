\chapter{Introducción} \label{chp:introduccion}

\todo[backgroundcolor=green, inline]{Cambiar fuentes al texto}
\todo[backgroundcolor=orange, inline]{Revisar introducción}
\todo[backgroundcolor=orange ,inline]{Añadir más cosas a la introducción}
\todo[backgroundcolor=red,inline]{Introducir nombres de rakings famosos. Ver \cite[pp. 6]{langville2012s}}

En la actualidad, vivimos rodeados de rankings de todo tipo: 
\begin{itemize}
\item Sociales, como los barómetros del CIS donde se miden distintos parámetros que afectan a la sociedad española (valoración de los políticos, ideología política, tema que más preocupan, etc), rankings de universidades que miden la calidad de la educación, la calidad del profesorado, etc., como el Academic Ranking of World University, el Times Higher Educaction, el Quacquarelli Symonds, etc.
\item Económicos, como el Producto Interior Bruto (PIB), que mide el conjunto de bienes y servicios finales producidos en un país durante un año, la renta per cápita, que mide la relación entre el PIB y la población de un país, la prima de riesgo, que mide la diferencia entre el interés que se paga por la deuda de un país y el que se paga por la de otro, etc.
\item Deportivos, como el ranking FIFA en fútbol, el ranking FIBA en baloncesto, el ranking ATP/WTA en tenis, el Draft de la NBA, etc.
\end{itemize}

Como podemos observar, existen miles de rankings de casi cualquier tema que podamos imaginar, ya que en definitiva, se trata de ordenar unos determinados elementos según unos determinados parámetros que es lo que denominaremos rating, y una vez ordenados estos elementos se obtiene un ranking.\\

Puesto que podemos obtener varios rankings de un conjunto determinado de elementos es interesante poder comparar de alguna forma estos rankings, o poder ``unirlos'' para obtener un nuevo ranking que contenga información de estos rankings iniciales. A este proceso lo denominaremos agregación de rankings.\\

En esta memoria veremos, desde un punto de vista matemático, qué es un rating y un ranking, distintos métodos para obtener ratings (y a su vez rankings), métodos de agregación de rankings con sus ventajas e inconvenientes, compararemos rankings introduciendo dos medidas como la tau de Kendall y la ro y la fi de Spearman. Por último veremos nuevos métodos para estudiar la competitividad en rankings mediante la teoría de grafos, introduciendo un nuevo tipo de grafo y cuáles son sus relaciones con otros tipos de grafos como los grafos de comparabilidad, de permutaciones, cordales y los f-grafos.