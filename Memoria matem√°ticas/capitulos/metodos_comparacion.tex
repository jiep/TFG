\chapter{Métodos de comparación}

En los capítulos anteriores hemos visto que se pueden crear distintos métodos para crear ratings (y rankings) y cómo se pueden agregar varios rankings en uno sólo. Ahora nos interesa saber cuál de todos los métodos es mejor. Para responder a ésto, veremos una serie de medidas que son permitirán comparar varios rankings. Estas medidas son la tau de Kendall (tanto en rankings parciales como en rankings completos), la ro de Spearman y la fi ponderada de Spearman (en rankings completos y en rankings parciales). Estas medidas están referidas a dos rankings.

\section{Tau de Kendall}

La tau de Kendall es una medida que permite conocer la similitud entre dos rankings. Consideraremos dos casos: cuando los rankings son completos y cuando son parciales.

\subsection{Tau de Kendall en rankings completos}

La tau de Kendall es una medida que determina la correlación entre dos rankings completos. Recordemos que dos rankings son completos cuando tienen los mismos elementos. Esta medida fue desarrolla en 1938 por Maurice Kendall.\\

\begin{defi}($\tau$ de Kendall en rankings completos) 

La $\tau$ de Kendall para rankings completos con tamaño $n$ se define como sigue

\begin{equation} \label{def:tau_kendall}
\tau = \dfrac{n_c - n_d}{n(n-1)/2}
\end{equation}

donde $n_c$ es el número de pares que concuerdan en ambos rankings y $n_d$ es el número de pares que no concuerdan en ambos rankings.\\

Un par $(i,j)$ se dice que concuerda si el elemento $i$ aparece por encima del elemento $j$ en ambos rankings. En caso contrario se dice que no concuerda.\\


La $\tau$ de Kendall está acotada por $-1$ y $1$, es decir, $-1 \leq \tau \leq 1$. Cuando $\tau = 1$, los rankings son iguales y cuando $\tau = -1$, los rankings están en orden inverso.  
\end{defi}

\begin{ejemplo}
Consideremos los siguientes rankings:

\begin{equation*}
c_1 = \left( \begin{array}{c}
3\\
2\\
1\\
4\\
5
\end{array} \right), \quad
c_2 = \left( \begin{array}{c}
3\\
2\\
1\\
4\\
5
\end{array} \right), \quad
c_3 = \left( \begin{array}{c}
5\\
4\\
1\\
2\\
3
\end{array} \right), \quad
c_4 = \left( \begin{array}{c}
2\\
1\\
4\\
5\\
3
\end{array} \right)
\end{equation*}

Calculemos la tau de Kendall para estos rankings.

\begin{align*}
\tau(c_1, c_2) & = \dfrac{10 - 0}{5 \cdot 4/2} = 1\\
\tau(c_1, c_3) & = \dfrac{0 - 10}{10} = -1\\
\tau(c_3, c_4) & = \dfrac{4 -  6}{10} = -\dfrac{2}{10} = - \dfrac{1}{5}
\end{align*}

Para el último caso, tenemos que los pares que concuerdan en ambos rankings son $(1,3)$, $(2,3)$, $(3,4)$ y $(3,5)$. Los pares que no concuerdan son $(1,2)$, $(1,4)$, $(1,5)$, $(2,4)$, $(2,5)$ y $(4,5)$.

\end{ejemplo}

\subsection{Tau de Kendall en rankings parciales}

No siempre disponemos de rankings completos, por lo que es necesario disponer de la tau de Kendall para rankings parciales. Una forma de definir la tau de Kendall para rankings parciales es la siguiente.

\begin{defi}[Tau de Kendall para rankings parciales]

La tau de Kendall para rankings parciales se define de la siguiente manera

\begin{eqnarray}
\tau_p = \frac{n_c - n_d - n_u}{\binom{n}{2} - n_u}
\end{eqnarray}

donde $n_c$ es el número de pares que concuerdan, $n_d$ es el número de pares que no concuerdan y $n_u$ es el número de pares que no aparecen en el ranking. En este caso, $n$ es el número de elementos únicos de los rankings.

\end{defi}

De esta manera $|\tau| \not \leq 1$, sino que se cumple que

\begin{equation}
\dfrac{-\binom{n}{2}}{\binom{n}{2} - n_u} \leq \tau_p \leq \dfrac{\binom{n}{2}}{\binom{n}{2} - n_u}
\end{equation}

Notar que $\tau_p$ generaliza a $\tau$. Cuando los rankings son completos, $n_u = 0$, que coincide con la definición de $\tau$ para rankings completos.

\begin{ejemplo}
Consideremos los siguientes rankings:

\begin{equation*}
c_1 = \left( \begin{array}{c}
1\\
3\\
2
\end{array} \right), \quad
c_2 = \left( \begin{array}{c}
4\\
2\\
1
\end{array} \right), \quad
c_3 = \left( \begin{array}{c}
5\\
4\\
2\\
3
\end{array} \right), \quad
c_4 = \left( \begin{array}{c}
4\\
1\\
5\\
6\\
7
\end{array} \right)
\end{equation*}

Calculamos algunos ejemplos de $\tau_p$

\begin{align*}
\tau_p(c_1, c_2) = & -3\\
\tau_p(c_1, c_3) = & -2.6667\\
\tau_p(c_1, c_4) = & -2.6250
\end{align*}

Hay que tener en cuenta que si queremos comparar estos valores, debemos calcular $\tau$ donde $n$ sea el número de elementos únicos entre los rankings que queremos comparar. En este caso, $n=7$. Los valores calculados con este valor de $n$ son  los siguientes:

\begin{align*}
\tau_p(c_1, c_2) = & -0.3529\\
\tau_p(c_1, c_3) = & -0.5714\\
\tau_p(c_1, c_4) = & -2.6250
\end{align*}

\end{ejemplo}

Esta medida tiene un problema: al penalizar las discrepancias $n_d$ y $n_u$ entre rankings mediante simples cuentas, no se tienen en cuenta las posiciones entre las que se producen estas discrepancias.

\section{Ro de Spearman}

\begin{defi}[Ro de Spearman]
Definimos la ro de Spearman como la distancia $L_1$ entre dos rankings completos $c_1$ y $c_2$ de tamaño $n$. Esto es, 

\begin{equation}
\rho = \norm{c_1 - c_2}_1 = \sum\limits_{i=1}^{n} |c_1^{-1}(i) - c_2^{-1}(i)|
\end{equation}

donde $c_1^{-1}(i)$ es la posición del equipo $i$ en el ranking $c_1$ y $c_2^{-1}(i)$ es la posición del equipo $i$ en el ranking $c_2$.
\end{defi}

\begin{ejemplo}
Consideremos los siguientes rankings:

\begin{equation*}
c_1 = \left( \begin{array}{c}
2\\
3\\
4\\
1\\
5
\end{array} \right), \quad
c_2 = \left( \begin{array}{c}
5\\
4\\
1\\
3\\
2
\end{array} \right), \quad
c_3 = \left( \begin{array}{c}
1\\
2\\
3\\
4\\
5
\end{array} \right), \quad
c_4 = \left( \begin{array}{c}
2\\
3\\
4\\
1\\
5
\end{array} \right)
\end{equation*}

Para los rankings $c_1$ y $c_2$ se tiene que, para el primer equipo, $c_1^{-1}(1) = 4$ y $c_2^{-1}(1) = 3$. Así, $|4-3| = 1$, para el segundo equipo, la diferencia es $4$, para el tercero es $2$, para el cuarto es $1$ y para el quinto $4$. De esta forma $\rho(c_1, c_2) = 12$. De la misma forma se obtiene que 

\begin{align*}
\rho(c_1, c_3) & = 6\\
\rho(c_1, c_4) & = 0\\
\rho(c_2, c_3) & = 12\\
\rho(c_3, c_4) & = 6
\end{align*}  
\end{ejemplo}

\section{Fi de Spearman}

A partir de la ro de Spearman, se puede definir otra medida, que llamaremos fi de Spearman, que tiene en cuenta las posiciones relativas entre los rankings.

\subsection{Fi de Spearman para rankings completos}

Si tenemos dos rankings completos de tamaño $n$, la fi de Spearman se define de la siguiente manera.

\begin{defi}[Fi de Spearman para rankings completos]
Se define la fi de Spearman entre dos rankings completos $c_1$ y $c_2$ de tamaño $n$ como 

\begin{equation}
\phi = \sum\limits_{i=1}^{n} \dfrac{|c_1^{-1}(i) - c_2^{-1}(i)|}{\mathrm{min}\{c_1^{-1}(i), c_2^{-1}(i)\}}
\end{equation}
\end{defi}

\begin{ejemplo} \label{ej:fi_spearman_completo}
Consideremos los siguientes rankings:

\begin{equation*}
c_1 = \left( \begin{array}{c}
2\\
3\\
4\\
1\\
5
\end{array} \right), \quad
c_2 = \left( \begin{array}{c}
5\\
4\\
1\\
3\\
2
\end{array} \right), \quad
c_3 = \left( \begin{array}{c}
1\\
2\\
3\\
4\\
5
\end{array} \right), \quad
c_4 = \left( \begin{array}{c}
2\\
3\\
4\\
1\\
5
\end{array} \right)
\end{equation*}

Algunos valores de $\phi$ para estos rankings son

\begin{align*}
\phi(c_1, c_2) & = 9.8333 \\
\phi(c_1, c_3) & = 4.8333\\
\phi(c_1, c_4) & = 0\\
\phi(c_2, c_3) & = 8.8333
\end{align*}  
\end{ejemplo}

Se tiene que $\phi \geq 0$, y cuanto más pequeño es $\phi$ más cerca están los rankings comparados. 

\subsection{Fi de Spearman para rankings parciales}

Cuando se trata de rankings parciales, la definición de fi de Spearman es más complicada como vemos a continuación.

\begin{defi}[Fi de Spearman para rankings parciales]
La fi de Spearman para dos rankings parciales $c_1$ y $c_2$ de longitud $n$ se define de la siguiente manera:

\begin{equation}
\phi = \dfrac{\sum\limits_{i=1}^{n} \phi_i}{\phi(l, l^c)}
\end{equation} 

donde 

\begin{equation}
\phi(l, l^c) = -2n + 2x\sum\limits_{i=1}^{n} \dfrac{1}{i}
\end{equation}

y $\phi_i$ se calcula de alguna de estas dos formas

\begin{itemize}
\item Si el elemento $i$ está en $c_1$ y $c_2$,

\begin{equation}
\phi_i = \dfrac{|c_1^{-1}(i) - c_2^{-1}(i)|}{\mathrm{min}\{c_1^{-1}(i), c_2^{-1}(i)\}}
\end{equation}

\item Si el elemento $i$ está en alguna de los dos rankings, que asumimos sin pérdida de generalidad que es $c_1$,


\begin{equation}
\phi_i = \dfrac{|c_1^{-1}(i) - x|}{\mathrm{min}\{c_1^{-1}(i), x\}}
\end{equation} 

donde $x$ se define como

\begin{equation}
x = \dfrac{n - 4 \lfloor n/2\rfloor + 2(n+1) \sum_{i=1}^{\lfloor n/2\rfloor} 1/i}{\sum_{i=1}^{n} 1/i}
\end{equation}
\end{itemize}
\end{defi}

Se tiene que $0 \leq \phi \leq 1$, y cuanto más cerca está $\phi$ de $0$, más se parecen los rankings.

\begin{ejemplo}
Consideremos los siguientes rankings:

\begin{equation*}
c_1 = \left( \begin{array}{c}
2\\
3\\
4\\
1\\
5
\end{array} \right), \quad
c_2 = \left( \begin{array}{c}
5\\
4\\
1\\
3\\
2
\end{array} \right), \quad
c_3 = \left( \begin{array}{c}
1\\
2\\
3\\
4\\
5
\end{array} \right), \quad
c_4 = \left( \begin{array}{c}
2\\
3\\
4\\
1\\
5
\end{array} \right)
\end{equation*}

Algunos valores de $\phi$ para estos rankings son

\begin{align*}
\phi(c_1, c_2) & = 0.4917 \\
\phi(c_1, c_3) & = 0.2417 \\
\phi(c_1, c_4) & = 0\\
\phi(c_2, c_3) & = 0.4417
\end{align*}  

Estos rankings son los mismos que en el Ejemplo \ref{ej:fi_spearman_completo}, por lo que son completos. Si comparamos resultados, vemos que no se tienen los mismos resultados en ambos casos, por lo que, en principio, no generaliza al caso de rankings completos. Si dividimos entre el valor $\phi(l, l^c)$, obtenemos los mismos resultados. En este caso, $\phi(l, l^c) = 20$, por lo que $\phi(c_1, c_2)/20 = 0.4917$ que coincide con el valor obtenido con por el método de los rankings parciales. Para los demás valores, también se obtiene el resultado del Ejemplo~\ref{ej:fi_spearman_completo}.\\

Si ahora consideramos estos rankings parciales,

\begin{equation*}
c_5 = \left( \begin{array}{c}
6\\
8\\
3\\
4\\
1
\end{array} \right), \quad
c_6 = \left( \begin{array}{c}
9\\
3\\
4\\
2\\
1
\end{array} \right), \quad
c_7 = \left( \begin{array}{c}
1\\
2\\
3\\
4\\
5
\end{array} \right), \quad
c_8 = \left( \begin{array}{c}
2\\
9\\
7\\
6\\
5
\end{array} \right)
\end{equation*}

Todos los valores de $\phi$ para estos rankings son

\begin{equation*}
\begin{array}{ccc}
\phi(c_5, c_6) = 0.7450, & \phi(c_5, c_7) = 0.7226, & \phi(c_5, c_8) = 0.8394\\
\phi(c_6, c_7) = 0.5858, & \phi(c_6, c_8) = 0.4967, & \phi(c_7, c_8) = 0.6258 
\end{array}
\end{equation*}
\end{ejemplo}