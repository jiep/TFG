\chapter{Conclusiones}

Durante esta memoria, hemos visto los conceptos básicos de rankings y ratings, así como distintos métodos para obtener más rankings. Estos métodos han sido los métodos de Massey, Colley y Markov, cada uno de ellos basado en distintas ideas: el primero de ellos, en mínimos cuadrados, el segundo en una generalización de la regla de Laplace, y el tercero en cadenas de Markov.\\

También, vimos una técnica llamada agregación de rankings que consiste en ``unir'' varios rankings para formar uno solo. De está técnica vimos varios métodos: el ranking promedio, el método de Borda, el método óptimo y el método de partidos simulados.\\

Para comparar los distintos rankings, explicamos dos medidas clásicas como lo son la tau de Kendall y la phi de Spearman.\\

Por último, vimos una forma de estudiar la competitividad en rankings: estudiando los cambios de posición entre dos rankings consecutivos. Ésto dio lugar a un nuevo tipo de grafo (el de competitividad) que tenía relación con otros tipos de grafos ya conocidos como los grafos de comparabilidad, de permutaciones, los f-grafos y los cordales.

\section{Mejoras y futuro trabajo}

\begin{itemize}

\item Añadir un modelo de predicción para ver cómo evolucionará la competitividad a lo largo del tiempo.

\item Estudiar la competitividad por zonas del ranking, es decir, dividir cada ranking en distintas regiones y estudiar la competitividad en cada una de estas regiones. 
\end{itemize}
