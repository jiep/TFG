\chapter{Programación lineal}

La programación lineal es un método de optimización de una función lineal llamada función objetivo, sujeta a un conjunto de restricciones formadas por ecuaciones e inecuaciones lineales.  \\

A continuación introduciremos los conceptos básicos de programación lineal y los conceptos de problema primal y problema dual.

\begin{defi}
Llamamos problema de programación lineal al siguiente conjunto de ecuaciones e inecuaciones:

\begin{equation}\label{eq:lineal}
\begin{array}{rl}
\mathrm{min} & c_1x_1 + c_2x_2 + \dots + c_nx_n\\
\mathrm{s.a} & a_{11}x_1 + a_{12}x_2 + \dots + a_{1n}x_n = b_1\\
             & a_{21}x_1 + a_{22}x_2 + \dots + a_{2n}x_n = b_2\\
             & \dots                                          \\
             & a_{m1}x_1 + a_{m2}x_2 + \dots + a_{mn}x_n = b_m\\
\text{y}     & x_1 \geq 0, x_2 \geq 0, \dots, x_n \geq 0
\end{array}
\end{equation}

donde $b_i$, $c_i$ y $a_{ij}$ son constantes reales y $x_i$ son nuestras incógnitas.\\

Si empleamos la notación vectorial, el problema tiene una notación más compacta:\\

\begin{equation} \label{eq:forma_normal}
\begin{array}{rl}
\mathrm{min} & \mathbf{c}^T \mathbf{x}\\
\mathrm{s.a} & \mathbf{A x} = \mathbf{b} \text{ y } \mathbf{x} \geq \mathbf{0}
\end{array}
\end{equation} 

\end{defi}

\begin{defi}
La función $\mathbf{c}^T \mathbf{x}$ recibe el nombre de función objetivo.
\end{defi}

\begin{defi}
Las ecuaciones $\sum_{j=1}^{n} a_{ij}x_i = b_i$ reciben el nombre de restricciones funcionales (o estructurales).\\

Las inecuaciones $x_i \geq 0$ reciben el nombre de restricciones de no negatividad.
\end{defi}

\begin{defi}
Se llama solución a cualquier conjunto de valores específico de las variables $\mathbf{x} = (x_1, x_2, \dots, x_n)^T$.
\end{defi}

\begin{defi}
Una solución se dice factible si todas las restricciones (funcionales y de no negatividad) se satisfacen. En caso contrario, se dice que es una solución no factible.
\end{defi}

\begin{defi}
La región factible es la unión de todas las soluciones factibles.
\end{defi}

\begin{ejemplo}
Dado el siguiente problema de programación lineal

\begin{equation}
\begin{array}{rl}
\mathrm{min} & 3x + 5y\\
\mathrm{s.a} & x \leq 4\\
			 & 2y \leq 12\\
			 & 3x + 2y \leq 18\\
			 & x, y \geq 0 
\end{array} 
\end{equation}

La región del problema se puede ver en la Figura \ref{fig:ejemplo_region_factible}.

\begin{figure}[htb]
\centering
\ejemploregionfactible
\caption{Región factible}
\label{fig:ejemplo_region_factible}
\end{figure}
\end{ejemplo}

\begin{defi}
Una solución es óptima si proporciona el valor más pequeño de la función objetivo.
\end{defi}

\begin{defi}
Sea $\mathbf{A x} = \mathbf{b}$ el conjunto de restricciones de no negatividad. Sea $\mathbf{B} \in \R^{m \times m}$ cualquier submatriz compuesta de las columnas de $\mathbf{A}$. Entonces, si todas las $n-m$ componentes de $\mathbf{x}$ no asociadas con las columnas de $\mathbf{A}$ son iguales a cero, la solución del sistema resultante se dice una solución básica de $\mathbf{A x} = \mathbf{b}$ con respecto a la base $\mathbf{B}$. Las componentes asociadas a $\mathbf{x}$ con las columnas de $\mathbf{B}$ se llaman variables básicas.
\end{defi}

\begin{defi}
Si una o más de las variables básicas de la solución básica tiene valor cero, se dice que es una solución básica degenerada.
\end{defi}

\begin{teo}[fundamental de programación lineal]
Dado un problema lineal de la forma \ref{eq:forma_normal} donde $\mathbf{A} \in \R^{m \times n}$ con rango $m$. Entonces,

\begin{enumerate}
\item Si hay una solución factible, entonces es una solución básica factible.
\item Si hay una solución factible óptima, es una solución factible básica óptima.
\end{enumerate}
\end{teo}

\begin{defi}
Llamamos problema primal al problema \ref{eq:dual}.\\

Llamamos problema dual al problema 

\begin{equation} \label{eq:dual}
\begin{array}{rl}
\mathrm{max} & \mathbf{\boldsymbol\lambda}^T \mathbf{b}\\
\mathrm{s.a} & \mathbf{\boldsymbol\lambda}^T \mathbf{A} \leq \mathbf{c}^T 
\end{array}
\end{equation} 
\end{defi}

\begin{lema}(de dualidad débil)
Si $\mathbf{x}$ y $\mathbf{\boldsymbol\lambda}$ son soluciones factibles de (\ref{eq:forma_normal}) y (\ref{eq:dual}), entonces

\[ \mathbf{c}^T \geq \mathbf{\boldsymbol\lambda}^T \mathbf{b} \]
\end{lema}

\begin{cor}
Si $\mathbf{x}_0$ y $\mathbf{\boldsymbol\lambda}_0$ son soluciones factibles de (\ref{eq:forma_normal}) y de (\ref{eq:dual}), respectivamente, y si $\mathbf{c}^T \mathbf{x_0} = \mathbf{\boldsymbol\lambda}_0^T \mathbf{b}$, entonces $\mathbf{x}_0$ y $\mathbf{\boldsymbol\lambda}_0$ son óptimas para sus respectivos problemas.
\end{cor}

\begin{teo}[de dualidad de programación lineal]
Si cualquiera de los problemas (\ref{eq:forma_normal}) o (\ref{eq:dual}) tiene una solución óptima finita, también lo hace el otro, y los correspondientes valores de la función objetivo son iguales. Si cualquier problema una función objetivo no acotada, el otro problema no tiene solución factible.  
\end{teo}
