\chapter{Conceptos básicos de rankings} \label{chp:conceptos_basicos}

En este capítulo veremos dos de los conceptos claves de esta memoria: ranking y rating, que darán lugar a cómo estudiar la competitividad en los rankings.\\

Para entender estos conceptos, comencemos con un ejemplo:

\begin{ejemplo}
Consideremos el número de goles, los partidos jugados y el promedio de goles de Messi (MES), Cristiano Ronaldo (CRO) y Raúl González (RAU) en la UEFA Champions League.

\begin{table}[h]
\centering
\caption[Goles, partidos jugados de los máximos goleadores de la Champions]{Goles, partidos jugados y promedio de goles de los máximos goleadores de la Champions}
\label{tbl:goles_champions}

\begin{tabular}{cccc}
\cline{2-4}
    & Goles & Partidos jugados & Promedio goleador \\ \cline{2-4} 
MES & 77    & 97               & 0.79              \\
CRO & 77    & 115              & 0.67              \\
RAU & 71    & 142              & 0.5               \\ \hline
\end{tabular}
\end{table}


Estas tres estadísticas del juego (goles, partidos jugados y promedio) pueden dar lugar a distintas ordenaciones de los jugadores. Por ejemplo, si consideramos el número de goles o el promedio goleador, el ranking de estos tres jugadores quedaría de la siguiente manera:

\[
\begin{array}{cc}
\text{Posición} & \text{Goles}\\ 
\begin{array}{c}
\text{1}\\
\text{2}\\
\text{3}
\end{array} & \left(\begin{array}{c}
\text{MES}\\
\text{CRO}\\
\text{RAU}
\end{array} \right)
\end{array}
\] 

Sin embargo, si consideramos el número de partidos jugados en la UEFA Champions League, el ranking quedaría de la siguiente manera:

\[
\begin{array}{cc}
\text{Posición} & \text{Partidos jugados}\\ 
\begin{array}{c}
\text{1}\\
\text{2}\\
\text{3}
\end{array} & \left(\begin{array}{c}
\text{RAU}\\
\text{CRO}\\
\text{MES}
\end{array} \right)
\end{array}
\] 

Vemos, que distintos parámetros pueden dar lugar a diferentes rankings.

\end{ejemplo}


Un rating, de forma intuitiva, es un criterio de ordenación de un conjunto de elementos; y un ranking es una posible ordenación de un conjunto de elementos.\\

Veamos las definiciones rigurosas de estos dos conceptos:

\begin{defi} \label{def:ranking}
Dado un conjunto $\mathcal{N} = \{1,\dots,n\}$ que llamamos nodos, definimos el ranking $c$ como cualquier biyección $c : \mathcal{N} \to \mathcal{N}$.
\end{defi}

\begin{defi}
Dado un conjunto $\mathcal{N} = \{1,\dots, n\}$ de nodos, decimos que es un rating si a cada $i \in \mathcal{N}$ se le asigna un valor real, es decir, una función $\mathrm{r} : \mathcal{N} \to \R$.
\end{defi}

De acuerdo a estas dos definiciones, podemos definir $\mathcal{N} = \{1,2,3 \}$ donde MES $\equiv 2$, CRO $\equiv 3$ y RAU $\equiv 1$. De esta forma podemos ver el ranking de goles como la biyección $c: \mathcal{N} \to \mathcal{N}$ definida de la siguiente manera:

\[ \begin{array}{rlll}
c: & \mathcal{N} & \to & \mathcal{N}\\
& 1 & \mapsto & 2\\
& 2 & \mapsto & 3\\
& 3 & \mapsto & 1
\end{array} \] 

Por lo que efectivamente, se trata de un ranking.\\

Para el promedio de goles, podemos definir la función $r : \mathcal{N} \to \R$ definida de la siguiente manera:

\[ \begin{array}{rlll}
c: & \mathcal{N} & \to & \R\\
& 2 & \mapsto & 0.97\\
& 3 & \mapsto & 0.67\\
& 1 & \mapsto & 0.50
\end{array} \] 

Por lo que se cumple que es un rating.\\

Además, escribiremos $i \prec_c j$ cuando el nodo $i \in \mathcal{N}$ aparezca antes que el nodo $j \in \mathcal{N}$ en el ranking $c$.\\

Si volvemos al ejemplo anterior, se tiene que $2 \prec_c 3$, puesto que el nodo $2$ (MES) aparece antes que en el nodo $3$ en el ranking $c$.\\

Existen diversos modos para obtener rankings que se basan en distintos conceptos como mínimos cuadrados o cadenas de Markov. Estos métodos se pueden consultar en~\cite{langville2012s}.

\section{Competitividad en rankings}

Uno de los aspectos más interesantes de los rankings, es la competitividad. Cuando se dispone de una gran cantidad de rankings, es interesante ver cómo se van intercambiando las posiciones los equipos. De esta forma estudiaremos la competitividad en los rankings. Los cambios de posiciones de los equipos darán lugar a dos nuevos tipos de grafos (grafo de competitividad y grafo de competitividad evolutivo). Una introducción a la teoría de grafos se puede encontrar en el Anexo~\ref{teoria_grafos}.\\

Empezamos definiendo la competitividad entre una familia de rankings.

\begin{defi}
Dada una familia $\mathcal{R} = \{c_1, c_2, \dots, c_r\}$ de rankings, decimos que los nodos $(i,j) \in \mathcal{N} \times \mathcal{N}$ compiten si existe $t \in \{1,2,\dots, r-1\}$ tal que $i$ y $j$ intercambian sus posiciones relativas entre rankings consecutivos $c_t$ y $c_{t+1}$.
\end{defi}

\begin{ejemplo}
Si consideramos la familia de rankings $\mathcal{R} = \{c_1, c_2\}$ donde $c_1$ y $c_2$ son, respectivamente

\begin{equation*}
c_1 = \left( \begin{array}{c}
1\\
3\\
2\\
4\\
5
\end{array} \right), \quad
c_2 = \left( \begin{array}{c}
1\\
2\\
4\\
3\\
5
\end{array} \right)
\end{equation*}

Los pares $(1,2)$ y $(1,3)$ no compiten, puesto que los nodos no intercambian sus posiciones relativas entre rankings consecutivos. Sin embargo, los pares $(3,4)$ y $(2,3)$ sí compiten puesto que intercambian sus posiciones relativas entre rankings consecutivos, $c_1$ y $c_2$.
\end{ejemplo}  

\subsection{Grafo de competitividad}

Los pares de nodos que compiten dan lugar a un grafo, que llamaremos grafo de competitividad. Este grafo tiene propiedades interesantes como veremos a continuación.

\begin{defi}[Grafo de competitividad]
Definimos grafo de competitividad de la familia de rankings $\mathcal{R}$, y lo denotaremos como $G_c(\mathcal{R}) = (\mathcal{N}, E_\mathcal{R})$, donde $\mathcal{N}$ es el conjunto de nodos $E_\mathcal{R}$ denota el conjunto de arcos dados por la siguiente regla: existe un arco entre $i$ y $j$ si el par $(i,j)$ compite.
\end{defi}

\begin{ejemplo} \label{ej:grafo_competitividad}
Consideremos la familia de rankings $\mathcal{R} =\{c_1, c_2, c_3, c_4\}$ formada por cuatro clasificaciones ficticias de los equipos Real Madrid (RMA), Barcelona (BAR), Sevilla (SEV) y Valencia (VAL). Las cuatro clasificaciones son las siguientes:

\begin{equation*}
c_1 = \left( \begin{array}{c}
\text{RMA}\\
\text{BAR}\\
\text{SEV}\\
\text{VAL}
\end{array} \right), \quad
c_2 = \left( \begin{array}{c}
\text{RMA}\\
\text{SEV}\\
\text{VAL}\\
\text{BAR}
\end{array} \right), \quad
c_3 = \left( \begin{array}{c}
\text{RMA}\\
\text{BAR}\\
\text{VAL}\\
\text{SEV}
\end{array} \right), \quad
c_4 = \left( \begin{array}{c}
\text{RMA}\\
\text{VAL}\\
\text{BAR}\\
\text{SEV}
\end{array} \right)
\end{equation*}

La Figura~\ref{fig:grafo_competitividad} muestra el grafo de competitividad de la familia de rankings $\mathcal{R}$. El grafo tiene dos componentes conexas: $\{\text{BAR}, \text{SEV}, \text{VAL}\}$ y $\{\text{RMA}\}$. Vemos que hay pares de equipos que compiten más de una vez, por ejemplo, el par $(\text{BAR}, \text{VAL})$ cambian sus posiciones relativas tres veces en los rankings $c_1$, $c_2$, $c_3$ y $c_4$. Esta información no la recoge el grafo de competitividad. Esta información la recogerá el grafo de competitividad evolutivo. 

\begin{figure}[htb]
\centering
\ejemplografocompetitividad
\caption[Grafo de competitividad]{Grafo de competitividad evolutivo de $\mathcal{R}$}
\label{fig:grafo_competitividad}
\end{figure}

\end{ejemplo}

\begin{defi}
Decimos que los nodos $i$, $j$ compiten $k$ veces si $k$ es el número de veces máximo de rankings donde $i$ y $j$ compiten.
\end{defi}

\begin{defi}
El grafo de competitividad evolutivo de $\mathcal{R}$, denotado por $G_c^e(\mathcal{R}) = (\mathcal{N}, E_\mathcal{R}^e)$ es el grafo ponderado con el conjunto $\mathcal{N}$ como nodos y aristas dadas por la siguiente regla: hay una arista entre $i$ y $j$ con peso $k$ si $(i,j)$ compiten $k$ veces.
\end{defi}

\begin{ejemplo}
Consideramos de nuevo la familia $\mathcal{R}$ de rankings del Ejemplo \ref{ej:grafo_competitividad}. El grafo de competitividad de $\mathcal{R}$ se muestra en la Figura~\ref{fig:grafo_competitividad_evolutivo}.

\begin{figure}[htb]
\centering
\ejemplografocompetitividadevolutivo
\caption[Grafo de competitividad evolutivo]{Grafo de competitividad evolutivo de $\mathcal{R}$}
\label{fig:grafo_competitividad_evolutivo}
\end{figure}

\end{ejemplo}

Notar que si eliminamos los pesos del grafo de competitividad evolutivo, obtenemos el grafo de competitividad.

\section{Medidas de competitividad}

Cuando se dispone de más de una familia de rankings, es necesario disponer de una serie de medidas que permitan medir la competitividad en cada uno de los grafos de competitividad. Algunas de estas medidas son grado medio normalizado, fuerza media normalizada y una generalización de la tau de Kendall para una familia de rankings.

\subsection*{Grado medio normalizado}

\begin{defi}
Se define grado medio normalizado de una familia de rankings $\mathcal{R}$, como la suma de todos los grados de los nodos en el grafo de competitividad $G_c(\mathcal{R})$ dividido por la suma sobre todos los nodos de sus grados más altos posibles. Esto es,

\begin{equation}
\mathrm{ND}(\mathcal{R}) = \dfrac{1}{n(n-1)} \sum_{i \in \mathcal{N}} d_i
\end{equation}

donde $d_i$ es el número de nodos adyacentes al nodo  $i$.
\end{defi}

\begin{defi}
Se dice que la familia de rankings $\mathcal{R}$ es más competitiva que la familia de rankings $\mathcal{S}$ con respecto al grado medio normalizado si $\mathrm{ND}(\mathcal{R}) > \mathrm{ND}(\mathcal{S})$.
\end{defi}

\begin{ejemplo}
Consideremos la familia de rankings $\mathcal{R}$ y el grafo de competitividad del Ejemplo~\ref{ej:grafo_competitividad} y el grafo de competitividad $G_c(\mathcal{S})$ de la Figura~\ref{fig:grado_medio}.\\

\begin{figure}[htb]
\centering
\ejemplogradomedio
\caption{Grafo de competitividad de $\mathcal{S}$}
\label{fig:grado_medio}
\end{figure}

El grado medio de las dos familias son:

\begin{eqnarray*}
\mathrm{ND}(\mathcal{R}) = \dfrac{1}{4\cdot 3} (0 + 2 + 2 + 2) = \dfrac{6}{12} = \dfrac{1}{2}\\
\mathrm{ND}(\mathcal{S}) = \dfrac{1}{4\cdot 3} (2 + 3 + 1 + 2) = \dfrac{8}{12} = \dfrac{2}{3}
\end{eqnarray*}

Puesto que $\mathrm{ND}(\mathcal{S}) > \mathrm{ND}(\mathcal{R})$, la familia $\mathcal{S}$ es más competitiva que la familia $\mathcal{R}$.
\end{ejemplo}

\subsection*{Fuerza media normalizada}

\begin{defi}
Se define la fuerza media normalizada de una familia de rankings $\mathcal{R}$ como la suma de todos pesos de las aristas del grafo de competitividad evolutivo $G_c^e(\mathcal{R})$ dividido entre la suma de todas las posibles aristas con sus posibles pesos. Esto es,

\begin{equation}
\mathrm{NS}(\mathcal{R}) = \dfrac{w(E_\mathcal{R}^e)}{\binom{n}{2} (r-1)}
\end{equation} 

donde $w(E_\mathcal{R}^e)$ denota la suma de todos los pesos de las aristas del grafo de competitividad evolutivo $G_c^e(\mathcal{R})$ y $r$ es el número de rankings de la familia $\mathcal{R}$.
\end{defi}

\begin{defi}
Se dice que $\mathcal{R}$ es más competitivo que $\mathcal{S}$ con respecto a la fuerza media normalizada si $\mathrm{NS}(\mathcal{R}) > \mathrm{NS}(\mathcal{S})$.
\end{defi}

\begin{ejemplo}
Consideremos la familia $\mathcal{R}$ del Ejemplo~\ref{ej:grafo_competitividad} y el grafo de competitividad evolutivo de la familia $\mathcal{S}$ compuesta por $4$ rankings de la Figura~\ref{fig:grado_medio}.\\

Calculamos la fuerza media de cada una de las familias de rankings,

\begin{eqnarray*}
\mathrm{NS}(\mathcal{R}) = \dfrac{6}{6 \cdot 3} = \dfrac{1}{3}\\
\mathrm{NS}(\mathcal{S}) = \dfrac{5}{6 \cdot 3} = \dfrac{5}{18}
\end{eqnarray*}

Como $\mathrm{NS}(\mathcal{R}) > \mathrm{NS}(\mathcal{S})$, la familia de rankings $\mathcal{R}$ es más competitiva con respecto a la fuerza media normalizada que la familia $\mathcal{S}$.

\end{ejemplo}

\subsection*{Tau de Kendall generalizada}

\begin{defi}
Llamamos tau de Kendall entre dos rankings $c_1$ y $c_2$ con $n$ elementos cada uno al siguiente valor

\begin{equation}
\tau(c_1, c_2) = \dfrac{\tilde{K}(c_1, c_2) - K(c_1, c_2)}{\binom{n}{2}}
\end{equation} 

donde $\tilde{K}(c_1, c_2)$ es el número de pares $(i,j)$ que no compiten con respecto a $\mathcal{R} = \{c_1, c_2\}$, y $K(c_1, c_2)$ denota el número de pares que compiten en $\mathcal{R}$.
\end{defi}

Esta medida de dos rankings se puede generalizar a una familia con $r$ rankings de la siguiente forma.

\begin{defi}
Se define tau de Kendall generalizada a la familia de rankings $\mathcal{R}$, con $r \geq 2$ de la siguiente manera:

\begin{equation}
\tau(\mathcal{R}) = \dfrac{\tilde{K}(c_1, c_2) - K(c_1, c_2)}{\binom{n}{2}} = 1 - \dfrac{4 |E_\mathcal{R}|}{n(n-1)}
\end{equation}

donde $|E_\mathcal{R}|$ es el número de aristas del grafo de competitividad $G_c(\mathcal{R})$.
\end{defi}

\begin{defi}
Se define tau de Kendall evolutiva de la familia de rankings $\mathcal{R}$ con $r \geq 2$ rankings de la siguiente manera:

\begin{equation}
\tau_e(\mathcal{R}) = 1 - \dfrac{2 w(E_\mathcal{R}^e)}{\binom{n}{2}(r-1)}
\end{equation} 

donde $w(E_\mathcal{R}^e)$ denota la suma de todos los pesos de las aristas del grafo de competitividad evolutivo.
\end{defi}

\begin{defi}
Decimos que la familia de rankings $\mathcal{R}$ es más competitiva con respecto a la tau de Kendall evolutiva que la familia de rankings $\mathcal{S}$ si $\tau_e(\mathcal{R}) < \tau_e(\mathcal{S})$.
\end{defi}

\begin{ejemplo}
Consideremos los grafos evolutivos de las Figuras~\ref{fig:grafo_competitividad_evolutivo}~y~\ref{fig:grado_medio}. Calculamos la tau de Kendall para la familia de rankings $\mathcal{R}$.

\begin{eqnarray}
\tau(\mathcal{R}) = 1 - \dfrac{1}{2} = \dfrac{1}{2}\\
\tau_e(\mathcal{R}) = 1 - \dfrac{112}{12} = 1
\end{eqnarray}

De la misma forma para la familia $\mathcal{S}$,

\begin{eqnarray}
\tau(\mathcal{S}) = 1 - \dfrac{16}{12} = -\dfrac{1}{3}\\
\tau_e(\mathcal{S}) = 1 - \dfrac{8}{9} = \dfrac{1}{9}
\end{eqnarray}

Como $\tau_e(\mathcal{S}) < \tau_e(\mathcal{R})$ se tiene que la familia $\mathcal{S}$ es más competitiva que la familia $\mathcal{S}$.

\end{ejemplo}

\subsection*{Diámetro del grafo de competitividad evolutivo}

\begin{defi}
Se define diámetro del grafo de competitividad evolutivo de la familia de rankings $\mathcal{R}$ de la siguiente forma:

\begin{equation}
\mathrm{diam}(\mathcal{R}) = \max \{ d(i,j) : i \neq j \}
\end{equation}

donde $d(i,j)$ es la distancia entre el nodo $i$ y el nodo $j$.
\end{defi}

\begin{defi}
Diremos que la familia de rankings $\mathcal{R}$ es más competitiva que la familia de rankings $\mathcal{S}$ con respecto al diámetro del grafo de competitividad evolutivo si $\mathrm{diam}(\mathcal{R}) < \mathrm{diam}(\mathcal{S})$.
\end{defi}

\begin{ejemplo}
Consideremos los grafos evolutivos de las Figuras~\ref{fig:grafo_competitividad_evolutivo}~y~\ref{fig:grado_medio}. Calculemos el diámetro para la familia de rankings $\mathcal{R}$ y la familia $\mathcal{S}$.\\

Necesitamos calcular la distancia entre todos los pares de vértices y coger el máximo. Para ello, aplicamos el algoritmo de Floyd-Warshall.\\

En el caso de la familia $\mathcal{R}$ de rankings, obtenemos que la matriz de distancias tras la ejecución del algoritmo, es:

\begin{equation*}
\left(\begin{array}{cccc}
0 & \infty & \infty & \infty\\
\infty & 0 & 3 & 2\\
\infty & 3 & 0 & 1\\
\infty & 2 & 1 & 0 
\end{array}\right)
\end{equation*}

Por lo tanto,

\begin{equation*}
\mathrm{diam}(\mathcal{R}) = \infty
\end{equation*}

Para la familia $\mathcal{S}$ de rankings, tenemos que la matriz de distancias es la siguiente:

\begin{equation*}
\left(\begin{array}{cccc}
0 & 1 & 2 & 2\\
1 & 0 & 1 & 1\\
2 & 1 & 0 & 2\\
2 & 1 & 2 & 0 
\end{array}\right)
\end{equation*}

Por tanto, 

\begin{equation*}
\mathrm{diam}(\mathcal{S}) = 2
\end{equation*}

Como $\mathrm{diam}(\mathcal{S}) < \mathrm{diam}(\mathcal{R})$, la familia $\mathcal{S}$ es más competitiva que la familia $\mathcal{R}$.

\end{ejemplo}

\subsection*{Eficiencia del grafo de competitividad evolutivo}

\begin{defi}
Se define eficiencia del grafo de competitividad evolutivo de la familia de rankings $\mathcal{R}$ de la siguiente forma:

\begin{equation}
\mathrm{E}(\mathcal{R}) = \dfrac{1}{\binom{n}{2}} \sum_{i\neq j} \dfrac{1}{d(i,j)}
\end{equation}

donde $d(i,j)$ es la distancia entre el nodo $i$ y el nodo $j$.
\end{defi}

\begin{defi}
Diremos que la familia de rankings $\mathcal{R}$ es más competitiva que la familia de rankings $\mathcal{S}$ con respecto a la eficiencia del grafo de competitividad evolutivo si $\mathrm{E}(\mathcal{R}) > \mathrm{E}(\mathcal{S})$.
\end{defi}

\begin{ejemplo}
Consideremos los grafos evolutivos de las Figuras~\ref{fig:grafo_competitividad_evolutivo}~y~\ref{fig:grado_medio}. Calculemos la eficiencia para la familia de rankings $\mathcal{R}$ y la familia $\mathcal{S}$.\\

En el ejemplo anterior ya calculamos la matriz de distancias de cada una de las familias, por lo que no es necesario volver a calcularlas.\\

Por tanto la eficiencia de la familia $\mathcal{R}$ es:

\begin{equation*}
\mathrm{E}(\mathcal{R}) = \dfrac{1}{\binom{4}{2}} \left( \dfrac{1}{\infty} + \dfrac{1}{\infty} + \dfrac{1}{\infty} + \dfrac{1}{\infty} + \dfrac{1}{3} + \dfrac{1}{2} + \dfrac{1}{\infty} + \dfrac{1}{3} + \dfrac{1}{1} + \dfrac{1}{\infty} + \dfrac{1}{2} + \dfrac{1}{1}\right) = \dfrac{10}{18} = \dfrac{5}{9}  
\end{equation*}

De forma análoga, se obtiene que la eficiencia de la familia $\mathcal{S}$ es:

\begin{equation*}
\mathrm{E}(\mathcal{S}) = \dfrac{1}{\binom{4}{2}} \left( \dfrac{1}{1} + \dfrac{1}{2} + \dfrac{1}{2} + \dfrac{1}{1} + \dfrac{1}{1} + \dfrac{1}{1} + \dfrac{1}{2} + \dfrac{1}{1} + \dfrac{1}{2} + \dfrac{1}{2} + \dfrac{1}{1} + \dfrac{1}{2}\right) = \dfrac{9}{6} = \dfrac{3}{2} 
\end{equation*}

\subsection*{Longitud del camino característico del grafo de competitividad evolutivo}

\begin{defi}
Se define longitud del camino característico\footnote{Traducción de characteristic path length} del grafo de competitividad evolutivo de la familia de rankings $\mathcal{R}$ de la siguiente forma:

\begin{equation}
\mathrm{L}(\mathcal{R}) = \dfrac{1}{\binom{n}{2}} \sum_{i\neq j} d(i,j)
\end{equation}

donde $d(i,j)$ es la distancia entre el nodo $i$ y el nodo $j$.
\end{defi}

\begin{defi}
Diremos que la familia de rankings $\mathcal{R}$ es más competitiva que la familia de rankings $\mathcal{S}$ con respecto a la eficiencia del grafo de competitividad evolutivo si $\mathrm{L}(\mathcal{R}) < \mathrm{L}(\mathcal{S})$.
\end{defi}

\begin{ejemplo}
Consideremos los grafos evolutivos de las Figuras~\ref{fig:grafo_competitividad_evolutivo}~y~\ref{fig:grado_medio}. Calculemos la eficiencia para la familia de rankings $\mathcal{R}$ y la familia $\mathcal{S}$.\\

En el ejemplo anterior ya calculamos la matriz de distancias de cada una de las familias, por lo que no es necesario volver a calcularlas.\\

Por tanto la eficiencia de la familia $\mathcal{R}$ es:

\begin{equation*}
\mathrm{L}(\mathcal{R}) = \dfrac{1}{\binom{4}{2}} \left( \infty + \infty + \infty + \infty + 3 + 2 + \infty + 3 + 2 + \infty + 3 + 1 + \infty + 2 + 1 \right) = \infty
\end{equation*}

De forma análoga, se obtiene que la eficiencia de la familia $\mathcal{S}$ es:

\begin{equation*}
\mathrm{L}(\mathcal{S}) = \dfrac{1}{\binom{4}{2}} \left( 1 + 2 + 2 + 1 + 1 + 1 + 2 + 1  + 2 + 2 + 1 + 2  \right) =  \dfrac{17}{6} 
\end{equation*}

Puesto que $\mathrm{L}(\mathcal{S}) < \mathrm{L}(\mathcal{R})$, la familia de rankings $\mathcal{S}$ es más competitiva que la familia de rankings $\mathcal{R}$.

\end{ejemplo}