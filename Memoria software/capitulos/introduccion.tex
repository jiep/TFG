\chapter{Introducción}

En la actualidad, vivimos rodeados de rankings de todo tipo: 
\begin{itemize}
\item Sociales, como los barómetros del CIS donde se miden distintos parámetros que afectan a la sociedad española (valoración de los políticos, ideología política, tema que más preocupan, etc), rankings de universidades que miden la calidad de la educación, la calidad del profesorado, etc., como el Academic Ranking of World University, el Times Higher Educaction, el Quacquarelli Symonds, etc.
\item Económicos, como el Producto Interior Bruto (PIB), que mide el conjunto de bienes y servicios finales producidos en un país durante un año, la renta per cápita, que mide la relación entre el PIB y la población de un país, la prima de riesgo, que mide la diferencia entre el interés que se paga por la deuda de un país y el que se paga por la de otro, etc.
\item Deportivos, como el ranking FIFA en fútbol, el ranking FIBA en baloncesto, el ranking ATP/WTA en tenis, el Draft de la NBA, etc.
\end{itemize}

Como podemos observar, existen miles de rankings de casi cualquier tema que podamos imaginar, ya que en definitiva, se trata de ordenar unos determinados elementos según unos determinados parámetros que es lo que denominaremos rating, y una vez ordenados estos elementos se obtiene un ranking.\\

Unos de los temas más interesantes es el poder analizar la competitividad de estos rankings. Veremos que se puede analizar la competitividad estudiando los cambios de posición entre dos rankings consecutivos. Esto dará lugar a dos tipos de grafos (grafo de competitividad y grafo de competitividad evolutivo). También veremos una serie de medidas para poder comparar esta competitividad entre dos familias de rankings.\\

Una vez entendido el concepto de competitividad y las definiciones de las distintas medidas de competitividad veremos el diseño de una aplicación que permitirá estudiar la competitividad de las distintas temporadas de la Liga BBVA así como ver distintas estadísticas relativas a cada uno de los equipos que ha jugado en primera división.\\
Esta aplicación consta de un parser que extrae los datos de la página de la Liga de Fútbol Profesional y una aplicación SPA implementada en AngularJS.\\

Por último, usaremos todo lo anterior para analizar y comparar la competitividad de las últimas cuatro temporadas de la Liga BBVA.