\chapter{Conclusiones}

Durante esta memoria, hemos visto cómo estudiar la competitividad en una familia de rankings y cómo poder compararlo con otras familias de rankings con la definición de algunas medidas de competitividad como lo son la fuerza y el grado medio normalizado, entre otros.\\

Además, hemos diseñado e implementado una aplicación que extrae datos de la página de la Liga de Fútbol Profesional de todas las temporadas desde 1928 hasta la actualidad y estadísticas de todos los equipos que han jugado en primera división, y calcula el grafo y las medidas de competitividad. Estos datos son intercambiados entre el cliente y el servidor mediante una API REST.\\

Por último, con la ayuda de la aplicación, hemos analizado la competitividad de las últimas cuatro temporadas de la Liga BBVA, quedando como la más competitiva la última disputa, es decir, la temporada 2014-2015.

\section{Mejoras y futuro trabajo}

\begin{itemize}
\item Posibilidad de ver la competitividad a lo largo de las jornadas, es decir, ver las medidas y el grafo de competitividad en una jornada concreta durante la temporada.

\item Añadir un módulo de predicción para ver cómo evolucionará la competitividad a lo largo del tiempo.

\item Posibilidad de implementar una aplicación con Ionic Framework.

\item Añadir un módulo de comparación, que permita comparar dos o más temporadas distintas.

\item Añadir un módulo que permita a los usuarios definir sus propios parsers para añadir nuevas ligas, o incluso, otras temáticas.

\item Mostrar el gráfico histórico de líneas invertido, es decir, que las posiciones más bajas del ranking estén en lo más alto del gráfico.
\end{itemize}
