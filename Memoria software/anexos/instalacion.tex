\chapter{Manual de instalación}

Para instalar la aplicación necesitaremos seguir los siguientes pasos\footnote{De ahora en adelante, supondremos que estamos en un entorno Linux (en particular, un entorno Ubuntu o derivados)}:

\begin{enumerate}
\item Instalar las dependencias con los siguientes comandos:

\begin{minted}{bash}
sudo apt-get install apache2
sudo apt-get install php5 php5-cli php5-mysql
sudo apt-get install php5-curl
sudo apt-get install mysql-client mysql-server
\end{minted}

Durante la instalación de la última dependencia se nos pedirá la contraseña del usuario \emph{root} de la base de datos MySQL.

\item Creación de un nuevo usuario para la base de datos. 

Entramos en MySQL con como root con el comando

\begin{minted}{bash}
mysql -u root -p
\end{minted}

Nos pedirá la contraseña del usuario \emph{root} que hemos configurado durante el paso anterior.

\begin{enumerate}
\item Escribimos el siguiente comando para crear un nuevo usuario:

\begin{minted}{sql}
CREATE USER 'rankings'@'localhost' IDENTIFIED by '1234';
GRANT ALL PRIVILEGES ON * . * TO 'rankings'@'localhost';
FLUSH PRIVILEGES;
\end{minted}

donde \emph{1234} es la contraseña del usuario \emph{rankings}.
\end{enumerate}

\item Habilitar el módulo \emph{mod\_rewrite} de Apache.

\begin{enumerate}
\item Escribimos los siguientes comandos en la Terminal

\begin{minted}{bash}
sudo a2enmod rewrite
sudo service apache2 restart
\end{minted}


\item Ejecutamos los comandos

\begin{minted}{bash}
sudo nano /etc/apache2/apache2.conf
\end{minted}

y buscamos en el archivo \emph{<Directory /var/www/>}.

\item Sustituimos el contenido de \emph{<Directory /var/www/>} por lo siguiente

\begin{minted}{bash}
<Directory /var/www/>
	Options Indexes FollowSymLinks
	AllowOverride All
	Require all granted
</Directory>
\end{minted}

\item Reiniciamos Apache con

\begin{minted}{bash}
sudo service apache2 restart
\end{minted}

\end{enumerate}

\item Instalación de Bower

\begin{enumerate}
\item Descargar node desde su página oficial \url{https://nodejs.org}
\item Instalar npm (si no viene instalado junto a node)
\item Ejecutar el comando 

\begin{minted}{bash}
npm install -g bower
\end{minted}

\item Para comprobar que tenemos instalado Bower ejecutar 

\begin{minted}{bash}
bower -v
\end{minted} 

Si nos aparece el número de versión, está instalado.

\item Ir a la carpeta web y ejecutar el comando 

\begin{minted}{bash}
bower install
\end{minted} 

Nos instalará todas las dependencias necesarias.
\end{enumerate}

\item Instalación de Composer

\begin{enumerate}
\item Ejecutar los siguientes comandos

\begin{minted}{bash}
curl -sS https://getcomposer.org/installer | php
mv composer.phar /usr/local/bin/composer
\end{minted} 

\item Ir a la carpeta \emph{web} y ejecutar el comando 

\begin{minted}{bash}
composer install
\end{minted}

Nos instalará todas las dependencias necesarias.

\end{enumerate}

\item Creación de las tablas de la base de datos y carga de los datos

\begin{enumerate}
\item Vamos a la carpeta \emph{parser} y escribimos

\begin{minted}{bash}
mysql -u rankings -p
source rankings.sql
source datos.sql
\end{minted}

\end{enumerate}

\item Movemos la carpeta de la aplicación hacia la carpeta \emph{var/www/html} del servidor Apache.

\item Abrimos un navegador y escribimos \emph{localhost}. Si todo ha ido bien, tendremos la aplicación funcionando.

\end{enumerate}