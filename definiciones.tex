\section{Grafos}

\begin{defi}
\textcolor{red}{Grafo}
\end{defi}

\begin{defi}
\textcolor{red}{Orden de un grafo}
\end{defi}

\begin{defi}
\textcolor{red}{Grado de un grafo}
\end{defi}

\begin{defi}
La fuerza de un nodo en un grafo ponderado es la suma de los pesos de los arcos incidentes.
\end{defi}

\begin{defi}
Una camarilla\footnote{Traducción de clique} es un conjunto de nodos mutuamente conectados entre ellos.
\end{defi}

\section{Rankings}

\begin{defi}
Dado un conjunto $\mathcal{N} = \{1,\dots,n\}$ que llamaremos nodos, definimos un ranking $c$ de $\mathcal{N}$ como cualquier biyección $c: \mathcal{N} \to \mathcal{N}$.\\
Escribiremos $i \prec_c j$ cuando el nodo $i$ aparezca antes que el nodo $j$ en el ranking $c$. 
\end{defi}

\begin{defi}
Dada una familia finita $\mathcal{R} = \{c_1,c_2,\dots, c_r\}$ de rankings, decimos que el par de nodos $(i,j) \in \mathcal{N}$ ``compiten'' si existe $t \in \{1,2,\dots, r-1\}$ tal que $i$ y $j$ intercambian sus posiciones relativas entre dos rankings consecutivos $c_t$ y $c_{t+1}$.
\end{defi}

\begin{defi}
Definimos el grafo de competitividad de la familia de rankings $R$, denotado como $G_c(\mathcal{R}) = (\mathcal{N}, E_\mathcal{R})$, donde $E_\mathcal{R}$ denota el conjunto de arcos, como el grafo no dirigido con nodos $\mathcal{N}$ y arcos dados por la siguiente regla: hay un enlace entre $i$ y $j$ si $(i,j)$ ``compiten''.
\end{defi}

\begin{defi}
Decimos que dos nodos $i$, $j$ ``compiten'' $k$ veces, si $k$ es el número máximo de rankings donde $i$ y $j$ ``compiten''.
\end{defi}

\begin{defi}
Definimos el grafo de competitividad evolutivo de $\mathcal{R}$, denotado como $G_c^e(\mathcal{R}) = (\mathcal{N}, E_\mathcal{R}^e)$, como el grafo no dirigido ponderado con nodos $\mathcal{N}$ y arcos dados por la siguiente regla: hay un arco entre $i$ y $j$ etiquetado con peso $k$ si $(i,j)$ compiten $k$ veces.  
\end{defi}

\begin{nota}
Notar que la red no ponderada detrás del grafo ponderado $G_c^e(\mathcal{R})$ es $G_c(\mathcal{R})$.
\end{nota}

\begin{nota}
Notar que el orden de los rankings es fundamental en el cálculo de los pesos del grafo de competitividad evolutivo, aunque no tiene influencia en el grafo de competitividad (no ponderado). 
\end{nota}

\begin{defi}
Definimos el grado medio normalizado (normalized mean degree) de una familia de rankings $\mathcal{R}$ como la suma de todos los grados del grafo competitivo $G_c(\mathcal{R})$, dividido por la suma de todos los nodos de su mayor grado

\[ND(\mathcal{R}) = \dfrac{1}{n(n-1)} \sum_{i \in \mathcal{N}} k_i\]

donde $k_i$ es el número de nodos adyacentes al nodo $i$. 
\end{defi}

\begin{ob}
Este parámetro nos da una idea global del número relativo de veces entre dos posibles competidores han intercambiado sus posiciones relativas a lo largo de la correspondiente familia de rankings.
\end{ob}

\begin{defi}
Decimos que el ranking $\mathcal{R}$ es más competitivo que el ranking $\mathcal{S}$ con respecto al grado medio normalizado si $ND(\mathcal{R}) > ND(\mathcal{S})$.
\end{defi}

\begin{defi}
Definimos la fuerza media generalizada de una familia de rankings $\mathcal{R}$ como la suma de todos los pesos de los arcos del grafo de competitividad evolutivo $G_c^e(\mathcal{R})$ dividido entre la suma de todos los posibles arcos con sus mayores pesos

\[ NS(\mathcal{R}) = \dfrac{w(E_\mathcal{R}^e)}{\dbinom{n}{2}(r-1)} \]

donde  $w(E_\mathcal{R}^e)$ denota la suma de todos los pesos de los arcos del grafo de competitividad evolutivo.
\end{defi}

\begin{defi}
Decimos que $\mathcal{R}$ es más competitivo que $\mathcal{S}$ con respecto a la fuerza media normalizada si $NS(\mathcal{R}) > NS(\mathcal{S})$.
\end{defi}

\begin{nota}
El coeficiente de ``clustering'' mide cuántos nodos tienden a agruparse\footnote{Traducción de cluster.}
\end{nota}

\begin{defi}
El coeficiente de ``clustering'' $C_i$ de un nodo $i$ se define como 

\[ C_i = \dfrac{e_i}{\dbinom{k_i}{2}} \]

donde $k_i$ es el número de vecinos del nodo $i$, $e_i$ es el número de pares conectados entre los vecinos de $i$, y $\binom{k_i}{2}$ representa todos los posibles entre los vecinos de $i$.
\end{defi}

\begin{defi}
Dada una familia de rankings $\mathcal{R}$, el coeficiente de ``clustering'' de $\mathcal{R}$ de $\mathcal{R}$ es la media de los coeficientes de ``clustering'' de los nodos del grafo de competitividad evolutivo $G_e(\mathcal{R})$, es decir,

\[ C(\mathcal{R}) = \dfrac{1}{n} \sum_{i \in \mathcal{N}} C_i \]
\end{defi}

\begin{defi}
Decimos que $\mathcal{R}$ es más competitivo que $\mathcal{S}$ con respecto al coeficiente de ``clustering'' si $C(\mathcal{R}) > C(\mathcal{S})$.
\end{defi}

\begin{defi}
Dados dos rankings $c_1$ y $c_2$ de un conjunto $\mathcal{N}$ de nodos de $n$ elementos, el coeficiente $\tau$ de Kendall se define como

\[ \tau(c_1, c_2) = \dfrac{\tilde{K}(c_1, c_2) - K(c_1, c_2)}{\dbinom{n}{2}} \]

donde $\tilde{K}(c_1, c_2)$ denota el número de pares $(i,j)$ que no ``compiten'' con respecto a $\mathcal{R} = \{c_1, c_2\}$, y $K(c_1, c_2)$ denota el número de pares $(i,j)$ que ``compiten''. 
\end{defi}

\begin{nota}
Notar que $\binom{n}{2}$ es el número de todos los posibles pares de nodos $(i,j)$. Si consideramos el grafo de competitividad $G_c(\mathcal{R})$ con respecto a $\mathcal{R} = \{c_1, c_2\}$, entonces $K(c_1, c_2) = |E_\mathcal{R}|$,el número de arcos de $G_c(\mathcal{R})$, y $\tilde{K}(c_1, c_2) = \binom{n}{2} - |E_\mathcal{R}|$, entonces

\[ \tau(c_1,c_2) = 1 - \dfrac{2 |E_\mathcal{R}|}{\dbinom{n}{2}} = 1- \dfrac{4 |E_\mathcal{R}|}{n(n-1)} \]

El número de arcos $E_\mathcal{R}$ del grafo de competitividad $G_c(\mathcal{R})$ de una familia $\mathcal{R}$ está relacionado con el coeficiente de correlación de Kendall de dos rankings: si denotamos por $E(c_1, c_2)$ los arcos del grafo de competitividad de la familia $\{c_1, c_2\}$, tenemos que

\[ |E_\mathcal{R}| \geq \max_{c_1, c_2 \in \mathcal{R}} |E(c_1, c_2)| \geq \dfrac{n(n-1)}{4}(1 - \min \tau(c_1, c_2)) \]

y esta desigualdad es, en efecto, una igualdad cuando $r=2$. De forma similar, si $E_\mathcal{R} = \cup_{c_1, c_2 \in \mathcal{R}} E(c_1, c_2)$, entonces

\[ |E_\mathcal{R}| \leq \sum_{c_1, c_2 \in \mathcal{R}} |E(c_1, c_2)| = \dfrac{n(n-1)}{4} \left( \dbinom{r}{2} - \sum_{c_1, c_2 \in \mathcal{R}} \tau(c_1, c_2) \right) \]

Esta desigualdad se hace igualdad cuando $r=2$.
\end{nota}

\begin{defi}
Podemos definir un coeficiente $\tau(\mathcal{R})$ de correlación de Kendall de una familia $\mathcal{R}$ de $r \geq 2$ rankings: siguiendo la idea original definición (número de pares $\tilde{K}(\mathcal{R})$ que no compiten, menos pares que compiten $K(\mathcal{R})$), dividido entre entre el número de todos los posibles pares $\binom{n}{2}$, tenemos

\[ \tau(\mathcal{R}) = \dfrac{\tilde{K}(c_1,c_2) - K(c_1, c_2)}{\dbinom{n}{2}} = 1 - \dfrac{2 |E_\mathcal{R}|}{\dbinom{n}{2}} = 1 - \dfrac{4 |E_\mathcal{R}|}{n(n-1)} \]

También podemos construir un coeficiente de correlación evolutivo de Kendall $\tau_e(\mathcal{R})$ teniendo en cuenta el número de veces que cada par de nodos compiten. Definimos

\[ \tau_e(\mathcal{R}) = 1 - \dfrac{2w(E_\mathcal{R}^e)}{\dbinom{n}{2}(r-1)} \] 

donde $w(E_\mathcal{R}^e)$ denota la suma de todos los pesos de los arcos del grafo de competitividad evolutivo. El denominador  $\binom{n}{2}(r-1)$ representa la suma de todos los posibles arcos de sus mayores pesos.
\end{defi}

\begin{nota}
El coeficiente evolutivo de Kendall de una familia de rankings $\mathcal{R}$ se relaciona directamente con la fuerza media normalizada 

\[\tau_e(\mathcal{R}) = 1 - 2NS(\mathcal{R}) \]
\end{nota}

\begin{defi}
Decimos que $\mathcal{R}$ es más competitivo que $\mathcal{S}$ con respecto al coeficiente de Kendall si $\tau_e(\mathcal{R}) < \tau_e(\mathcal{S})$.
\end{defi}

\begin{nota}
Notar que un coeficiente más pequeño de Kendall $\tau_e(\mathcal{R})$ hace que $\mathcal{R}$ sea más competitivo.
\end{nota}

\begin{defi}
La desviación típica de victorias $w_i$ que cada equipo ha logrado a lo largo en una temporada puede ser definida así:

\[ \sigma = \sqrt{\dfrac{1}{n} \sum_{i=1}^{n}\left(w_i - \dfrac{1}{2}\right)^2 } \]

donde $n$ es el número de equipos.
\end{defi}

\begin{nota}
En general, cuanto mayor es $\sigma$, menor es el equilibrio competitivo (y mayor es la desigualdad competitiva).\\

En una liga donde cada equipo tiene la misma probabilidad de ganar un partido, $\sigma = 0$, y por tanto, un incremento de $\sigma$ significa un decremento en el equilibrio competitivo.
\end{nota}

\begin{defi}
Se dice que $r$ rankings han aumentado su competitividad cuando ha disminuido su $\sigma$.
\end{defi}

\begin{defi}
Definimos NAMSI como 
\[ NAMSI =  \sqrt{\dfrac{\sum\limits_{i=1}^{n} \left( w_i - \dfrac{1}{2} \right)^2}{\sum\limits_{i=1}^{n} \left(w_{i, \max} -\dfrac{1}{2}\right)^2}} \]

donde $w_{i, \max}$ es el ratio de victorias del equipo $i$ cuando hay predictibilidad completa: equipo $1$ gana todos sus partidos, equipo $2$ gana todos sus partidos menos $2$ (como local y visitante),..., equipo $n$ pierde todos sus partidos. 

\end{defi}

\begin{defi}
El índice HICB se define como

\[ HICB = 100n \sum_{i=1}^{n} s_i^2 \]

donde $s_i$ es el ratio de puntos conseguidos en una temporada por el equipo $i$.
\end{defi}

\begin{nota}
Un incremento del coeficiente HICB significa una disminución del equilibrio competitivo.
\end{nota}

\begin{defi}
Se define el Churn como

\[ C_t = \dfrac{1}{n} \sum_{i=1}^{n} |c_{i,t} - c_{i,t-1}| \]

donde $c_{i,t}$ es el ranking del equipo $i$ in la temporada $t$.
\end{defi}

\begin{defi}
Se difine el Churn Ajustado como

\[ C_{ad} = \dfrac{C_t}{C_{t, \max}} \]

donde $C_{t, \max}$ es el máximo valor de $C_t$ dada una liga de $n$ equipos.
\end{defi}

\begin{nota}
Este indice indica equilibrio competitivo entre temporadas.\\

Si dos rankings son iguales, $C_t = 0$.\\

Se considera que valores grandes de $C_t$ implican altos valores en el equilibrio competitivo. 
\end{nota}


\begin{nota}
Notar que el Churn Ajustado es una ``footrule'' de Spearman generalizada $D(c_t, c_{t+1})$. De hecho, $D(c_t, c_{t+1}) = nC_t$ 
\end{nota}

\end{document}