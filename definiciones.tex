\section{Grafos}

\begin{defi}
\textcolor{red}{Grafo}
\end{defi}

\begin{defi}
\textcolor{red}{Orden de un grafo}
\end{defi}

\begin{defi}
\textcolor{red}{Grado de un grafo}
\end{defi}

\begin{defi}
La fuerza de un nodo en un grafo ponderado es la suma de los pesos de los arcos incidentes.
\end{defi}

\section{Rankings}

\begin{defi}
Dado un conjunto $\mathcal{N} = \{1,\dots,n\}$ que llamaremos nodos, definimos un ranking $c$ de $\mathcal{N}$ como cualquier biyección $c: \mathcal{N} \to \mathcal{N}$.\\
Escribiremos $i \prec_c j$ cuando el nodo $i$ aparezca antes que el nodo $j$ en el ranking $c$. 
\end{defi}

\begin{defi}
Dada una familia finita $\mathcal{R} = \{c_1,c_2,\dots, c_r\}$ de rankings, decimos que el par de nodos $(i,j) \in \mathcal{N}$ ``compiten'' si existe $t \in \{1,2,\dots, r-1\}$ tal que $i$ y $j$ intercambian sus posiciones relativas entre dos rankings consecutivos $c_t$ y $c_{t+1}$.
\end{defi}

\begin{defi}
Definimos el grafo de competitividad de la familia de rankings $R$, denotado como $G_c(\mathcal{R}) = (\mathcal{N}, E_\mathcal{R})$, donde $E_\mathcal{R}$ denota el conjunto de arcos, como el grafo no dirigido con nodos $\mathcal{N}$ y arcos dados por la siguiente regla: hay un enlace entre $i$ y $j$ si $(i,j)$ ``compiten''.
\end{defi}

\begin{defi}
Decimos que dos nodos $i$, $j$ ``compiten'' $k$ veces, si $k$ es el número máximo de rankings donde $i$ y $j$ ``compiten''.
\end{defi}

\begin{defi}
Definimos el grafo de competitividad evolutivo de $\mathcal{R}$, denotado como $G_c^e(\mathcal{R}) = (\mathcal{N}, E_\mathcal{R}^e)$, como el grafo no dirigido ponderado con nodos $\mathcal{N}$ y arcos dados por la siguiente regla: hay un arco entre $i$ y $j$ etiquetado con peso $k$ si $(i,j)$ compiten $k$ veces.  
\end{defi}

\begin{nota}
Notar que la red no ponderada detrás del grafo ponderado $G_c^e(\mathcal{R})$ es $G_c(\mathcal{R})$.
\end{nota}

\begin{nota}
Notar que el orden de los rankings es fundamental en el cálculo de los pesos del grafo de competitividad evolutivo, aunque no tiene influencia en el grafo de competitividad (no ponderado). 
\end{nota}

\begin{defi}
Definimos el grado medio normalizado (normalized mean degree) de una familia de rankings $\mathcal{R}$ como la suma de todos los grados del grafo competitivo $G_c(\mathcal{R})$, dividido por la suma de todos los nodos de su mayor grado

\[ND(\mathcal{R}) = \dfrac{1}{n(n-1)} \sum_{i \in \mathcal{N}} k_i\]

donde $k_i$ es el número de nodos adyacentes al nodo $i$. 
\end{defi}

\begin{ob}
Este parámetro nos da una idea global del número relativo de veces entre dos posibles competidores han intercambiado sus posiciones relativas a lo largo de la correspondiente familia de rankings.
\end{ob}

\begin{defi}
Decimos que el ranking $\mathcal{R}$ es más competitivo que el ranking $\mathcal{S}$ con respecto al grado medio normalizado si $ND(\mathcal{R}) > ND(\mathcal{S})$.
\end{defi}

\begin{defi}
\textcolor{red}{Normalized mean strength}
\end{defi}
\end{document}