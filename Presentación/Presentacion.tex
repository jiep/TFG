\documentclass[10pt,hyperref={unicode}]{beamer}
\usetheme{Madrid}
\setbeamertemplate{caption}[numbered]

\usepackage{fontspec}
%\usepackage[unicode=true]{hyperref}
\usepackage[spanish, es-nodecimaldot, es-tabla]{babel}
\usepackage{amsmath}
\usepackage{amsfonts}
\usepackage{amssymb}
\usepackage{graphicx}
\usepackage{subfig}
\usepackage{url}
\usepackage{booktabs}
\usepackage{adjustbox}

\usepackage{amsmath}
\usepackage{xparse}
\usepackage{amsthm}
\usepackage{tikz}

% Seno
\def\sen{\mathop{\mbox{\normalfont sen}}\nolimits}

% Seno hiperbólico
\def\senh{\mathop{\mbox{\normalfont senh}}\nolimits}

% Arcoseno
\def\arcsen{\mathop{\mbox{\normalfont arcsen}}\nolimits}

% Gradiente
\def\grad{\mathop{\mbox{\normalfont grad}}\nolimits}

% Argumento principal
\def\Arg{\mathop{\mbox{\normalfont Arg}}\nolimits}

% Logaritmo complejo
\def\Log{\mathop{\mbox{\normalfont Log}}\nolimits}

% Rotacional
\def\rot{\mathop{\mbox{\normalfont rot}}\nolimits}

% Divergencia
\def\diver{\mathop{\mbox{\normalfont div}}\nolimits}

% Longitud
\def\longi{\mathop{\mbox{\normalfont long}}\nolimits}

% Índice de una curva
\def\ind{\mathop{\mbox{\normalfont Ind}}\nolimits}

% Norma
\newcommand{\norm}[1]{\lVert #1 \rVert}

% Derivada parcial
\newcommand{\derpar}[1]{\dfrac{\partial}{\partial #1}}

% Derivada parcial con función
\newcommand{\derparcial}[2]{\dfrac{\partial #1}{\partial #2}}

% Derivada parcial de orden n con funcion
\newcommand{\derparcialn}[3]{\dfrac{\partial^{#3} #1}{\partial #2^{#3}}}

% Integral doble en [a,b] x [c,d]
\newcommand{\intdob}[4]{\displaystyle \int_{#1}^{#2} \int_{#3}^{#4}}

% Integral triple en [a,b] x [c,d] x [e,f]
\newcommand{\inttri}[6]{\displaystyle \int_{#1}^{#2} \int_{#3}^{#4} \int_{#5}^{#6}}

% Números complejos
\def\C{\ensuremath{\mathbb{C}}}

% Números reales
\def\R{\ensuremath{\mathbb{R}}}

% Números racionales
\def\Q{\ensuremath{\mathbb{Q}}}

% Números enteros
\def\Z{\ensuremath{\mathbb{Z}}}

% Números naturales
\def\N{\ensuremath{\mathbb{N}}}

% Teorema
\newtheorem{teo}{Teorema}

% Corolario
\newtheorem{cor}{Corolario}

% Proposición
\newtheorem{prop}{Proposición}

% Lema
\newtheorem{lema}{Lema}

% Definición
\newtheorem{defi}{Definición}

% Observaciones
\newtheorem*{obs}{Observaciones}

\newtheorem*{ob}{Observación}

\newtheorem*{nota}{Nota}

\newtheorem*{problem}{Problema}

\newtheorem*{pregunta}{Pregunta}


\NewDocumentCommand{\overarrow}{O{=} O{\uparrow} m}{%
  \overset{\makebox[0pt]{\begin{tabular}{@{}c@{}}#3\\[0pt]\ensuremath{#2}\end{tabular}}}{#1}
}
\NewDocumentCommand{\underarrow}{O{=} O{\downarrow} m}{%
  \underset{\makebox[0pt]{\begin{tabular}{@{}c@{}}\ensuremath{#2}\\[0pt]#3\end{tabular}}}{\ensuremath{#1}}
}

\newcommand{\contradiction}{%
\begin{tikzpicture}[rotate=45,x=0.5ex,y=0.5ex]
\draw[color=red, line width=.1ex] (0,2) -- (3,2) (0,1) -- (3,1) (1,3) -- (1,0) (2,3) -- (2,0);
\end{tikzpicture}
}

\usepackage{pgf}
\usepackage{tikz}
\usetikzlibrary{arrows,automata}

\newcommand{\ejemplomarkov}{
\begin{tikzpicture}[->,>=stealth',shorten >=1pt,auto,node distance=5cm,semithick]
  \tikzstyle{every state}=[draw=black,text=black]

  \node[state]         (A)                    {$A$};
  \node[state]         (B) [below left of=A]  {$B$};
  \node[state]         (C) [below right of=A] {$C$};
  
  \path (A) edge  [loop above] node {$0.65$} (A)
            edge  [bend left]  node {$0.3$}  (B)
            edge  [bend left]  node {$0.05$} (C)
        (B) edge  [loop below] node {$0.75$} (B)
            edge  [bend left]  node {$0.1$}  (A)
            edge               node {$0.15$} (C)
        (C) edge  [loop below] node {$0.8$}  (C)
            edge  [bend left]  node {$0.01$} (A)
            edge  [bend left]  node {$0.19$} (B);
          


\end{tikzpicture}}


\author[José Ignacio Escribano]{Trabajo Fin de Grado de Ingeniería del Software y Matemáticas \\ \vspace*{1em} José Ignacio Escribano}
\title[Trabajo Fin de Grado]{Métodos matemáticos para el estudio de rankings y diseño e implementación de una aplicación SPA para el estudio de la competitividad de la liga BBVA}
\date{16 de julio de 2015}
%\subtitle{}
%\logo{}
\institute[URJC]{\includegraphics[width=0.1\linewidth]{imagenes/logoURJC.jpg}}

%\subject{}
\setbeamercovered{transparent}
\setbeamertemplate{navigation symbols}{}

\setsansfont[Ligatures=TeX]{texgyreadventor}
\setmainfont[Ligatures=TeX]{texgyrepagella}

\begin{document}
	
	\begin{frame}
		\maketitle
	\end{frame}

	\begin{frame}[allowframebreaks]{Índice}
		\tableofcontents
	\end{frame}
	
	\begin{frame}
		\begin{center}
			\Huge\textbf{\textsf{\textcolor{blue}{Trabajo Fin de Grado de Matemáticas}}} \\ 
			\vspace*{1em}
			\huge\textbf{\textsf{\textcolor{black}{ Métodos matemáticos para el estudio de rankings }}}
		\end{center}
	\end{frame}
		
	\begin{frame}{Objetivos}
		\begin{itemize}
			\item Estudiar los métodos matemáticos para el estudio de rankings, con especial interés en los métodos que utilizan la teoría de grafos.
		\end{itemize}
	\end{frame}
	
	\section{Introducción}
	
	\begin{frame}
		\begin{center}
			\Huge\textbf{\textsf{\textcolor{blue}{Introducción}}}
		\end{center}
	\end{frame}
	
	\begin{frame}{Introducción}
		Vivimos rodeados de rankings
		
		\begin{itemize}
			\item Rankings socioeconómicos
			\begin{itemize}
				\item Producto Interior Bruto
				\item Prima de riesgo
				\item Igualdad de ingreso
			\end{itemize}
			\item Rankings deportivos
			\begin{itemize}
				\item En fútbol, el ranking FIFA
				\item En tenis, el ranking ATP/WTA
				\item En baloncesto, el draft de la NBA
			\end{itemize}
			\item Etc.
		\end{itemize}
	\end{frame}
	
	\begin{frame}{Introducción}
		\begin{figure}
			\centering
			\includegraphics[width=0.9\linewidth]{imagenes/ranking_fifa}
			\caption{Ranking FIFA \url{http://es.fifa.com/fifa-world-ranking/ranking-table/men/}}
			\label{fig:ranking_fifa}
		\end{figure}
	\end{frame}
	
	\begin{frame}{Introducción}
		\begin{figure}
			\centering
			\includegraphics[width=0.5\linewidth]{imagenes/ranking_deloitte}
			\caption{Football Money League \url{http://www2.deloitte.com/content/dam/Deloitte/uk/Documents/sports-business-group/deloitte-football-money-league-2015.PDF}}
			\label{fig:ranking_deloitte}
		\end{figure}
	\end{frame}
	
	\begin{frame}{Introducción}
		\begin{figure}
			\centering
			\includegraphics[width=0.65\linewidth]{imagenes/ranking_atp}
			\caption{Ranking ATP \url{http://es.atpworldtour.com/Rankings/Singles.aspx}}
			\label{fig:ranking_atp}
		\end{figure}
	\end{frame}
	
	\subsection*{Rating vs ranking}
	
	\begin{frame}{Rating vs ranking}
		\begin{defi}
			Dado un conjunto $\mathcal{N} = \{1,\dots, n\}$ de nodos, decimos que es un rating si a cada $i \in \mathcal{N}$ se le asigna un valor real, es decir, una función $\mathrm{r} : \mathcal{N} \to \R$.
		\end{defi}
		
		
		\begin{defi} \label{def:ranking}
			Dado un conjunto $\mathcal{N} = \{1,\dots,n\}$ que llamamos nodos, definimos el ranking $c$ como cualquier biyección $c : \mathcal{N} \to \mathcal{N}$.
		\end{defi}
		
		Un rating cuando se ordena (ascendente o descendentemente), crea un ranking.
	\end{frame}
	
	
	
	\begin{frame}{Rating vs ranking}
		Otra forma de ver los rankings y los ratings es en forma de vector. Si tenemos un ranking $c$, cada una de las posiciones del vector $c$, se corresponde con alguno de los nodos, y la posición en la aparezca en el vector es la posición que tiene en el ranking $c$. De igual forma para un rating.
		
		\begin{ejemplo}
			$$ c = \left(
			\begin{array}{c}
			\text{Equipo } 1 \\
			\text{Equipo } 4 \\
			\text{Equipo } 2 \\
			\vdots \\
			\text{Equipo } n
			\end{array}
			\right) $$
				
			En el ranking $c$ el Equipo $1$ ocupa la primera posición, el Equipo $4$ ocupa la segunda posición, el Equipo 2 la tercera posición y así sucesivamente.
			
		\end{ejemplo}
	\end{frame}
	
	\begin{frame}{Referencias}
		\begin{itemize}
		\item Criado, R., García, E., Pedroche F. y Romance M. (2013). Comparing rankings by means of competitivity graphs: structural properties and computation.
		
		\item Criado, R., García E., Pedroche F. y Romance M. (2013). A new method for comparing rankings though complex networks: Models and analysis of competitiveness of major european soccer leagues. Chaos: An interdisciplinary Journal of Nonlinear Science.
		
		\item Langville A. y Meyer C. (2012). Who's \#1?: The Science of Rating of Ranking. Princeton University Press.  
		\end{itemize}
		\end{frame}
	
	\begin{frame}{Algunos problemas de rankings}
			\begin{itemize}
			\item Métodos para obtener ratings (y rankings)
			\item Agregación de rankings
			\item Métodos de comparación
			\item Estudio de la competitividad en rankings
			\end{itemize}
		\end{frame}
	
	
	\begin{frame}
			\begin{center}
				\Huge\textbf{\textsf{\textcolor{blue}{Métodos para obtener ratings (y rankings)}}}
			\end{center}
		\end{frame}
		
		\begin{frame}{Métodos para obtener ratings (y rankings)}
				Existen muchos métodos para obtener ratings. A modo de ejemplo sólo veremos tres métodos, aunque existen multitud de ellos:
				
				\begin{itemize}
					\item Método de Massey: basado en mínimos cuadrados
					\item Método de Colley: basado en la regla de Laplace
					\item Método de Markov: basado en cadenas de Markov
				\end{itemize}
			\end{frame}
	

	
	
	%\begin{frame}
		%\begin{center}
			%\Huge\textbf{\textsf{\textcolor{blue}{Método de Massey}}}
		%\end{center}
	%\end{frame}
	
	
	%\subsection*{Método de Massey}
	
	%\begin{frame}{Método de Massey}
		
		%El método de Massey consiste en resolver el sistema $\mathbf{M r} = \mathbf{p}$
		%donde 
		
		%\[ m_{ij} = \begin{cases}
		%- \text{número de partidos jugados por } i \text{ y } j & \text{ si } i \neq j \\
	      %\text{número de partidos totales de } i & \text{ si } i = j 
		
		%\end{cases}\]
		
		%y $p_i$ es la suma de las diferencias de puntos del equipo $i$.\\
		
		%La suma de las filas de la matriz $\mathbf{M}$ es cero, por lo que el sistema no tiene solución única. Esto se soluciona sustituyendo una fila de la matriz $\mathbf{M}$ por una fila de unos, y el elemento correspondiente de $\mathbf{p}$ se sustituye por un $0$.
	
	%\end{frame}
	
	%\begin{frame}{Método de Massey}
	 
		%\begin{ejemplo}
			%\begin{table}[h]
				%\centering
				%\caption{Datos del método de Massey}
				%\label{tbl:massey}
				 %\begin{adjustbox}{max width=\textwidth} 
				%\begin{tabular}{ccccccc}
					%\cline{2-7}
					%& 1     & 2     & 3     & 4     & 5     & \begin{tabular}[c]{@{}c@{}}Diferencia\\ de puntos\end{tabular} \\ \hline
					%1 & ---     & 137-115 & 103-108 & 89-117  & 90-117  & \textcolor{red}{-38}                                        \\ \hline
					%2 & 115-137 & ---     & 113-100 & 100-102 & 103-110 & \textcolor{red}{-18}                                         \\ \hline
					%3 & 108-103 & 100-113 & ---     & 104-96  & 110-89  & \textcolor{green}{21}                                                            \\ \hline
					%4 & 117-89  & 102-100 & 96-104  & ---     & 89-100  & \textcolor{green}{11}                                         \\ \hline
					%5 & 117-90  & 110-103 & 89-110  & 100-89  & ---     & \textcolor{green}{24}                                        \\ \hline
				%\end{tabular}
				%\end{adjustbox}
			%\end{table}
		%\end{ejemplo}
		
	%\end{frame}
	
	%\begin{frame}{Método de Massey}
		
		%\begin{ejemplo}[continuación]
			%El sistema queda de la siguiente manera
			
			%\begin{equation*}
			%\left(\begin{array}{r r r r r}
			%4  & -1 & -1 & -1 & -1\\
			%-1 &  4 & -1 & -1 & -1\\
			%-1 & -1 &  4 & -1 & -1\\
			%-1 & -1 & -1 &  4 & -1\\
			%1 &  1 &  1 &  1 &  1 
			%\end{array}\right)
			%\left(\begin{array}{c}
			%\mathbf{r}_1\\
			%\mathbf{r}_2\\
			%\mathbf{r}_3\\
			%\mathbf{r}_4\\
			%\mathbf{r}_5
			%\end{array}\right)
			%=
			%\left(\begin{array}{c}
			%-38\\
			%-18\\
			%21\\
			%11\\
			%0
			%\end{array}\right)
			%\end{equation*}
					
		%\end{ejemplo}
		
	%\end{frame}
	
		%\begin{frame}{Método de Massey}
			
			%\begin{ejemplo}[continuación]
				%Resolviendo el sistema
				
				%\begin{table}[h]
					%\centering
					%\caption{Resultados del método de Massey}
					%\label{tbl:massey_resultados}
					%\begin{tabular}{@{}ccc@{}}
						%\cmidrule(l){2-3}
						%& $\mathbf{r}$ & Ranking \\ \cmidrule(l){2-3} 
						%1 & -7.6         & 5       \\ \midrule
						%2 & -3.6         & 4       \\ \midrule
						%3 & 4.2          & 2       \\ \midrule
						%4 & 2.2          & 3       \\ \midrule
						%5 & 4.8          & 1       \\ \bottomrule
					%\end{tabular}
				%\end{table}
				
				
			%\end{ejemplo}
			
	%\end{frame}
	
	%\subsection*{Método de Colley}
	
	%\begin{frame}
		%\begin{center}
			%\Huge\textbf{\textsf{\textcolor{blue}{Método de Colley}}}
		%\end{center}
	%\end{frame}
	
	%\begin{frame}{Método de Colley}
		%El método de Colley se basa, de nuevo, en resolver un sistema lineal. En este caso se trata de $\mathbf{Cr} = \mathbf{b}$ donde
		
		%\[ c_{ij} = \begin{cases}
		%2 + t_i & \text{si } i = j \\
		%-n_{ij} & \text{si } i \neq j  
		%\end{cases} \]
		
		%y 
		
		%\[ b_i = \dfrac{1}{2}(w_i - l_i) \]
		
		%con $w_i$ el número de victorias del equipo $i$, $l_i$ el número de derrotas del equipo $i$, $t_i = w_i + l_i$ y $n_{ij}$ es el número de partidos jugados entre $i$ y $j$.
	%\end{frame}
	
	%\begin{frame}{Método de Colley}
		%\begin{ejemplo}
			%\begin{table}[h]
				%\centering
				%\caption{Datos del método de Colley}
				%\label{tbl:colley}
				%\begin{tabular}{@{}ccc@{}}
					%\cmidrule(l){2-3}
					%& Victorias & Derrotas \\ \midrule
					%1 & 1         & 3        \\
					%2 & 1         & 3        \\
					%3 & 3         & 1        \\
					%4 & 2         & 2        \\
					%5 & 3         & 1        \\ \bottomrule
				%\end{tabular}
			%\end{table}
		%\end{ejemplo}
			
	%\end{frame}
	
	%\begin{frame}{Método de Colley}
		%\begin{ejemplo}[continuación]
			%El sistema queda de la siguiente forma:
			
			%\begin{equation*}
			%\left(\begin{array}{r r r r r}
			%6 & -1 & -1 & -1 & -1\\
			%-1 &  6 & -1 & -1 & -1\\
			%-1 & -1 &  6 & -1 & -1\\
			%-1 & -1 & -1 &  6 & -1\\
			%-1 & -1 & -1 & -1 &  6
			%\end{array}\right)
			%\left(\begin{array}{c}
			%\mathbf{r}_1\\
			%\mathbf{r}_2\\
			%\mathbf{r}_3\\
			%\mathbf{r}_4\\
			%\mathbf{r}_5
			%\end{array}\right)
			%=
			%\left(\begin{array}{c}
			%0\\
			%0\\
			%2\\
			%1\\
			%2
			%\end{array}\right)
			%\end{equation*}
		%\end{ejemplo}
		
	%\end{frame}
	
	%\begin{frame}{Método de Colley}
		%\begin{ejemplo}[continuación]
			%Resolviendo y ordenando descendentemente
			
			%\begin{table}[h]
				%\centering
				%\caption{Resultados del método de Colley}
				%\label{tbl:colley_con_empates}
				%\begin{tabular}{@{}ccc@{}}
					%\cmidrule(l){2-3}
					%& $\mathbf{r}$ & Ranking \\ \midrule
					%1 & 0.3571       & 4       \\
					%2 & 0.3571       & 4       \\
					%3 & 0.6429       & 1       \\
					%4 & 0.5          & 3       \\
					%5 & 0.6429       & 1       \\ \bottomrule
				%\end{tabular}
			%\end{table}
		%\end{ejemplo}
		
	%\end{frame}
	
	\subsection*{Método de Markov}
	
	\begin{frame}
		\begin{center}
			\Huge\textbf{\textsf{\textcolor{blue}{Método de Markov}}}
		\end{center}
	\end{frame}
	
	\begin{frame}{Método de Markov}
		El método de Markov sigue los siguientes pasos:
		
		\begin{enumerate}
			\item Formar $\mathbf{S}$ usando las matrices $\mathbf{V}_i$ para cada una de las $k$ estadísticas.
			
			\[ \mathbf{S} = \sum_{i=1}^{k} \alpha_i \mathbf{S}_i \]
			
			donde $a_i \geq 0$ y $\sum_{i=1}^{k} \alpha_i = 1$.
			
			\item Calcular $\mathbf{r}$, el punto fijo de la matriz de transición $\mathbf{S}$. Si $\mathbf{S}$ es reducible, usar la matriz irreducible $\overline{\mathbf{S}}$
			
			\[\overline{\mathbf{S}} = \beta \mathbf{S} + \dfrac{(1 - \beta)}{n} \mathbf{E}\]
			
			donde $\mathbf{E}$ es una matriz de unos, $n$ es el número de equipos y $\beta \in [0, 1]$.
		\end{enumerate}
	\end{frame}
	
%	\begin{frame}{Método de Markov}
		%\begin{ejemplo}
			%Consideramos los datos de victorias y derrotas y, diferencia de puntos de los ejemplos anteriores. Formamos las matrices $\mathbf{V}_p$ y $\mathbf{V}_v$ que representan las estadísticas de puntos y de victorias, respectivamente.
			
			%\[ \mathbf{V}_p = \left(\begin{array}{rrrrr}
			%0   & 115 & 108 & 117 & 117\\
			%137 & 0   & 100 & 102 & 110\\
			%103 & 113 & 0   & 96  & 89 \\
			%89  & 100 & 104 & 0   & 100\\
			%90  & 103 & 100 & 89  & 0
			%\end{array}\right); \quad 
			%\mathbf{V}_v = \left(\begin{array}{rrrrr}
			%0 & 0 & 1 & 1 & 1\\
			%1 & 0 & 0 & 1 & 1\\
			%0 & 1 & 0 & 0 & 0\\
			%0 & 0 & 1 & 0 & 1\\
			%0 & 0 & 1 & 0 & 0
			%\end{array}\right) \]
		%\end{ejemplo}
	%\end{frame}
	
	%\begin{frame}{Método de Markov}
		%\begin{ejemplo}[continuación]
			%Creamos las matrices estocásticas $\mathbf{S}_p$ y $\mathbf{S}_v$, dividiendo cada elemento de la fila entre la suma de su fila:
			
			%\[ \mathbf{S}_p = \left(\begin{array}{rrrrr}
			%0      & %0.2516 & %0.2363 & %0.2560 & %0.2560 \\
			%0.3051 & 0      & 0.2227 & 0.2272 & 0.2450 \\
			%0.2569 & 0.2818 & 0      & 0.2394 & 0.2219 \\
			%0.2265 & 0.2545 & 0.2646 & 0      & 0.2545 \\
			%0.2356 & 0.2696 & 0.2618 & 0.2330 & 0
			%\end{array}\right) \] 
			
			%\[ \mathbf{S}_v = \left(\begin{array}{rrrrr}
			%0     & 0   & 1/3 & 1/3 & 1/3\\
			%1/3   & 0   & 0   & 1/3 & 1/3\\
			%0     & 1   & 0   & 0   & 0  \\
			%0     & 0   & 1/2 & 0   & 1/2\\
			%0     & 0   & 1   & 0   & 0
			%\end{array}\right) \]
			
			
		%\end{ejemplo}
	%\end{frame}
	
	%\begin{frame}{Método de Markov}
		%\begin{ejemplo}[continuación]
		%Creamos la matriz $\mathbf{S} = \alpha_1 \mathbf{S}_p + \alpha_2 \mathbf{S}_v$. Elegimos $\alpha_1 = 0.7$ y $\alpha_2 = 0.3$.\\
		
		%La matriz $\mathbf{S}$ queda así:
		
		%\[ \mathbf{S} = \left(\begin{array}{rrrrr}
		%0      & 0.0915 & 0.3104 & 0.3013 & 0.3040 \\
		%0.3088 & 0      & 0.0845 & 0.3097 & 0.3142 \\
		%0.0709 & 0.7668 & 0      & 0.0794 & 0.0785 \\
		%0.0768 & 0.0682 & 0.4218 & 0      & 0.4199 \\
		%0.0768 & 0.0735 & 0.7666 & 0.0763 & 0
		
		%\end{array}\right) \]
		
		%Necesitamos saber si $\mathbf{S}$ es irreducible. Para ello, aplicamos el siguiente resultado.
		%\end{ejemplo}
	%\end{frame}
	
	%\begin{frame}{Método de Markov}
	%\begin{prop}
		%Sea $\mathbf{S}$ la matriz de transición de una cadena de Markov, cuyo autovalor $\lambda = 1$ corresponde con el autovector $\mathbf{r}$. Entonces, la cadena de Markov es irreducible si todos los elementos de $\mathbf{r}$ son no nulos.
	%\end{prop}
	%\end{frame}
	
	%\begin{frame}{Método de Markov}
		%\begin{ejemplo}[continuación]
		%Los autovalores de $\mathbf{S}$ son $\lambda = 1, -0.2454 \pm 0.536i, -0.2549 \pm 0.0833i$.\\
		%El autovalor asociado a $\lambda = 1$ es $\mathbf{r} = (-0.2649, -0.5337, -0.5946, -0.3250, -0.4311)^T$. Como todas las componentes de $\mathbf{r}$ son no nulas, entonces $\mathbf{S}$ es irreducible. Sólo nos queda normalizar el vector $\mathbf{r}$. Los resultados se muestran a continuación. 
		
		%\begin{table}[h]
			%\centering
			%\caption{Resultados del método de Markov}
			%\label{tbl:markov_resultados}
			%\begin{tabular}{@{}ccc@{}}
				%\cmidrule(l){2-3}
				%& $\mathbf{r}$ & Ranking \\ \midrule
				%1 & 0.1232       & 5       \\
				%2 & 0.2483       & 2       \\
				%3 & 0.2766       & 1       \\
				%4 & 0.1512       & 4       \\
				%5 & 0.2006       & 3       \\ \bottomrule
			%\end{tabular}
		%\end{table}
		%\end{ejemplo}
	%\end{frame}
	
	\section{Agregación de rankings}
	
	\begin{frame}
		\begin{center}
			\Huge\textbf{\textsf{\textcolor{blue}{Agregación de rankings}}}
		\end{center}
	\end{frame}
	
	%\begin{frame}{Agregación de rankings}
		
		%\begin{itemize}
		%\item En la práctica, se tienen distintos rankings sobre un misma estadística con resultados totalmente distintos.\\
		
		%\item Es recomendable agregar todos esos rankings iniciales para crear un nuevo ranking que contenga la información de todos esos rankings iniciales.\\
		
		%\item Hay que destacar que si partimos de rankings con buena calidad, obtendremos un buen ranking agregado. De la misma forma, si tenemos unos malos rankings iniciales, obtendremos un ranking agregado de mala calidad.
		%\end{itemize}
	%\end{frame}
	
	\begin{frame}{Agregación de rankings}
		\begin{figure}
			\centering
			\agregacionrankings
			\caption{Agregación de rankings}
			\label{fig:agregación_rankings}
		\end{figure}
	\end{frame}
	
	\subsection*{Métodos de agregación de rankings}
	
	\begin{frame}
		\begin{center}
			\Huge\textbf{\textsf{\textcolor{blue}{Métodos de agregación de rankings}}}
		\end{center}
	\end{frame}
	
	%\subsubsection*{Método de Borda}
	
	%\begin{frame}
		%\begin{center}
			%\Huge\textbf{\textsf{\textcolor{blue}{Método de Borda}}}
		%\end{center}
	%\end{frame}
	
	%\begin{frame}{Método de Borda}
		
		%\begin{itemize}
			%\item Para cada ranking, cada equipo recibe una puntuación igual al número de equipos que le superan.
			
			%\item La puntuación de cada ranking es sumada para cada equipo para crear un nuevo ranking, que se llama recuento de Borda.
			
			%\item Los equipos son ordenados en orden ascendente según este método para obtener el ranking.
		%\end{itemize}
		
	%\end{frame}
	
	%\begin{frame}{Método de Borda}
		
		%\begin{ejemplo}
			%Consideremos la familia de rankings $\mathcal{R} = \{c_1, c_2, c_3\}$ donde
			
			%\begin{equation*}
			%c_1 = \left( \begin{array}{c}
			%1\\
			%2\\
			%3\\
			%4\\
			%5
			%\end{array} \right), \quad
			%c_2 = \left( \begin{array}{c}
			%1\\
			%2\\
			%3\\
			%4\\
			%5
			%\end{array} \right), \quad
			%c_3 = \left( \begin{array}{c}
			%2\\
			%4\\
			%1\\
			%3\\
			%5
			%\end{array} \right)
			%\end{equation*}
			 
		%\end{ejemplo}
	
		
	%\end{frame}
	
	%\begin{frame}{Método de Borda}
		%\begin{ejemplo}[continuación]
			
			%\begin{table}[h]
				%\centering
				%\caption{Resultados del método de Borda}
				%\label{tbl:borda_resultados}

				%\begin{tabular}{@{}ccc@{}}
					%\cmidrule(l){2-3}
					%& Recuento de Borda & Ranking agregado \\ \midrule
					%5 & $4 + 4 + 4 = 12$       & 5       \\
					%4 & $3 + 3 + 1 =  7$       & 3       \\
					%2 & $1 + 1 + 0 =  2$       & 1       \\
					%3 & $2 + 2 + 3 =  7$       & 3       \\
					%1 & $0 + 0 + 2 =  2$       & 1       \\ \bottomrule
				%\end{tabular}

			%\end{table}
			
		%\end{ejemplo}
	%\end{frame}
	
	%\subsubsection*{Método del ranking promedio}
	
	%\begin{frame}
		%\begin{center}
			%\Huge\textbf{\textsf{\textcolor{blue}{Método del ranking promedio}}}
		%\end{center}
	%\end{frame}
	
	%\begin{frame}{Método del ranking promedio}
		%\begin{itemize}
			%\item Para cada uno de los distintos números enteros que representan el orden de cada equipo en los distintos rankings, calculamos la media aritmética con éstos para crear el nuevo ranking agregado.
			
			%\item Sólo es válido para rankings completos (rankings con el mismo número de elementos).
		%\end{itemize}
	%\end{frame}
	
	%\begin{frame}{Método del ranking promedio}
		%\begin{ejemplo}
			%Consideremos los rankings del ejemplo anterior. Este método nos da el siguiente resultado:
			
			%\begin{table}[h]
				%\centering
				%\caption{Resultados del método del ranking promedio}
				%\label{tbl:promedio_resultados}
				%\begin{tabular}{@{}cccccc@{}}
					%\cmidrule(l){2-6}
					%& $c_1$ & $c_2$ & $c_3$ & Ranking promedio & Ranking agregado \\ \midrule
					%5 & $5$    & $5$    & $5$    & $5$              & $5$              \\
					%4 & $4$    & $4$    & $2$    & $3.33$           & $3$              \\
					%2 & $2$    & $2$    & $1$    & $1.67$           & $1$              \\
					%3 & $3$    & $3$    & $4$    & $3.33$           & $3$              \\
					%1 & $1$    & $1$    & $3$    & $1.67$           & $1$              \\ \bottomrule
				%\end{tabular}
			%\end{table} 
		%\end{ejemplo}
	%\end{frame}
	
	%\subsubsection*{Método de datos de partidos simulados}
	
	%\begin{frame}
		%\begin{center}
			%\Huge\textbf{\textsf{\textcolor{blue}{Método de datos de partidos simulados}}}
		%\end{center}
	%\end{frame}
	
	%\begin{frame}{Método de datos de partidos simulados}
		%\begin{figure}
			%\centering
			%\resizebox{.6\linewidth}{!}{\partidossimulados}
			%\caption{Método de datos de partidos simulados}
			%\label{fig:partidos_simulados}
		%\end{figure}
	%\end{frame}
	
	%\begin{frame}{Método de datos de partidos simulados}
		%\begin{ejemplo}
			%Consideremos la familia de rankings $\mathcal{R} = \{c_1, c_2, c_3\}$ donde
			
			%\begin{equation*}
			%c_1 = \left( \begin{array}{c}
			%1\\
			%2\\
			%3\\
			%4\\
			%5
			%\end{array} \right), \quad
			%c_2 = \left( \begin{array}{c}
			%1\\
			%2\\
			%3\\
			%4\\
			%5
			%\end{array} \right), \quad
			%c_3 = \left( \begin{array}{c}
			%2\\
			%4\\
			%1\\
			%3\\
			%5
			%\end{array} \right)
			%\end{equation*}
			
		%\end{ejemplo}
	%\end{frame}
	
	%\begin{frame}{Método de datos de partidos simulados}
		%\begin{ejemplo}[continuación]
			%\begin{table}[h]
				%\centering
				%\caption{Partidos simulados del ejemplo}
				%{\tiny
				  	
				%\begin{tabular}{@{}cccccc@{}}
					%\cmidrule(l){2-6}
					%& 5                                                                 & 4                                                                 & 3                                                                 & 2                                                                 & 1                                                                 \\ \midrule
					%5 & ---                                                                 & \begin{tabular}[c]{@{}c@{}}100-101\\ 100-101\\ 100-103\end{tabular} & \begin{tabular}[c]{@{}c@{}}100-103\\ 100-103\\ 100-104\end{tabular} & \begin{tabular}[c]{@{}c@{}}100-102\\ 100-102\\ 100-101\end{tabular} & \begin{tabular}[c]{@{}c@{}}100-104\\ 100-104\\ 100-102\end{tabular} \\ \midrule
					%4 & \begin{tabular}[c]{@{}c@{}}101-100\\ 101-100\\ 103-100\end{tabular} & ---                                                                 & \begin{tabular}[c]{@{}c@{}}100-102\\ 100-102\\ 100-101\end{tabular} & \begin{tabular}[c]{@{}c@{}}100-101\\ 100-101\\ 102-100\end{tabular} & \begin{tabular}[c]{@{}c@{}}100-103\\ 100-103\\ 101-100\end{tabular} \\ \midrule
					%3 & \begin{tabular}[c]{@{}c@{}}103-100\\ 103-100\\ 104-100\end{tabular} & \begin{tabular}[c]{@{}c@{}}102-100\\ 102-100\\ 101-100\end{tabular} & ---                                                                 & \begin{tabular}[c]{@{}c@{}}101-100\\ 101-100\\ 103-100\end{tabular} & \begin{tabular}[c]{@{}c@{}}100-101\\ 100-101\\ 102-100\end{tabular} \\ \midrule
					%2 & \begin{tabular}[c]{@{}c@{}}102-100\\ 102-100\\ 101-100\end{tabular} & \begin{tabular}[c]{@{}c@{}}101-100\\ 101-100\\ 100-102\end{tabular} & \begin{tabular}[c]{@{}c@{}}100-101\\ 100-101\\ 100-103\end{tabular} & ---                                                                 & \begin{tabular}[c]{@{}c@{}}100-102\\ 100-102\\ 100-101\end{tabular} \\ \midrule
					%1 & \begin{tabular}[c]{@{}c@{}}104-100\\ 104-100\\ 102-100\end{tabular} & \begin{tabular}[c]{@{}c@{}}103-100\\ 103-100\\ 100-101\end{tabular} & \begin{tabular}[c]{@{}c@{}}101-100\\ 101-100\\ 100-102\end{tabular} & \begin{tabular}[c]{@{}c@{}}102-100\\ 102-100\\ 101-100\end{tabular} & ---                                                                 \\ \bottomrule
				%\end{tabular}}

			%\end{table}
		%\end{ejemplo}
	%\end{frame}
	
	%\begin{frame}{Método de datos de partidos simulados}
		%\begin{ejemplo}[continuación]
			%Formando los sistemas con los métodos de Massey y Colley obtenemos lo siguiente
			
			%\begin{table}[h]
			%\centering
			%\caption{Resultados del método de datos de partidos simulados}
			%\label{tbl:partidos_simulados_resultados}
			%\begin{tabular}{@{}ccccc@{}}
				%\cmidrule(l){2-5}
				%& \begin{tabular}[c]{@{}c@{}}$\mathbf{r}$\\ de Massey\end{tabular} & \begin{tabular}[c]{@{}c@{}}Ranking\\ agregado\end{tabular} & \begin{tabular}[c]{@{}c@{}}$\mathbf{r}$\\ de Colley\end{tabular} & \begin{tabular}[c]{@{}c@{}}Ranking\\ agregado\end{tabular} \\ \midrule
				%5 & -2.3077                                                          & 5                                                          & 0.5000                                                           & 5                                                          \\
				%4 & -0.3846                                                          & 4                                                          & 0.7941                                                           & 4                                                          \\
				%3 & 1.5385                                                           & 1                                                          & 1.0882                                                           & 1                                                          \\
				%2 & -0.3846                                                          & 3                                                          & 1.0294                                                           & 3                                                          \\
				%1 & 1.5385                                                           & 2                                                          & 1.0884                                                           & 2                                                          \\ \bottomrule
			%\end{tabular}
		%\end{table}
			
		%\end{ejemplo}
	%\end{frame}
	
	\begin{frame}{Algunos métodos de agregación de rankings}
	\begin{itemize}
		\item Método de Borda
		\item Método del ranking promedio
		\item Método de datos de partidos simulados
		\item Método óptimo de agregación de rankings
	\end{itemize}
	\end{frame}
	
	\subsubsection*{Método óptimo de agregación de rankings}
	
	\begin{frame}
		\begin{center}
			\Huge\textbf{\textsf{\textcolor{blue}{Método óptimo de agregación de rankings}}}
		\end{center}
	\end{frame}
	
	\begin{frame}{Método óptimo de agregación de rankings}
		El método óptimo de agregación de rankings consiste en resolver el siguiente problema de programación lineal
		
		 \begin{equation}
		 \begin{array}{rl}
		 \max         & \sum\limits_{i=1}^{n} \sum\limits_{j=1}^{n} c_{ij} x_{ij}\\
		 \mathrm{s.a} & x_{ij} + x_{ji} = 1\\
		 & x_{ij} + x_{jk} + x_{ki} \leq 2\\
		 & x_{ij} \geq 0
		 \end{array}
		 \end{equation}
		 
		 donde 
		 
		 \begin{multline} \label{eq:conformidad}
		 c_{ij} = \text{número de rankings con $i$ por encima de $j$} \\ - \text{número de rankings con $i$ por debajo de $j$}
		 \end{multline}
		 
		 y
		 
		 \begin{equation}
		 x_{ij} = \begin{cases}
		 1 & \text{ si el equipo $i$ está por encima de $j$ en el ranking}\\
		 0 & \text{ en caso contrario}
		 \end{cases}
		 \end{equation}
	\end{frame}
	
	%\begin{frame}{Método óptimo de agregación de rankings}
		%\begin{ejemplo}
			%Consideremos la familia de rankings $\mathcal{R} = \{c_1, c_2, c_3\}$ donde
			
			%\begin{equation*}
			%c_1 = \left( \begin{array}{c}
			%1\\
			%2\\
			%3\\
			%4\\
			%5
			%\end{array} \right), \quad
			%c_2 = \left( \begin{array}{c}
			%1\\
			%2\\
			%3\\
			%4\\
			%5
			%\end{array} \right), \quad
			%c_3 = \left( \begin{array}{c}
			%2\\
			%4\\
			%1\\
			%3\\
			%5
			%\end{array} \right)
			%\end{equation*}
			
		%\end{ejemplo}
	%\end{frame}
	
	%\begin{frame}{Método óptimo de agregación de rankings}
		%\begin{ejemplo}[continuación]
			%Primero debemos formar la matriz de conformidad $\mathbf{C}$ quedando de la siguiente manera:
					
			%\begin{equation*}
			%\mathbf{C} = \bordermatrix{
			  %& 5 & 4 & 3 & 2 & 1 \cr
			%5 & 0 & -3 & -3 & -3 & -3 \cr
			%4 & 3 &  0 & -3 & -1 & -1 \cr
			%3 & 3 &  3 &  0 &  3 & -1 \cr
			%2 & 3 &  1 & -3 &  0 & -3 \cr
			%1 & 3 &  1 &  1	&  3 &	0 \cr
			%}
			%\end{equation*}
			
			%Resolviendo el problema de programación la matriz $\mathbf{X}$ con las variables de decisión queda así:
			
			%\begin{equation*}
			%\mathbf{X} = \left(\begin{array}{rrrrr}
			%0 & 0 & 0 & 0 & 0 \\
			%1 & 0 & 0 & 0 & 0 \\
			%1 & 1 & 0 & 1 & 0 \\
			%1 & 1 & 0 & 0 & 0 \\
			%1 & 1 & 1 & 1 &	0
			%\end{array}\right)
			%\end{equation*}
			
			
		%\end{ejemplo}
	%\end{frame}
	
	%\begin{frame}{Método óptimo de agregación de rankings}
		%\begin{ejemplo}[continuación]
			
			%El valor de la función objetivo es $24$. El ranking producido es el siguiente:
			
			%\begin{table}[h]
				%\centering
				%\caption{Resultados del método óptimo}
				%\label{tbl:optimo_resultados}
				%\begin{tabular}{@{}cc@{}}
					%\cmidrule(l){2-2}
					%& Ranking \\ \midrule
					%5 & 5       \\
					%4 & 4       \\
					%3 & 2       \\
					%2 & 3       \\
					%1 & 1       \\ \bottomrule
				%\end{tabular}
			%\end{table}
			
			
		%\end{ejemplo}
	%\end{frame}
	
	\section{Métodos de comparación}
	
	\begin{frame}
		\begin{center}
			\Huge\textbf{\textsf{\textcolor{blue}{Métodos de comparación}}}
		\end{center}
	\end{frame}
	
	\subsection*{Tau de Kendall}
	
%	\begin{frame}
%		\begin{center}
%			\Huge\textbf{\textsf{\textcolor{blue}{Tau de Kendall}}}
%		\end{center}
%	\end{frame}
	
	\begin{frame}{Tau de Kendall}
		\begin{defi}
			La $\tau$ de Kendall para rankings completos con tamaño $n$ se define como sigue
			
			\begin{equation} \label{def:tau_kendall}
			\tau = \dfrac{n_c - n_d}{n(n-1)/2}
			\end{equation}
			
			donde $n_c$ es el número de pares que concuerdan en ambos rankings y $n_d$ es el número de pares que no concuerdan en ambos rankings.\\
			
			\begin{itemize}
			\item Un par $(i,j)$ se dice que concuerda si el elemento $i$ aparece por encima del elemento $j$ en ambos rankings. En caso contrario se dice que no concuerda.
			
			\item La $\tau$ de Kendall está acotada por $-1$ y $1$, es decir, $-1 \leq \tau \leq 1$. Cuando $\tau = 1$, los rankings son iguales y cuando $\tau = -1$, los rankings están en orden inverso.  
			\end{itemize}
		\end{defi}
	\end{frame}
	
	%\begin{frame}{Tau de Kendall}
		%\begin{ejemplo}
			%Consideremos los siguientes rankings:
			
			%\begin{equation*}
			%c_1 = \left( \begin{array}{c}
			%3\\
			%2\\
			%1\\
			%4\\
			%5
			%\end{array} \right), \quad
			%c_2 = \left( \begin{array}{c}
			%3\\
			%2\\
			%1\\
			%4\\
			%5
			%\end{array} \right), \quad
			%c_3 = \left( \begin{array}{c}
			%5\\
			%4\\
			%1\\
			%2\\
			%3
			%\end{array} \right), \quad
			%c_4 = \left( \begin{array}{c}
			%2\\
			%1\\
			%4\\
			%5\\
			%3
			%\end{array} \right)
			%\end{equation*}
			
			%\begin{align*}
			%\tau(c_1, c_2) & = \dfrac{10 - 0}{5 \cdot 4/2} = 1\\
			%\tau(c_1, c_3) & = \dfrac{0 - 10}{10} = -1\\
			%\tau(c_3, c_4) & = \dfrac{4 -  6}{10} = -\dfrac{2}{10} = - \dfrac{1}{5}
			%\end{align*}
			
		%\end{ejemplo}		
	%\end{frame}
	
	\subsection*{Ro de Spearman}
	
%	\begin{frame}
%		\begin{center}
%			\Huge\textbf{\textsf{\textcolor{blue}{Ro de Spearman}}}
%		\end{center}
%	\end{frame}
	
	\begin{frame}{Ro de Spearman}
		\begin{defi}
			Definimos la ro de Spearman como la distancia $L_1$ entre dos rankings completos $c_1$ y $c_2$ de tamaño $n$. Esto es, 
			
			\begin{equation}
			\rho = \norm{c_1 - c_2}_1 = \sum\limits_{i=1}^{n} |c_1^{-1}(i) - c_2^{-1}(i)|
			\end{equation}
			
			donde $c_1^{-1}(i)$ es la posición del equipo $i$ en el ranking $c_1$ y $c_2^{-1}(i)$ es la posición del equipo $i$ en el ranking $c_2$.
		\end{defi}
	
	\end{frame}
	
	%\begin{frame}{Ro de Spearman}
		%\begin{ejemplo}
			%Consideremos los siguientes rankings:
			
			%\begin{equation*}
			%c_1 = \left( \begin{array}{c}
			%2\\
			%3\\
			%4\\
			%1\\
			%5
			%\end{array} \right), \quad
			%c_2 = \left( \begin{array}{c}
			%5\\
			%4\\
			%1\\
			%3\\
			%2
			%\end{array} \right), \quad
			%c_3 = \left( \begin{array}{c}
			%1\\
			%2\\
			%3\\
			%4\\
			%5
			%\end{array} \right), \quad
			%c_4 = \left( \begin{array}{c}
			%2\\
			%3\\
			%4\\
			%1\\
			%5
			%\end{array} \right)
			%\end{equation*}
			
			%\begin{align*}
			%\rho(c_1, c_3) & = 6\\
			%\rho(c_1, c_4) & = 0\\
			%\rho(c_2, c_3) & = 12\\
			%\rho(c_3, c_4) & = 6
			%\end{align*}  
		%\end{ejemplo}
		
	%\end{frame}
	
	
	\section{Competitividad en rankings}
	
	\begin{frame}
		\begin{center}
			\Huge\textbf{\textsf{\textcolor{blue}{Competitividad en rankings}}}
		\end{center}
	\end{frame}
	
	\begin{frame}{Competitividad en rankings}
		\begin{defi}
			Dada una familia $\mathcal{R} = \{c_1, c_2, \dots, c_r\}$ de rankings, decimos que los nodos $(i,j) \in \mathcal{N} \times \mathcal{N}$ compiten si existe $t \in \{1,2,\dots, r-1\}$ tal que $i$ y $j$ intercambian sus posiciones relativas entre rankings consecutivos $c_t$ y $c_{t+1}$.
		\end{defi}
	\end{frame}
	
	\begin{frame}{Competitividad en rankings}
		\begin{ejemplo}
			Si consideramos la familia de rankings $\mathcal{R} = \{c_1, c_2\}$ donde $c_1$ y $c_2$ son, respectivamente
			
			\begin{equation*}
			c_1 = \left( \begin{array}{c}
			1\\
			3\\
			2\\
			4\\
			5
			\end{array} \right), \quad
			c_2 = \left( \begin{array}{c}
			1\\
			2\\
			4\\
			3\\
			5
			\end{array} \right)
			\end{equation*}
			
			Los pares $(1,2)$ y $(1,3)$ no compiten, puesto que los nodos no intercambian sus posiciones relativas entre rankings consecutivos. Sin embargo, los pares $(3,4)$ y $(2,3)$ sí compiten puesto que intercambian sus posiciones relativas entre rankings consecutivos, $c_1$ y $c_2$.
		\end{ejemplo}
	\end{frame}
	
	\subsection*{Grafo de competitividad}
	
	\begin{frame}
		\begin{center}
			\Huge\textbf{\textsf{\textcolor{blue}{Grafo de competitividad}}}
		\end{center}
	\end{frame}
	
	\begin{frame}{Grafo de competitividad}
		\begin{defi}
			Definimos grafo de competitividad de la familia de rankings $\mathcal{R}$, y lo denotaremos como $G_c(\mathcal{R}) = (\mathcal{N}, E_\mathcal{R})$, donde $\mathcal{N}$ es el conjunto de nodos y $E_\mathcal{R}$ denota el conjunto de arcos dados por la siguiente regla: existe un arco entre $i$ y $j$ si el par $(i,j)$ compite.
		\end{defi}
	\end{frame}
	
	\begin{frame}{Grafo de competitividad}
		\begin{ejemplo}
			Consideremos la familia de rankings $\mathcal{R} = \{c_1, c_2, c_3, c_4\}$ con los siguientes rankings:
			
			\begin{equation*}
			%\resizebox{.8\linewidth}{!}{
			c_1 = \left( \begin{array}{c}
			1 \\
			2 \\
			3 \\
			4 \\
			5
			\end{array} \right), \quad
			c_2 = \left( \begin{array}{c}
			2 \\
			3 \\
			4 \\
			1 \\
			5
			\end{array} \right), \quad
			c_3 = \left( \begin{array}{c}
			2 \\
			3 \\
			1 \\
			4 \\
			5
			\end{array} \right), \quad
			c_4 = \left( \begin{array}{c}
			3 \\
			4 \\
			2 \\
			1 \\
			5
			\end{array} \right)
			%}
			\end{equation*}
		
		\end{ejemplo}
	\end{frame}
	
	\begin{frame}{Grafo de competitividad}
		\begin{ejemplo}[continuación]
			\begin{figure}
				\centering
				\ejemplografocompetitividad
				\caption{Grafo de competitividad de $\mathcal{R}$}
				\label{fig:grafo_competitividad}
			\end{figure}	
		\end{ejemplo}
	\end{frame}
	
	\begin{frame}{Grafo de competitividad evolutivo}
		\begin{defi}
			Decimos que los nodos $i$, $j$ compiten $k$ veces si $k$ es el número de veces máximo de rankings donde $i$ y $j$ compiten.
		\end{defi}
		
		\begin{defi}
			El grafo de competitividad evolutivo de $\mathcal{R}$, denotado por $G_c^e(\mathcal{R}) = (\mathcal{N}, E_\mathcal{R}^e)$ es el grafo ponderado con el conjunto $\mathcal{N}$ como nodos y aristas dadas por la siguiente regla: hay una arista entre $i$ y $j$ con peso $k$ si $(i,j)$ compiten $k$ veces.
		\end{defi}
	\end{frame}
	
	\begin{frame}{Grafo de competitividad evolutivo}
		\begin{ejemplo}
			Consideremos la familia de rankings $\mathcal{R}$ del ejemplo anterior.
			
			\begin{figure}
				\centering
				\resizebox{!}{0.5\textheight}{\ejemplografocompetitividadevolutivo}
				
				\caption{Grafo de competitividad evolutivo de $\mathcal{R}$}
				\label{fig:grafo_competitividad_evolutivo}
			\end{figure}
			
		\end{ejemplo}
	\end{frame}
	
	\subsection*{Algunas medidas de competitividad}
	
	\begin{frame}
		\begin{center}
			\Huge\textbf{\textsf{\textcolor{blue}{Algunas medidas de competitividad}}}
		\end{center}
	\end{frame}
	
	\begin{frame}{Medidas de competitividad}
	
		Las medidas de competitividad que consideraremos son las siguientes:
	
		\begin{itemize}
		\item Grado medio normalizado
		\item Fuerza media normalizada
		\item Tau de Kendall generalizada
		\item Diámetro del grafo de competitividad evolutivo
		\item Eficiencia del grafo de competitividad evolutivo
		\item Longitud del camino característico
		\end{itemize}
			
	\end{frame}
	
	
	\subsubsection*{Grado medio normalizado}
	
%	\begin{frame}
%		\begin{center}
%			\Huge\textbf{\textsf{\textcolor{blue}{Grado medio normalizado}}}
%		\end{center}
%	\end{frame}
	
	\begin{frame}{Grado medio normalizado}
		\begin{defi}
			Se define grado medio normalizado de una familia de rankings $\mathcal{R}$, como la suma de todos los grados de los nodos en el grafo de competitividad $G_c(\mathcal{R})$ dividido por la suma sobre todos los nodos de sus grados más altos posibles. Esto es,
			
			\begin{equation}
			\mathrm{ND}(\mathcal{R}) = \dfrac{1}{n(n-1)} \sum_{i \in \mathcal{N}} d_i
			\end{equation}
			
			donde $d_i$ es el número de nodos adyacentes al nodo  $i$.
		\end{defi}
		
		\begin{defi}
			Se dice que la familia de rankings $\mathcal{R}$ es más competitiva que la familia de rankings $\mathcal{S}$ con respecto al grado medio normalizado si $\mathrm{ND}(\mathcal{R}) > \mathrm{ND}(\mathcal{S})$.
		\end{defi}
	\end{frame}
	
%	\begin{frame}{Grado medio normalizado}
%		\begin{ejemplo}
%			\begin{columns}[t] % align columns
%				\begin{column}{.50\textwidth}
%					\begin{figure}
%						\centering
%						\resizebox{!}{0.3\textheight}{\ejemplografocompetitividad}
%						\caption{Grafo de competitividad de $\mathcal{R}$}
%					\end{figure}
%					\[ \mathrm{ND}(\mathcal{R}) = \dfrac{1}{5\cdot 4} (3 + 2 + 3 + 2 + 0) =  \dfrac{1}{2} \]
%				\end{column}%
%				\hfill%
%				\begin{column}{.50\textwidth}
%					\begin{figure}
%						\centering
%						\resizebox{!}{0.3\textheight}{\ejemplogradomedio}
%						\caption{Grafo de competitividad de $\mathcal{S}$}
%					\end{figure}
%					\[ \mathrm{ND}(\mathcal{S}) = \dfrac{1}{5\cdot 4} (2 + 4 + 2 + 3 + 3) =  \dfrac{7}{10} \]
%				\end{column}%
%			\end{columns}
%		\end{ejemplo}
%	\end{frame}
	
%	\subsubsection*{Fuerza media normalizada}
%	
%	\begin{frame}
%		\begin{center}
%			\Huge\textbf{\textsf{\textcolor{blue}{Fuerza media normalizada}}}
%		\end{center}
%	\end{frame}
%	
%	\begin{frame}{Fuerza media normalizada}
%		\begin{defi}
%			Se define la fuerza media normalizada de una familia de rankings $\mathcal{R}$ como la suma de todos pesos de las aristas del grafo de competitividad evolutivo $G_c^e(\mathcal{R})$ dividido entre la suma de todas las posibles aristas con sus posibles pesos. Esto es,
%			
%			\begin{equation}
%			\mathrm{NS}(\mathcal{R}) = \dfrac{w(E_\mathcal{R}^e)}{\binom{n}{2} (r-1)}
%			\end{equation} 
%			
%			donde $w(E_\mathcal{R}^e)$ denota la suma de todos los pesos de las aristas del grafo de competitividad evolutivo $G_c^e(\mathcal{R})$ y $r$ es el número de rankings de la familia $\mathcal{R}$.
%		\end{defi}
%		
%		\begin{defi}
%			Se dice que $\mathcal{R}$ es más competitivo que $\mathcal{S}$ con respecto a la fuerza media normalizada si $\mathrm{NS}(\mathcal{R}) > \mathrm{NS}(\mathcal{S})$.
%		\end{defi}
%	\end{frame}
	
%	\begin{frame}{Fuerza media normalizada}
%		\begin{ejemplo}
%			\begin{columns}[t] % align columns
%				\begin{column}{.50\textwidth}
%					\begin{figure}
%						\centering
%						\resizebox{!}{0.3\textheight}{\ejemplografocompetitividadevolutivo}
%						\caption{Grafo de competitividad de $\mathcal{R}$}
%					\end{figure}
%					\[ \mathrm{NS}(\mathcal{R}) = \dfrac{7}{10 \cdot 3} = \dfrac{7}{30} \]
%				\end{column}%
%				\hfill%
%				\begin{column}{.50\textwidth}
%					\begin{figure}
%						\centering
%						\resizebox{!}{0.3\textheight}{\ejemplofuerzamedia}
%						\caption{Grafo de competitividad de $\mathcal{S}$}
%					\end{figure}
%					\[ \mathrm{NS}(\mathcal{S}) = \dfrac{11}{10\cdot 4} = \dfrac{11}{40} \]
%				\end{column}%
%			\end{columns}
%		\end{ejemplo}
%	\end{frame}
	
%	\subsubsection*{Tau de Kendall generalizada}
%	
%	\begin{frame}
%		\begin{center}
%			\Huge\textbf{\textsf{\textcolor{blue}{Tau de Kendall generalizada}}}
%		\end{center}
%	\end{frame}
	
%	\begin{frame}{Tau de Kendall}
%		\begin{defi}
%			Llamamos tau de Kendall entre dos rankings $c_1$ y $c_2$ con $n$ elementos cada uno al siguiente valor
%			
%			\begin{equation}
%			\tau(c_1, c_2) = \dfrac{\tilde{K}(c_1, c_2) - K(c_1, c_2)}{\binom{n}{2}}
%			\end{equation} 
%			
%			donde $\tilde{K}(c_1, c_2)$ es el número de pares $(i,j)$ que no compiten con respecto a $\mathcal{R} = \{c_1, c_2\}$, y $K(c_1, c_2)$ denota el número de pares que compiten en $\mathcal{R}$.
%		\end{defi}
%	\end{frame}
	
%	\begin{frame}{Tau de Kendall generalizada}
%		\begin{defi}
%			Se define tau de Kendall generalizada a la familia de rankings $\mathcal{R}$, con $r \geq 2$ de la siguiente manera:
%			
%			\begin{equation}
%			\tau(\mathcal{R}) = \dfrac{\tilde{K}(c_1, c_2) - K(c_1, c_2)}{\binom{n}{2}} = 1 - \dfrac{4 |E_\mathcal{R}|}{n(n-1)}
%			\end{equation}
%			
%			donde $|E_\mathcal{R}|$ es el número de aristas del grafo de competitividad $G_c(\mathcal{R})$.
%		\end{defi}
%		
%	\end{frame}
	
	\begin{frame}{Tau de Kendall evolutiva}
		\begin{defi}
			Se define tau de Kendall evolutiva de la familia de rankings $\mathcal{R}$ con $r \geq 2$ rankings de la siguiente manera:
			
			\begin{equation}
			\tau_e(\mathcal{R}) = 1 - \dfrac{2 w(E_\mathcal{R}^e)}{\binom{n}{2}(r-1)}
			\end{equation} 
			
			donde $w(E_\mathcal{R}^e)$ denota la suma de todos los pesos de las aristas del grafo de competitividad evolutivo.
		\end{defi}
		
		\begin{defi}
			Decimos que la familia de rankings $\mathcal{R}$ es más competitiva con respecto a la tau de Kendall evolutiva que la familia de rankings $\mathcal{S}$ si $\tau_e(\mathcal{R}) < \tau_e(\mathcal{S})$.
		\end{defi}
	\end{frame}
	
	\begin{frame}{Conclusiones}
	\begin{itemize}
		\item Existen multitud de métodos matemáticos para el estudio de rankings que resuelven los problemas clásicos de rankings (agregación de rankings, comparación de rankings, etc).
		
		\item Las medidas de competitividad dan información valiosa sobre el grafo de competitividad.
	\end{itemize}
	\end{frame}
%	
%	\begin{frame}{Tau de Kendall generalizada}
%		\begin{ejemplo}
%			\begin{columns}[t] % align columns
%				\begin{column}{.50\textwidth}
%					\begin{figure}
%						\centering
%						\resizebox{!}{0.3\textheight}{\ejemplografocompetitividadevolutivo}
%						\caption{Grafo de competitividad de $\mathcal{R}$}
%					\end{figure}
%					\[ \tau_e(\mathcal{R}) = 1 - \dfrac{4 \cdot 7}{5 \cdot 4} = -\dfrac{2}{5} \]
%				\end{column}%
%				\hfill%
%				\begin{column}{.50\textwidth}
%					\begin{figure}
%						\centering
%						\resizebox{!}{0.3\textheight}{\ejemplofuerzamedia}
%						\caption{Grafo de competitividad de $\mathcal{S}$}
%					\end{figure}
%					\[ \tau_e(\mathcal{S}) = 1 - \dfrac{11 \cdot 2}{10 \cdot 4} = -\dfrac{9}{20} \]
%				\end{column}%
%			\end{columns}
%		\end{ejemplo}
%	\end{frame}
	
	%\subsubsection*{Diámetro del grafo de competitividad evolutivo}
	
	%\begin{frame}
		%\begin{center}
			%\Huge\textbf{\textsf{\textcolor{blue}{Diámetro del grafo de competitividad evolutivo}}}
		%\end{center}
	%\end{frame}
	
	%\begin{frame}{Diámetro del grafo de competitividad evolutivo}
		%\begin{defi}
			%Se define diámetro del grafo de competitividad evolutivo de la familia de rankings $\mathcal{R}$ de la siguiente forma:
			
			%\begin{equation}
			%\mathrm{diam}(\mathcal{R}) = \max \{ d(i,j) : i \neq j \}
			%\end{equation}
			
			%donde $d(i,j)$ es la distancia entre el nodo $i$ y el nodo $j$.
		%\end{defi}
		
		%\begin{defi}
			%Diremos que la familia de rankings $\mathcal{R}$ es más competitiva que la familia de rankings $\mathcal{S}$ con respecto al diámetro del grafo de competitividad evolutivo si $\mathrm{diam}(\mathcal{R}) < \mathrm{diam}(\mathcal{S})$.
		%\end{defi}
	%\end{frame}
	
	%\begin{frame}{Diámetro del grafo de competitividad evolutivo}
		%\begin{ejemplo}
			%\begin{columns}[t] % align columns
				%\begin{column}{.50\textwidth}
					%\begin{figure}
						%\centering
						%\resizebox{!}{0.4\textheight}{\ejemplografocompetitividadevolutivo}
						%\caption{Grafo de competitividad de $\mathcal{R}$}
					%\end{figure}
				%\end{column}%
				%\hfill%
				%\begin{column}{.50\textwidth}
					%\begin{figure}
						%\centering
						%\resizebox{!}{0.4\textheight}{\ejemplofuerzamedia}
						%\caption{Grafo de competitividad de $\mathcal{S}$}
					%\end{figure}
				%\end{column}%
			%\end{columns}
		%\end{ejemplo}
	%\end{frame}
	
	%\begin{frame}{Diámetro del grafo de competitividad evolutivo}
		%\begin{ejemplo}[continuación]
			%\begin{columns}[t] % align columns
				%\begin{column}{.50\textwidth}
					%\begin{equation*}
					%\bordermatrix{
						%& 1 & 2 & 3 & 4 & 5 \cr
						%1 &	0 & 0 & 1 & 1 & \infty \cr
						%2 & 1 & 0 & 1 & 2 & \infty \cr
						%3 & 1 & 1 & 0 & 2 & \infty \cr
						%4 & 1 & 2 & 2 & 0 & \infty \cr
						%5 & \infty & \infty & \infty & \infty  &  0  \cr
					%}
					%\end{equation*}
					%\[ \mathrm{diam}(\mathcal{R}) = \infty \]
				%\end{column}%
				%\hfill%
				%\begin{column}{.50\textwidth}
					%\begin{equation*}
					%\bordermatrix{
						%& 1 & 2 & 3 & 4 & 5 \cr
						%1 &	0 & 2 & 4 & 1 & 3 \cr
						%2 & 2 & 0 & 2 & 1 & 1 \cr
						%3 & 4 & 2 & 0 & 3 & 1 \cr
						%4 & 1 & 1 & 3 & 0 & 2 \cr
						%5 & 3 & 1 & 1 & 2 & 0  \cr
					%}
					%\end{equation*}
					%\[ \mathrm{diam}(\mathcal{S}) = 4 \]
				%\end{column}%
			%\end{columns}
		%\end{ejemplo}
	%\end{frame}
	
	%\subsubsection*{Longitud del camino característico}
	
	%\begin{frame}
		%\begin{center}
			%\Huge\textbf{\textsf{\textcolor{blue}{Longitud del camino característico}}}
		%\end{center}
	%\end{frame}
	
	%\begin{frame}{Longitud del camino característico}
		%\begin{defi}
			%Se define longitud del camino característico del grafo de competitividad evolutivo de la familia de rankings $\mathcal{R}$ de la siguiente forma:
			
			%\begin{equation}
			%\mathrm{L}(\mathcal{R}) = \dfrac{1}{\binom{n}{2}} \sum_{i\neq j} d(i,j)
			%\end{equation}
			
			%donde $d(i,j)$ es la distancia entre el nodo $i$ y el nodo $j$.
		%\end{defi}
		
		%\begin{defi}
			%Diremos que la familia de rankings $\mathcal{R}$ es más competitiva que la familia de rankings $\mathcal{S}$ con respecto a la longitud del camino característico del grafo de competitividad evolutivo si $\mathrm{L}(\mathcal{R}) < \mathrm{L}(\mathcal{S})$.
		%\end{defi}
	%\end{frame}
	
	%\begin{frame}{Longitud del camino característico}
	%\begin{ejemplo}
		%\begin{columns}[t] % align columns
			%\begin{column}{.50\textwidth}
				%\begin{figure}
					%\centering
					%\resizebox{!}{0.4\textheight}{\ejemplografocompetitividadevolutivo}
					%\caption{Grafo de competitividad de $\mathcal{R}$}
				%\end{figure}
			%\end{column}%
			%\hfill%
			%\begin{column}{.50\textwidth}
				%\begin{figure}
					%\centering
					%\resizebox{!}{0.4\textheight}{\ejemplofuerzamedia}
					%\caption{Grafo de competitividad de $\mathcal{S}$}
				%\end{figure}
			%\end{column}%
		%\end{columns}
	%\end{ejemplo}
	%\end{frame}
	
	%\begin{frame}{Longitud del camino característico}
%		\begin{ejemplo}[continuación]
%			\begin{columns}[t] % align columns
%				\begin{column}{.50\textwidth}
%					\begin{equation*}
%					\bordermatrix{
%						& 1 & 2 & 3 & 4 & 5 \cr
%						1 &	0 & 0 & 1 & 1 & \infty \cr
%						2 & 1 & 0 & 1 & 2 & \infty \cr
%						3 & 1 & 1 & 0 & 2 & \infty \cr
%						4 & 1 & 2 & 2 & 0 & \infty \cr
%						5 & \infty & \infty & \infty & \infty  &  0  \cr
%					}
%					\end{equation*}
%					\[ \mathrm{L}(\mathcal{R}) = \infty \]
%				\end{column}%
%				\hfill%
%				\begin{column}{.50\textwidth}
%					\begin{equation*}
%					\bordermatrix{
%						& 1 & 2 & 3 & 4 & 5 \cr
%						1 &	0 & 2 & 4 & 1 & 3 \cr
%						2 & 2 & 0 & 2 & 1 & 1 \cr
%						3 & 4 & 2 & 0 & 3 & 1 \cr
%						4 & 1 & 1 & 3 & 0 & 2 \cr
%						5 & 3 & 1 & 1 & 2 & 0  \cr
%					}
%					\end{equation*}
%					\[ \mathrm{L}(\mathcal{S}) = \dfrac{41}{10} \]
%				\end{column}%
%			\end{columns}
%		\end{ejemplo}
%	\end{frame}
	
%	\subsubsection*{Eficiencia del grafo de competitividad}
%	
%	\begin{frame}
%		\begin{center}
%			\Huge\textbf{\textsf{\textcolor{blue}{Eficiencia del grafo de competitividad}}}
%		\end{center}
%	\end{frame}
%	
%	\begin{frame}{Eficiencia del grafo de competitividad}
%		\begin{defi}
%			Se define eficiencia del grafo de competitividad evolutivo de la familia de rankings $\mathcal{R}$ de la siguiente forma:
%			
%			\begin{equation}
%			\mathrm{E}(\mathcal{R}) = \dfrac{1}{\binom{n}{2}} \sum_{i\neq j} \dfrac{1}{d(i,j)}
%			\end{equation}
%			
%			donde $d(i,j)$ es la distancia entre el nodo $i$ y el nodo $j$.
%		\end{defi}
%		
%		La eficiencia es la media armónica de todos los caminos del grafo de competitividad.
%		
%		\begin{defi}
%			Diremos que la familia de rankings $\mathcal{R}$ es más competitiva que la familia de rankings $\mathcal{S}$ con respecto a la eficiencia del grafo de competitividad evolutivo si $\mathrm{E}(\mathcal{R}) > \mathrm{E}(\mathcal{S})$.
%		\end{defi}
%	\end{frame}
%	
%	\begin{frame}{Eficiencia del grafo de competitividad}
%		\begin{ejemplo}
%			\begin{columns}[t] % align columns
%				\begin{column}{.50\textwidth}
%					\begin{figure}
%						\centering
%						\resizebox{!}{0.4\textheight}{\ejemplografocompetitividadevolutivo}
%						\caption{Grafo de competitividad de $\mathcal{R}$}
%					\end{figure}
%				\end{column}%
%				\hfill%
%				\begin{column}{.50\textwidth}
%					\begin{figure}
%						\centering
%						\resizebox{!}{0.4\textheight}{\ejemplofuerzamedia}
%						\caption{Grafo de competitividad de $\mathcal{S}$}
%					\end{figure}
%				\end{column}%
%			\end{columns}
%		\end{ejemplo}
%	\end{frame}
%	
%	\begin{frame}{Eficiencia del grafo de competitividad}
%		\begin{ejemplo}[continuación]
%			\begin{columns}[t] % align columns
%				\begin{column}{.50\textwidth}
%					\begin{equation*}
%					\bordermatrix{
%						& 1 & 2 & 3 & 4 & 5 \cr
%						1 &	0 & 0 & 1 & 1 & \infty \cr
%						2 & 1 & 0 & 1 & 2 & \infty \cr
%						3 & 1 & 1 & 0 & 2 & \infty \cr
%						4 & 1 & 2 & 2 & 0 & \infty \cr
%						5 & \infty & \infty & \infty & \infty  &  0  \cr
%					}
%					\end{equation*}
%					\[ \mathrm{E}(\mathcal{R}) = 1 \]
%				\end{column}%
%				\hfill%
%				\begin{column}{.50\textwidth}
%					\begin{equation*}
%					\bordermatrix{
%						& 1 & 2 & 3 & 4 & 5 \cr
%						1 &	0 & 2 & 4 & 1 & 3 \cr
%						2 & 2 & 0 & 2 & 1 & 1 \cr
%						3 & 4 & 2 & 0 & 3 & 1 \cr
%						4 & 1 & 1 & 3 & 0 & 2 \cr
%						5 & 3 & 1 & 1 & 2 & 0  \cr
%					}
%					\end{equation*}
%					\[ \mathrm{E}(\mathcal{S}) = 1.28 \]
%				\end{column}%
%			\end{columns}
%		\end{ejemplo}
%	\end{frame}
	
	\begin{frame}
			\begin{center}
				\Huge\textbf{\textsf{\textcolor{blue}{Trabajo Fin de Grado de Ingeniería del Software}}} \\ 
				\vspace*{1em}
				\huge\textbf{\textsf{\textcolor{black}{Diseño e implementación de una aplicación SPA para el estudio de la competitividad de la liga BBVA}}}
			\end{center}
		\end{frame}
			
		\begin{frame}{Objetivos}
			\begin{itemize}
				\item Desarrollar una aplicación web que implementa métodos matemáticos para el análisis de rankings.
				\item Aplicar la aplicación para estudiar las últimas cuatro temporadas de la liga BBVA.
			\end{itemize}
		\end{frame}
	
	\section{Diseño de la aplicación}
	
	\begin{frame}
		\begin{center}
			\Huge\textbf{\textsf{\textcolor{blue}{Diseño de la aplicación}}}
		\end{center}
	\end{frame}
	
	\subsection*{Arquitectura de la aplicación}
	
	\begin{frame}
		\begin{center}
			\Huge\textbf{\textsf{\textcolor{blue}{Arquitectura de la aplicación}}}
		\end{center}
	\end{frame}
	
	\begin{frame}{Arquitectura de la aplicación}
		\begin{figure}
			\centering
			\resizebox{!}{0.7\textheight}{\arquitectura}
			\caption{Arquitectura de la aplicación}
			\label{fig:arquitectura}
		\end{figure}
	\end{frame}
	
	\subsection*{Tecnología utilizada}
	
	\begin{frame}
		\begin{center}
			\Huge\textbf{\textsf{\textcolor{blue}{Tecnología utilizada}}}
		\end{center}
	\end{frame}
	
	\begin{frame}{Tecnología utilizada}
		\begin{figure}
			\centering
			\resizebox{!}{0.7\textheight}{\tecnologia}
			\caption{Tecnología de la aplicación}
			\label{fig:tecnologia}
		\end{figure}
	\end{frame}
	
	\subsection*{Vista de la aplicación}
	
	\begin{frame}
		\begin{center}
			\Huge\textbf{\textsf{\textcolor{blue}{Vista de la aplicación}}}\\
			
			\begin{figure}
			\centering
			\includegraphics[width=0.5\linewidth]{./imagenes/inicio}
			\end{figure}
	
			\Huge\textbf{\textsf{{[DEMO]}}}
		\end{center}
	\end{frame}
	

	\section{Estudio de la competitividad de la Liga BBVA de las últimas cuatro temporadas}
	
	\begin{frame}
		\begin{center}
			\Huge\textbf{\textsf{\textcolor{blue}{Estudio de la competitividad de la Liga BBVA de las últimas cuatro temporadas}}}
		\end{center}
	\end{frame}	
	
	\begin{frame}{Estudio de la competitividad de la Liga BBVA de las últimas cuatro temporadas}	
		\begin{figure}
			\centering
			\subfloat[Temporada 2011-2012]{
				\includegraphics[width=0.3\linewidth]{imagenes/pantallazos-competitividad/2011-2012/grafo}
			}
			\subfloat[Temporada 2012-2013]{
				\includegraphics[width=0.3\linewidth]{imagenes/pantallazos-competitividad/2012-2013/grafo}
			}
			\hspace{0mm}
			\subfloat[Temporada 2013-2014]{
				\includegraphics[width=0.3\linewidth]{imagenes/pantallazos-competitividad/2013-2014/grafo}
			}
			\subfloat[Temporada 2014-2015]{
				\includegraphics[width=0.3\linewidth]{imagenes/pantallazos-competitividad/2014-2015/grafo}
			}
					
			\caption{Grafos de competitividad de las cuatro últimas temporadas de la liga BBVA}

		\end{figure}
	\end{frame}
	
	
	\begin{frame}{Estudio de la competitividad de la Liga BBVA de las últimas cuatro temporadas}	

%		\begin{table}[h]
%			\centering
%			\caption[Medidas de competitividad de las últimas cuatro temporadas]{Medidas de competitividad de las temporadas 2011-2012, 2012-2013, 2013-2014 y 2014-2015 de la Liga BBVA. Las medidas son fuerza media normalizada (NMS), el diámetro, la tau de Kendall, el grado medio normalizado (NMD), la longitud del camino característico (CPL) y la eficiencia.}
%			\label{tbl:medidas}
%			\begin{adjustbox}{max width=\textwidth}
%			\begin{tabular}{@{}ccccccc@{}}
%				\toprule
%				Temporada & NMS & Diámetro & Tau de Kendall & NMD & CPL & Eficiencia \\ \midrule
%				2014-2015 & 0.642                    & 3        & 0.054          & 0.891                   & 2.768                              & 1.633      \\
%				2013-2014 & 0.594                    & 3        & 0.059          & 0.880                   & 2.9368                             & 1.573      \\
%				2012-2013 & 0.673                    & $\infty$ & 0.061          & 0.876                   & $\infty$                           & 1.574      \\
%				2011-2012 & 0.636                    & 3        & 0.066          & 0.867                   & 2.789                              & 1.626      \\ \bottomrule
%			\end{tabular}
%			\end{adjustbox}
%		\end{table}
%		
%		La temporada 2014-2015 es la más competitiva de las cuatro estudiadas.

\begin{figure}
\centering
\subfloat[Diámetro]{
	\includegraphics[width=0.3\linewidth]{./imagenes/pantallazos-competitividad/Medida1}
}
\subfloat[Tau de Kendall]{
	\includegraphics[width=0.3\linewidth]{./imagenes/pantallazos-competitividad/Medida2}
}
\subfloat[NMS]{
	\includegraphics[width=0.3\linewidth]{./imagenes/pantallazos-competitividad/Medida3}
}
\hspace{0mm}
\subfloat[NMD]{
	\includegraphics[width=0.3\linewidth]{./imagenes/pantallazos-competitividad/Medida4}
}
\subfloat[CPL]{
	\includegraphics[width=0.3\linewidth]{./imagenes/pantallazos-competitividad/Medida5}
}
\subfloat[Eficiencia]{
	\includegraphics[width=0.3\linewidth]{./imagenes/pantallazos-competitividad/Medida6}
}


\end{figure}
		
	\end{frame}
	
	\begin{frame}{Estudio de la competitividad de la Liga BBVA de las últimas cuatro temporadas}
	
	Haciendo agregación de rankings,
	
	\vspace*{2em}
	
	\begin{equation*}
	c_a = \left( \begin{array}{c}
					2014-2015 \\
					2013-2014 \\
					2011-2012 \\
					2012-2013
					\end{array} \right)
	\end{equation*}
	
	\vspace*{2em}
	
	La temporada 2014-2015 es la más competitiva de las últimas cuatro.
	
	
	
	\end{frame}
	
	\section{Conclusiones}
		
		\begin{frame}
			\begin{center}
				\Huge\textbf{\textsf{\textcolor{blue}{Conclusiones}}}
			\end{center}
		\end{frame}
		
		\begin{frame}{Conclusiones}
				\begin{itemize}
					\item Las medidas de competitividad nos ayudan a entender propiedades del grafo de competitividad.
					
					\item Y junto con la aplicación web implementada permiten visualizar estas medidas de forma gráfica.
					
					\item La temporada 2014-2015 ha sido la más competitiva de las últimas cuatro en la liga BBVA.
					
				\end{itemize}
			\end{frame}
			
	
	\section{Futuro trabajo}
	
	\begin{frame}
		\begin{center}
			\Huge\textbf{\textsf{\textcolor{blue}{Futuro trabajo}}}
		\end{center}
	\end{frame}	
	
	
	\begin{frame}{Futuro trabajo}
	
		\begin{itemize}
		\item TFG Matemáticas
		
		\begin{itemize}
					\item Añadir un modelo de predicción para ver cómo evolucionará la competitividad a lo largo del tiempo.
					
					\item Estudiar la competitividad por zonas del ranking, es decir, dividir cada ranking en distintas regiones y estudiar la competitividad en cada una de estas regiones. 
				\end{itemize}
				
		\item TFG Ingeniería del Software
		
			\begin{itemize}
							\item Posibilidad de ver la competitividad a lo largo de las jornadas, es decir, ver las medidas y el grafo de competitividad en una jornada concreta durante la temporada.
												
							\item Posibilidad de implementar una aplicación móvil con el framework con Ionic Framework.
							
							\item Añadir un módulo de comparación, que permita comparar dos o más temporadas distintas.
							
						\end{itemize}
		
		\end{itemize}
		
		
	\end{frame}
	
	
	\begin{frame}{¿Preguntas?}
		\begin{figure}
			\centering
			\includegraphics[width=0.9\linewidth]{imagenes/preguntas}
		\end{figure}

	\end{frame}
	
	
\end{document}