\documentclass[10pt,a4paper]{article}
\usepackage[utf8]{inputenc}
\usepackage{amsmath}
\usepackage{amsfonts}
\usepackage{amssymb}
\usepackage{makeidx}
\usepackage{graphicx}
\usepackage[width=15.00cm, height=22.00cm]{geometry}
\usepackage[spanish]{babel}
\author{José Ignacio Escribano}
\title{Notas}
\begin{document}
\maketitle

\section{Notas}

\begin{enumerate}
\item A new method for comparing rankings through complex networks: Model and analysis of competitiveness of major European soccer leagues.
	\begin{itemize}
		\item Dos equipos compiten cuando intercambian sus posiciones relativas en rankings consecutivos.
		\item Permite definir un grafo uniendo los equipos que compiten.
		\item Algunas propiedades estructurales del grafo de competitividad (GC o CG): ``grado medio'', ``la fuerza media'' y ``coeficiente de clustering''.
		\item Generalización del coeficiente de correlación de Kendall para más de dos rankings.
		\item Para estudiar el ranking de una liga deportiva podemos usar las medidas estadísticas ordinarias como media o desviación típica. Normalmente ésto se hace.
		\item Nos interesa el comportamiento dinámico de una liga deportiva, por lo que necesitamos comparar $r \geq 2$ rankings. 
		\item Estudios de comparación de $r$ rankings pueden ser ``rastreados'' hasta el artículo de Kendall, donde se define el coeficiente de ``concordancia'' de Kendall. 
		\item Trabajos previos pusieron el foco en la correlación de sólo dos rankings. Ver Ref. 3, donde el coeficiente de correlación $\tau$ de Kendall se define. 
		\item Podemos distinguir tres formas de comparar dos rankings:
		
			\begin{enumerate}
				\item \label{correlacion} Usar un coeficiente de correlación, como por ejemplo, la $\rho$ de Spearman o la $\tau$ de Kendall.
				\item \label{distancia} Usar una distancia entre los rankings como la ``footrule'' $D$ de Spearman (ver Ref. 5) u otras métricas (ver Ref. 6).
				\item Usar el coeficiente de ``concordancia'' de Kendall (ver Ref. 7).
			\end{enumerate}
		\item \ref{correlacion} y \ref{distancia} son equivalentes (ver Ref. 8).
		\item Clasificación de los rankings atendiendo a su dinámica y diferentes tipos de rankings aplicados a varios deportes (ver Ref. 9--14). 
		\item La definición teórica de GC y su relación con otros objetos conocidos de la teoría de grafos se puede encontrar en Ref. 15.
		\item El interés por la competitividad viene del concepto ``grupo competitivo'' (Ref. 20), que está relacionado con rankings de usuarios en redes sociales basadas en PageRank Personalizado (PPR).
		
	\end{itemize}
\item Comparing rankings by means of competitivity graphs: structural properties and computation.
\end{enumerate}

\section{Bibliografía}

\end{document}