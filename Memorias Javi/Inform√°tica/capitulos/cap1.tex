\chapter{Introducción}
Los rankings son algo que forma parte de nuestra vida cotidiana: cuando en el telediario analizan los resultados del CIS (Centro de Investigaciones Sociológicas) en cuanto a la valoración de los líderes políticos, cuando estudiamos el ranking de universidades de España para decidir cuál es la mejor para nuestro futuro, cuando escuchamos en la radio el top 10 de éxitos musicales del momento, cuando antes de echar la Quiniela para la próxima jornada consultamos la posición de cada equipo en la clasificación para decidir por quién apostar que ganará cada partido, incluso cuando hacemos una búsqueda en Google y accedemos al primer enlace que es obtenido a partir del algoritmo PageRank. Estos son solo unos pocos ejemplos de la cantidad de rankings que existen en nuestra sociedad. Y siempre que hablamos de rankings viene asociado el concepto de rating, que es el criterio usado en la ordenación de los elementos del ranking, por ejemplo, los puntos que llevan los distintos equipos en cualquier competición deportiva.\\

El proyecto consiste en crear un módulo de predicción de resultados para la próxima jornada de la Liga BBVA estimando las probabilidades de que gane el equipo local, el visitante o de que empaten en cada partido que se disputará. Este módulo formará parte de una aplicación ya existente que entre otras funcionalidades tiene: obtener estadísticas de las distintas temporadas y equipos de la Liga española de fútbol desde 1928 o el estudio de la competitividad en el fútbol.\\ 

En el capítulo 2 repasaremos todos los conceptos matemáticos relativos a rankings que debemos conocer, ya que serán usados en la aplicación informática.
En él recordaremos la definición matemática de ranking y rating y dos métodos de comparación de rankings que usaremos para calcular la tasa de aciertos de los distintos modelos predictivos en capítulos posteriores.\\ 

En el capítulo 3 plantearemos dos modelos matemáticos para predecir cómo quedará el ranking de la próxima jornada de la Liga. Uno basado en la posición que ocupan los equipos en el ranking y otro basado en su tendencia de los últimos enfrentamientos. Finalmente propondremos uno que consista en combinar ambos. Dentro de cada uno de los modelos existirán distintas alternativas de implementación que propondremos con el objetivo de quedarnos con las que mejor se aproximen al resultado real. 
\newpage
El cuarto capítulo constará de la descripción informática del proyecto. En él comenzaremos presentando la herramienta que ya existe, a la que vamos a incorporar el  módulo de predicción. Posteriormente explicaremos la arquitectura de la aplicación y enumeraremos todas las tecnologías usadas para el proyecto. Finalizaremos mencionando los aspectos más relevantes de la implementación de la aplicación.\\

En el quinto explicaremos el funcionamiento del módulo implementado haciendo uso de la aplicación para realizar la predicción de la próxima jornada.\\

En el último capítulo mostraremos los resultados obtenidos por la aplicación en las últimas temporadas. En él obtendremos distintas estadísticas y compararemos los distintos modelos matemáticos propuestos eligiendo las alternativas de implementación que tengan mayor porcentaje de aciertos.\\

