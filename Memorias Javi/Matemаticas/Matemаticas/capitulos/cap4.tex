\chapter{Conclusiones.}
Comenzamos el segundo capítulo diferenciando qué es un ranking y qué un rating y viendo la relación entre ambos términos (recordemos que los ratings se utilizan para generar rankings, pero nunca al revés). Presentamos cuatro modelos con los cuales generamos nuevos rankings para nuestros ejemplos: dos basados en el número de victorias de los equipos (los métodos de Colley y Keener) y otros dos basados en las puntuaciones (los métodos de Massey y Ataque-Defensa).
Después vimos tres formas de combinar varios rankings con el objetivo de generar nuevos rankings más robustos: haciendo la media de las posiciones que ocupa cada elemento en los distintos rankings, un sistema de votación usado en elecciones conocido como la suma de Borda y un método de optimización que trata de maximizar la concordancia entre los rankings de entrada con una serie de restricciones. 
Una vez creados varios rankings distintos comenzamos a compararlos para ver el grado de coincidencia entre ellos. La tau de Kendall estudia las concordancias en la posición relativa de los equipos mientras que la rho de Spearman estudia la distancia entre las posiciones que ocupan los equipos en los rankings. 
Finalmente hicimos un estudio de cómo los equipos van intercambiando sus posiciones en los rankings introduciendo el concepto de competitividad con su correspondiente representación en los grafos de competitividad.\\

En el tercer capítulo propusimos una aplicación de rankings en la que poder aplicar los conceptos aprendidos en el capítulo anterior. Empezamos introduciendo lo que se ha hecho hasta ahora en investigación relativo a la predicción de resultados deportivos. Posteriormente se plantearon distintos modelos para intentar predecir cómo acabará el ranking tras una jornada que aún no ha comenzado a disputarse. Para ello hay que obtener las probabilidades de que gane el equipo local, el visitante o empaten para cada partido de esa jornada. El primero de los modelos halla las probabilidades en base a la posición de los equipos en el ranking. El segundo las halla teniendo en cuenta la tendencia de los equipos en los últimos partidos (diferenciando los partidos jugados en casa y fuera de ella). Finalmente conjugamos ambos modelos mediante una combinación convexa para la cuál tendremos que optimizar el valor que debe tomar $\lambda$ entrenándola con un histórico de datos.\\

Lo que querremos realizar en un futuro es llevar a la práctica estos modelos de predicción. Lo ideal sería tener varios modelos de interpolación y jugar con distintas funciones memoria en el modelo de tendencias para ver cuál de las distintas variantes tendría mayor tasa de aciertos y realiza una predicción más aproximada al ranking final una vez acabe la jornada en cuestión. También habrá que estudiar el histórico de datos para obtener el valor de $\lambda$ que determine la mejor combinación de los modelos de interpolación y tendencias para tener el mayor número de aciertos.
