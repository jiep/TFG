\chapter{Introducción y objetivos}
Los rankings son algo que forma parte de nuestra vida cotidiana: cuando en el telediario analizan los resultados del CIS (Centro de Investigaciones Sociológicas) en cuanto a la valoración de los líderes políticos, cuando estudiamos el ranking de universidades de España para decidir cuál es la mejor para nuestro futuro, cuando escuchamos en la radio el top 10 de éxitos musicales del momento, cuando antes de echar la Quiniela para la próxima jornada consultamos la posición de cada equipo en la clasificación para decidir por quién apostar que ganará cada partido, incluso cuando hacemos una búsqueda en Google y accedemos al primer enlace que es obtenido a partir del algoritmo PageRank\footnote{El sistema PageRank es utilizado por Google para determinar la importancia o relevancia de una página. Interpreta un enlace de una página A a una página B como un voto, de la página A, para la página B. Los votos emitidos por las páginas consideradas importantes (con un PageRank elevado) valen más, y ayudan a hacer a otras páginas importantes.}. Estos son solo unos pocos ejemplos de la cantidad de rankings que existen en nuestra sociedad.\\

Y siempre que hablamos de rankings viene asociado el concepto de rating, que es el criterio usado en la ordenación de los elementos del ranking, por ejemplo, la puntuación numérica del 0 al 10 que obtienen los políticos al hacer la media de las valoraciones de múltiples usuarios en los barómetros del CIS, los puntos que llevan los distintos equipos en cualquier competición deportiva suponiendo que solo obtienen puntos cuando ganan o empatan, etc.\\

El objetivo principal de este proyecto es proponer modelos de predicción de resultados deportivos. Para ello debemos comenzar presentando distintas herramientas matemáticas existentes para trabajar con rankings y ratings.\\

A menudo no solo nos interesará el orden en que se disponen los elementos en un ranking, sino que querremos hacer estudios más profundos y nos interesará combinar varios rankings para obtener otros más robustos, comparar varios rankings obtenidos en base a distintos criterios para ver cuanto coinciden o difieren, estudiar la evolución de los rankings u otras operaciones que analizaremos en el capítulo 2.\\

Cada concepto, método o herramienta explicado vendrá acompañado de un ejemplo. Los ejemplos usarán los resultados de la fase regular\footnote{En la primera fase o fase regular de la Liga ACB cada uno de los 18 equipos se enfrenta dos veces a todos los demás, una vez como local y otra como visitante. Al final de la fase regular, los ocho primeros clasificados se enfrentan en los playoffs por el título.} de la Liga Endesa de Baloncesto (ACB) del año 2014/2015 (las tablas de resultados usadas se incluirán en un anexo por si el lector deseara realizar alguno de los ejemplos por su cuenta). Pero... \textit{¿Por qué baloncesto?}

\begin{itemize}
	\item A diferencia de otros deportes, un partido no puede acabar empate, siempre tiene que haber un vencedor del enfrentamiento, lo que nos evitará problemas al aplicar determinados métodos de los que explicaremos en el capítulo 2. 
	\item A diferencia de otros deportes, no se ``valora'' el número de goles metidos (que suelen ser pocos por partido) sino puntuaciones generalmente altas (que suelen acabar por encima de los 65 puntos por equipo en cada partido), lo que nos permitirá obtener mejores rankings aplicando algunos de los métodos que veremos más adelante. 
\end{itemize}

Tras conocer todos los conceptos y algunas de las herramientas más importantes para trabajar con rankings, en el capítulo 3 haremos un estudio sobre predicción de resultados deportivos, con el objetivo de aplicarlo posteriormente a la Liga BBVA de fútbol. Queremos predecir cómo quedará el ranking de la próxima jornada (que aún no se ha disputado) estimando las probabilidades de que gane el equipo local, el visitante o de que empaten en cada partido que se disputará. Veremos qué estudios e investigaciones se han hecho al respecto y se plantearán varios modelos en función a distintos criterios que pensamos influirán en el resultado de los partidos.