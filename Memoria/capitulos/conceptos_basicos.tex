\chapter{Conceptos básicos}

En este capítulo veremos dos conceptos centrales de esta memoria: ranking y rating, y veremos varios métodos para crear rankings.\\

Para entender estos conceptos intuitivamente veamos un ejemplo:

\begin{ejemplo}
Si consideramos los siguientes jugadores de la NBA: Kareem Abdul-Jabbar, Karl Malone, Michael Jordan, Kobe Bryant y Wilt Chamberlain, y consideramos los puntos y rebotes\footnote{Datos extraídos de \url{http://stats.nba.com/leaders/alltime}} anotados por cada uno de estos jugadores a lo largo de su carrera en temporada regular, obtenemos la siguiente tabla:

\begin{table}[h]
\centering
\caption[Puntos y rebotes de los máximos anotadores de la NBA]{Puntos y rebotes de los máximos anotadores de la NBA en temporada regular}
\label{tbl:puntos_rebotes}
\begin{tabular}{@{}lcc@{}}
\cmidrule(l){2-3}
\multicolumn{1}{c}{}      & \multicolumn{2}{c}{En temporada regular} \\ \cmidrule(l){2-3} 
\multicolumn{1}{c}{}      & Puntos             & Rebotes             \\ \midrule
Kareem Adbul-Jabbar (KAJ) & 38387              & 17440               \\ \midrule
Karl Malone (KM)          & 36928              & 14968               \\ \midrule
Michael Jordan (MJ)       & 32292              & 6662                \\ \midrule
Kobe Bryant (KB)          & 32030              & 6672                \\ \midrule
Wilt Chamberlaint (WC)    & 31419              & 23924               \\ \bottomrule
\end{tabular}
\end{table}

Estas dos ordenaciones de los jugadores es lo que recibe el nombre de rating. De cada uno de estos ratings, se puede obtener un ranking que son las posiciones de cada jugador en cada rating. En el ejemplo,

\[
\begin{array}{ccc}
\text{Posición} & \text{Puntos} & \text{Rebotes}\\ 
\begin{array}{c}
\text{1}\\
\text{2}\\
\text{3}\\
\text{4}\\
\text{5}\\
\end{array} & \left(\begin{array}{c}
\text{KAJ}\\
\text{KM}\\
\text{MJ}\\
\text{KB}\\
\text{WC}\\
\end{array} \right) & \left(\begin{array}{c}
\text{WC}\\
\text{KAJ}\\
\text{KM}\\
\text{KB}\\
\text{MJ}\\
\end{array} \right)
\end{array}
\]

En general, los ratings producen distintos rankings. En el ejemplo, se puede ver como en el ranking de rebotes Wilt Chamberlain se encuentra en 1ª posición, mientras que en el de rebotes se encuentra en 5ª posición. Otros rankings ficticios para el ejemplo podrían ser

\[
\begin{array}{ccc}
\begin{array}{c}
\text{1}\\
\text{2}\\
\text{3}\\
\text{4}\\
\text{5}\\
\end{array} & \left(\begin{array}{c}
\text{MJ}\\
\text{KB}\\
\text{KM}\\
\text{KAJ}\\
\text{WC}\\
\end{array} \right) & \left(\begin{array}{c}
\text{KB}\\
\text{MJ}\\
\text{KM}\\
\text{WC}\\
\text{KAJ}\\
\end{array} \right)
\end{array}
\]

\end{ejemplo}

De forma intuitiva, podríamos decir que un rating es un criterio para ordenar una serie de elementos, y un ranking son todas los posibles ordenamientos de un un conjunto de elementos. De forma rigurosa, las definiciones anteriores quedan así:

\begin{defi} \label{def:ranking}
Dado un conjunto $\mathcal{N} = \{1,\dots,n\}$ que llamamos nodos, definimos el ranking $c$ como cualquier biyección $c : \mathcal{N} \to \mathcal{N}$.
\end{defi}

De acuerdo a esta definición, en el ejemplo, $\mathcal{N} = \{1,2,3,4,5\}$, donde $\text{KAJ}\equiv 1$, $\text{KM}\equiv 2$, $\text{MJ}\equiv 3$, $\text{KB}\equiv 4$, $\text{WC}\equiv 5$. Así, si consideramos el ranking de rebotes, entonces

\[ \begin{array}{rlll}
c: & \mathcal{N} & \to & \mathcal{N}\\
& 1 & \mapsto & 5\\
& 2 & \mapsto & 1\\
& 3 & \mapsto & 2\\
& 4 & \mapsto & 4\\
& 5 & \mapsto & 3\\
\end{array} \] 

que es claramente una biyección, por lo que $c$ es un ranking de acuerdo a la definición \ref{def:ranking}. De la misma forma se obtienen que los demás rankings también cumplen la definición.\\

Además, escribiremos $i \prec_c j$ cuando el nodo $i \in \mathcal{N}$ aparezca antes que el nodo $j \in \mathcal{N}$ en el ranking $c$.

En el ejemplo, el nodo $5 \prec_c 3$ ya que el nodo $5$, WC aparece más alto que el nodo $3$, MJ.

\begin{defi}
Dado un conjunto $\mathcal{N} = \{1,\dots, n\}$ de nodos, decimos que es un rating si a cada $i \in \mathcal{N}$ se le asigna un valor real. 
\end{defi}

Notar que un rating cuando se ordena (ascendente o descendentemente), crea un ranking.\\

En el ejemplo anterior, la cantidad de puntos y rebotes a lo largo de la carrera en temporada regular en la NBA son ratings, y dados éstos obtenemos un ranking. En estos dos casos, hemos ordenado descendentemente el rating, pero también podría ordenarse de forma ascendente como en el caso de número de pérdidas de balón por partido a lo largo de la temporada regular. 

\section{Métodos para obtener ratings (y rankings)}
\todo[backgroundcolor=red, inline]{Añadir pequeña introducción}
\subsection{Método de Massey}
El método de Massey fue creado por Kenneth Massey en 1997 con la intención de obtener rankings de los equipos universitarios de EEUU. Este método usa la teoría de mínimos cuadrados. La idea fundamental de este método se basa en la siguiente ecuación idealizada:

\begin{equation}
r_i - r_j = y_k
\end{equation} 

donde $y_k$ es el margen de victoria para el partido $k$, y $r_i$ y $r_j$ son los ratings de los equipos $i$ y $j$, respectivamente. En otras palabras, la diferencia de los ratings $r_i$ y $r_j$ de dos equipos predice idealmente el margen de victoria en un enfrentamiento entre estos dos equipos. \\

El objetivo es asociar un rating a cada equipo en una liga con $n$ equipos, donde $m$ es el total de partidos disputados hasta la fecha. No conocemos los ratings $r_i$, pero sí conocemos quién jugó con quién y el margen de victoria. De esta manera, podemos formar para cada partido $k$ un sistema lineal con $m$ ecuaciones y $n$ incógnitas:

\[\mathbf{X r } = \mathbf{y}\]

Los coeficientes de cada fila de la matriz $\mathbf{X}$ son casi todos nulos, excepto en la posición $i$ que tiene un $1$, y en la posición $j$ con un $-1$, para indicar que el equipo $i$ gana al equipo $j$ en un partido. Normalmente, $m \gg n$ que hace que el sistema sea incompatible y no tenga solución. Se puede obtener una solución aplicando las ecuaciones normales de los mínimos cuadrados, quedando el sistema $\mathbf{X^T X r} = \mathbf{X^T y}$. \\

Massey descubrió que debido a la estructura de $\mathbf{X}$ era ventajoso la utilización de la matriz $\mathbf{M} = \mathbf{X^T X}$. De hecho, esta matriz no necesita ser calculada. Se puede obtener fácilmente usando el hecho de que los elementos de la diagonal $\mathbf{M}_{ii}$ es el total de partidos jugados por el $i$ contra el equipo $j$, y los elementos $\mathbf{M}_{ij}$ con $(i \neq j)$ es la negación del número de partidos jugados por el equipo $i$ contra el equipo $j$. El elemento i-ésimo de parte derecha de la ecuación se puede calcular es la suma de las diferencias de puntos para cada partidos disputado por el equipo $i$, y así definimos $\mathbf{p} = \mathbf{X^T y}$.\\
Usando estas notaciones el sistema se transforma en el siguiente sistema:

\[ \mathbf{M r} = \mathbf{p} \]

\paragraph*{Propiedades de la matriz $\mathbf{M}$}

\begin{enumerate}
\item La matriz $\mathbf{M}$ es una matriz cuadrada de orden $n$.
\item $\mathbf{M}$ es una matriz diagonalmente dominante.
\item Sus filas suman $0$. Como consecuencia, sus columnas son linealmente dependientes. Esto hace que el sistema  $\mathbf{M r} = \mathbf{p}$ no tenga solución única cuando el $\rang(\mathbf{M}) < n$. El problema se puede solventar añadiendo una fila entera de $1$ y en la posición correspondiente de $\mathbf{p}$ añadir un $0$. Con la nueva fila, el sistema queda 
\begin{equation}
\overline{\mathbf{M}} \mathbf{r} = \overline{\mathbf{p}} \label{eq:massey_general}
\end{equation}
\end{enumerate}

\todo[backgroundcolor=red, inline]{Añadir ejemplo}

\subsubsection{Método de Massey avanzado}

El método de Massey avanzado crea dos nuevos vectores respecto al método anterior: el rating ofensivo $\mathbf{o}$ y el rating defensivo $\mathbf{d}$, con $\mathbf{r} = \mathbf{o} + \mathbf{d}$. Este método descompone $\mathbf{p}$ en otros dos vectores: $ \mathbf{p} = \mathbf{f} - \mathbf{a}$, donde $\mathbf{f}$ es el número total de puntos marcados por cada equipo, y $\mathbf{a}$ es el número total de puntos marcados contra cada equipo durante la temporada. También se descompone la matriz $\mathbf{M}$ como $\mathbf{M} = \mathbf{T} - \mathbf{P}$, donde $\mathbf{T}$ es una matriz diagonal que contiene el número total de partidos disputados por cada equipo, y $\mathbf{P}$ una matriz sin elementos en la diagonal que contiene el número de parejas de partidos entre los equipos a lo largo de la temporada. Haciendo las sustituciones al sistema $\mathbf{M r } = \mathbf{p}$, el sistema se convierte en

\begin{equation}
\mathbf{T o} - \mathbf{P o} + \mathbf{T d} - \mathbf{P d} = \mathbf{f} - \mathbf{a}
\end{equation}

que puede ser descompuesta en dos ecuaciones

\begin{eqnarray}
\mathbf{T o} - \mathbf{P d} = \mathbf{f}\\ \label{eq:massey_f}
\mathbf{P o} - \mathbf{T d} = \mathbf{a} \label{eq:massey}
\end{eqnarray} 

Sabiendo que $\mathbf{r} = \mathbf{o} + \mathbf{d}$, despejando $\mathbf{o}$ y sustituyendo en la ecuación \ref{eq:massey_f} obtenemos

\begin{equation}
(\mathbf{T} + \mathbf{P})\mathbf{d} = \mathbf{T r} - \mathbf{f} \label{eq:massey_d}
\end{equation}

Esta última ecuación nos permite calcular el vector $\mathbf{d}$ resolviendo el sistema. Notar que $\mathbf{r}$ ya ha sido calculado de la ecuación \ref{eq:massey_general}.

\todo[backgroundcolor=red, inline]{Añadir ejemplo}

\paragraph*{Resumen y algoritmo de cómputo}

\todo[backgroundcolor=red, inline]{Añadir tabla del resumen}

El algoritmo que permite calcular este método es el siguiente:

\begin{enumerate}
\item Resolver el sistema \ref{eq:massey_general}.
\item Resolver el sistema \ref{eq:massey_d}.
\item Calcular el vector $\mathbf{o} = \mathbf{r} - \mathbf{d}$.
\end{enumerate}

\section{Método de Colley}

\todo[backgroundcolor=red, inline]{Añadir información}
