\chapter{Agregación de rankings}

En el capítulo anterior hemos visto distintos métodos para crear ratings, y por tanto, rankings. Estos ratings crean distintos rankings para un mismo conjunto de datos, según el método que apliquemos. Estamos interesados en alguna forma de combinar estos rankings para crear uno sólo a partir de los rankings iniciales. Esto es lo que se conoce como agregación de rankings. \\

La idea es de la agregación de rankings es que dados $k$ rankings obtenidos por distintos métodos, obtener un sólo ranking a partir de los rankings iniciales (Figura \ref{fig:agregacion_rankings}). Naturalmente, la calidad del ranking agregado dependerá de la calidad de los rankings iniciales. Es decir, si partimos de unos rankings con buena calidad tendremos un buen ranking agregado. De la misma forma, con malos rankings iniciales, obtendremos malos rankings agregados. 

\begin{figure}[htb]
\centering
\agregacionrankings
\caption{Agregación de rankings}
\label{fig:agregacion_rankings}
\end{figure}

\section{Métodos de agregación de rankings}
\todo[backgroundcolor=red, inline]{Añadir pequeña introducción}
\subsection{Método de Borda}
El método de Borda fue creado por Jean-Charles de Borda y data de 1770. Borda intentaba agregar rankings de listas de candidatos de unas elecciones políticas. Para cada lista de candidatos, cada candidato recibía una puntuación igual al número de candidatos que le superaban. La puntuación de cada lista es sumada para cada candidato para crear un solo ranking, que se llama recuento de Borda. Los candidatos son ordenados en orden descendente según este método. Este método puede manejar rankings de entrada con empates. Además, también puede producir un ranking de salida que contenga empates.  Aunque este método es muy sencillo, tiene un gran problema: que es fácil manipulable.

\begin{ejemplo}\label{ej:borda}
Consideramos los rankings producidos por el método de Massey (Tabla \ref{tbl:massey_resultados}), de Colley (Tabla \ref{tbl:colley_resultados_sin_empates}) y de Markov (Tabla \ref{tbl:markov_resultados}).\\

Aplicamos a estos tres rankings el método de Borda. Para DEN, en los tres rankings se encuentra en 5ª y última posición, por lo que ha sido superado por cuatro equipos. Por lo que su recuento de borda será $4 + 4 + 4 = 12$. Para LAL, tanto el ranking de Massey como de Colley está 4º posición, por lo que ha sido superado por $3$ equipos, y en el ranking de Markov está en 2ª posición, por lo que ha sido superado por $1$ equipo. Así, su recuento de Borda será de $3 + 3 + 1 = 7$. Análogo para el resto de equipos. El resultado y el ranking agregado por el método de Borda se muestra en la Tabla \ref{tbl:borda_resultados}.

\begin{table}[h]
\centering
\caption{Resultados del método de Borda}
\label{tbl:borda_resultados}
\begin{tabular}{@{}ccc@{}}
\cmidrule(l){2-3}
    & Recuento de Borda & Ranking agregado \\ \midrule
DEN & $4 + 4 + 4 = 12$       & 5       \\
LAL & $3 + 3 + 1 =  7$       & 3       \\
SAS & $1 + 1 + 0 =  2$       & 1       \\
CHI & $2 + 2 + 3 =  7$       & 3       \\
NYK & $0 + 0 + 2 =  2$       & 1       \\ \bottomrule
\end{tabular}
\end{table}

Se producen dos empates que rompemos teniendo en cuenta su enfrentamiento directo. Así, CHI ganó en su enfrentamiento a LAL, por lo que gana el tercer puesto y LAL queda en cuarta posición. De la misma forma, SAS gana la primera posición y NYK se queda con la segunda posición del ranking. 

\end{ejemplo}


\begin{ejemplo}[Ejemplo de manipulación que puede sufrir el método. Adaptado de \cite{burger2005heart}]\label{ej:borda_elecciones} 

En un país de $76.2$ millones de habitantes, se convocan elecciones para Presidente, Ministro y Alcalde, ordenados de mayor a menor responsabilidad. Se presentan a estos cargos tres candicatos A, B y C. Todo votante elige a cada candicato según sus preferencias, pero elegiendo a cada candidato en algún cargo. En la elecciones se aplica el método de Borda. Los resultados de las elecciones se muestran en la Tabla \ref{tbl:borda_elecciones}.\\
 
\begin{savenotes}
\begin{table}[h]
\centering
\caption{Votaciones del Ejemplo \ref{ej:borda_elecciones}}
\label{tbl:borda_elecciones}
\begin{tabular}{@{}cccc@{}}
\toprule
Elección \textbackslash Nº de votos\footnote{En millones de votos} & $37.2$ & $10.7$ & $28.3$ \\ \midrule
Presidente           & A    & C    & C    \\
Ministro             & B    & A    & B    \\
Alcalde              & C    & B    & A    \\ \bottomrule
\end{tabular}
\end{table}
\end{savenotes}

Aplicando el recuento de Borda, el candidato A tendría $67.3$ millones de ``votos'', el candidato B, $74.4$ millones y, el candicato C, $86.9$ millones. Por tanto, A sería elegido como Presidente, B como Ministro y C como alcalde. Una decisión polémica, ya que A sólo ha sido votado por $37.2$ millones como Presidente y a C lo han votado $39$ millones. Esto demuestra la facilidad con la que puede ser manipulado este método.  

\end{ejemplo}

\subsection{Ranking promedio}
Otro método muy sencillo de agregación de rankings es el ranking promedio. En este caso, a los enteros que representan el orden en el ranking en los múltiples rankings, se hace la media con éstos para crear el ranking agregado. Una desventaja de este método es la frecuente aparición de empates en el rating. Este método sólo puede usarse cuando todos los rankings contienen los mismos elementos. 

\begin{ejemplo}
Consideramos los mismos rankings que en el Ejemplo \ref{ej:borda}. DEN aparece en 5ª posición en los tres rankings por lo que la media es $5$. Para LAL, que aparece en dos cuartas posiciones y un segundo lugar, la media es $3.33$. Análogo para el resto de equipos. Los resultados se pueden ver en la Tabla \ref{tbl:promedio_resultados}. Se producen los mismos empates que en el Ejemplo \ref{ej:borda}, por lo que rompemos los empates de la misma forma.

\begin{table}[h]
\centering
\caption{Resultados del método del ranking promedio}
\label{tbl:promedio_resultados}
\begin{tabular}{@{}cccccc@{}}
\cmidrule(l){2-6}
    & Massey & Colley & Markov & Ranking promedio & Ranking agregado \\ \midrule
DEN & $5$    & $5$    & $5$    & $5$              & $5$              \\
LAL & $4$    & $4$    & $2$    & $3.33$           & $4$              \\
SAS & $2$    & $2$    & $1$    & $1.67$           & $1$              \\
CHI & $3$    & $3$    & $4$    & $3.33$           & $3$              \\
NYK & $1$    & $1$    & $3$    & $1.67$           & $2$              \\ \bottomrule
\end{tabular}
\end{table}

\end{ejemplo}


\subsection{Método de datos de partidos simulados}

El método de datos de partidos simulados\footnote{Simulated Game Data}, nace de una simple interpretación de un ranking. Si el equipo $A$ aparece por encima del equipo $B$ en un ranking, entonces en un emparejamiento entre estos dos equipos, $A$ debería batir al equipo $B$. Es más, si el equipo $A$ aparece en la parte de arriba del ranking y juega contra $B$, que aparece al final del ranking, esperaríamos que que $A$ gane a $B$ con un amplio margen de victoria. Un ranking da información implícita sobre los futuros resultados. Usamos rankings para generar los datos de partidos simulados, que usamos para agregar varios rankings en uno solo. Cada ranking de $n$ equipos proporciona datos para $\binom{n}{2}$ partidos simulados. Por tanto, si tenemos $k$ rankings de $n$ equipos tendremos $k \binom{n}{2}$. partidos simulados.\\

La forma de generar los partidos simulados es, para cada ranking, asignar un punto de margen de victoria por cada posición de diferencia en el ranking. La ventaja de este método es que no es necesario tener rankings completos, es decir, que todos los rankings que contengan todos los equipos. \\

Una vez que se han creado los partidos simulados, se aplica un método de combinado, que no es más que cualquiera de los métodos de rankings que hemos visto en el Capítulo \ref{chp:conceptos_basicos}. Se puede utilizar un método diferente distinto al de los rankings de entrada o utilizar el mismo método para favorecer los rankings con ese método. El método de combinación suaviza el efecto producido por los outliers (los datos atípicos del ranking). La Figura \ref{fig:partidos_simulados} muestra el esquema general del método.

\begin{figure}[htb]
\centering
\partidossimulados
\caption{Método de datos de partidos simulados}
\label{fig:partidos_simulados}
\end{figure}

\begin{ejemplo} \label{ej:partidos_simulados}
Consideremos los métodos rankings del Ejemplo \ref{ej:borda}. Tenemos tres rankings generados por tres métodos; Massey, Colley y Markov con $5$ equipos cada uno, por lo que tenemos que generar $3\binom{5}{2} = 30$ partidos simulados. \\

Comenzamos con el ranking producido por el método de Massey. DEN aparece en la 5ª y última posición, por lo que hay una posición de diferencia con LAL, tres con SAS, dos con CHI y cuatro con NYK. De esta forma los partidos simulados serán de $100-101$, $100-103$, $100-102$ y $100-104$, respectivamente. Hemos considerado $100$ como la media de puntos a favor de todos los equipos, y se le suma la diferencia al equipo ganador. De la misma forma se repite para todos los equipos y todos los métodos. La Tabla \ref{tbl:partidos_simulados} muestra todos los partidos simulados. \\

\begin{table}[h]
\centering
\caption[Partidos simulados del Ejemplo \ref{ej:partidos_simulados}]{Partidos simulados del Ejemplo \ref{ej:partidos_simulados}. El primer resultado de cada celda de la tabla es la simulación del partido según el ranking de Massey, el segundo por el método de Colley, y el tercero por el método de Markov}
\label{tbl:partidos_simulados}
\begin{tabular}{@{}cccccc@{}}
\cmidrule(l){2-6}
    & DEN                                                                 & LAL                                                                 & SAS                                                                 & CHI                                                                 & NYK                                                                 \\ \midrule
DEN & ---                                                                 & \begin{tabular}[c]{@{}c@{}}100-101\\ 100-101\\ 100-103\end{tabular} & \begin{tabular}[c]{@{}c@{}}100-103\\ 100-103\\ 100-104\end{tabular} & \begin{tabular}[c]{@{}c@{}}100-102\\ 100-102\\ 100-101\end{tabular} & \begin{tabular}[c]{@{}c@{}}100-104\\ 100-104\\ 100-102\end{tabular} \\ \midrule
LAL & \begin{tabular}[c]{@{}c@{}}101-100\\ 101-100\\ 103-100\end{tabular} & ---                                                                 & \begin{tabular}[c]{@{}c@{}}100-102\\ 100-102\\ 100-101\end{tabular} & \begin{tabular}[c]{@{}c@{}}100-101\\ 100-101\\ 102-100\end{tabular} & \begin{tabular}[c]{@{}c@{}}100-103\\ 100-103\\ 101-100\end{tabular} \\ \midrule
SAS & \begin{tabular}[c]{@{}c@{}}103-100\\ 103-100\\ 104-100\end{tabular} & \begin{tabular}[c]{@{}c@{}}102-100\\ 102-100\\ 101-100\end{tabular} & ---                                                                 & \begin{tabular}[c]{@{}c@{}}101-100\\ 101-100\\ 103-100\end{tabular} & \begin{tabular}[c]{@{}c@{}}100-101\\ 100-101\\ 102-100\end{tabular} \\ \midrule
CHI & \begin{tabular}[c]{@{}c@{}}102-100\\ 102-100\\ 101-100\end{tabular} & \begin{tabular}[c]{@{}c@{}}101-100\\ 101-100\\ 100-102\end{tabular} & \begin{tabular}[c]{@{}c@{}}100-101\\ 100-101\\ 100-103\end{tabular} & ---                                                                 & \begin{tabular}[c]{@{}c@{}}100-102\\ 100-102\\ 100-101\end{tabular} \\ \midrule
NYK & \begin{tabular}[c]{@{}c@{}}104-100\\ 104-100\\ 102-100\end{tabular} & \begin{tabular}[c]{@{}c@{}}103-100\\ 103-100\\ 100-101\end{tabular} & \begin{tabular}[c]{@{}c@{}}101-100\\ 101-100\\ 100-102\end{tabular} & \begin{tabular}[c]{@{}c@{}}102-100\\ 102-100\\ 101-100\end{tabular} & ---                                                                 \\ \bottomrule
\end{tabular}
\end{table}

Tras generar los partidos simulados, debemos aplicar un método de combinación para crear el ranking agregado. Aplicaremos como método de combinación el método de Massey y el de Colley para comparar ambos resultados.\\

El sistema de Massey $\mathbf{M r} = \mathbf{p}$ queda de la siguiente manera:

\begin{equation*}
\left(\begin{array}{rrrrr}
12 & -1 & -1 & -1 & -1 \\
-1 & 12 & -1 & -1 & -1 \\
-1 & -1 & 12 & -1 & -1 \\
-1 & -1 & -1 & 12 & -1 \\
 1 &  1 &  1 &  1 &  1
\end{array}\right)
\left(\begin{array}{c}
\mathbf{r}_\text{DEN}\\
\mathbf{r}_\text{LAL}\\
\mathbf{r}_\text{SAS}\\
\mathbf{r}_\text{CHI}\\
\mathbf{r}_\text{NYK}
\end{array}\right)
= 
\left(\begin{array}{c}
-30\\
-5\\
20\\
-5\\
0
\end{array}\right)
\end{equation*}

El sistema de Colley $\mathbf{C r} = \mathbf{b}$ queda de la siguiente manera:

\begin{equation*}
\left(\begin{array}{rrrrr}
14 & -3 & -3 & -3 & -3 \\
-3 & 14 & -3 & -3 & -3 \\
-3 & -3 & 14 & -3 & -3 \\
-3 & -3 & -3 & 14 & -3 \\
-3 & -3 & -3 & -3 & 14
\end{array}\right)
\left(\begin{array}{c}
\mathbf{r}_\text{DEN}\\
\mathbf{r}_\text{LAL}\\
\mathbf{r}_\text{SAS}\\
\mathbf{r}_\text{CHI}\\
\mathbf{r}_\text{NYK}
\end{array}\right)
= 
\left(\begin{array}{c}
-5\\
-0\\
5\\
4\\
5
\end{array}\right)
\end{equation*}

La Tabla \ref{tbl:partidos_simulados_resultados} muestra los resultados del cáculo de rating de Massey y de Colley, y su ranking. Notar que se producen empates al calcular los ratings. Los resolvemos de igual manera que en el Ejemplo \ref{ej:borda}.

\begin{table}[h]
\centering
\caption{Resultados del Ejemplo \ref{ej:partidos_simulados}}
\label{tbl:partidos_simulados_resultados}
\begin{tabular}{@{}ccccc@{}}
\cmidrule(l){2-5}
    & \begin{tabular}[c]{@{}c@{}}$\mathbf{r}$\\ de Massey\end{tabular} & \begin{tabular}[c]{@{}c@{}}Ranking\\ agregado\end{tabular} & \begin{tabular}[c]{@{}c@{}}$\mathbf{r}$\\ de Colley\end{tabular} & \begin{tabular}[c]{@{}c@{}}Ranking\\ agregado\end{tabular} \\ \midrule
DEN & -2.3077                                                          & 5                                                          & 0.5000                                                           & 5                                                          \\
LAL & -0.3846                                                          & 4                                                          & 0.7941                                                           & 4                                                          \\
SAS & 1.5385                                                           & 1                                                          & 1.0882                                                           & 1                                                          \\
CHI & -0.3846                                                          & 3                                                          & 1.0294                                                           & 3                                                          \\
NYK & 1.5385                                                           & 2                                                          & 1.0884                                                           & 2                                                          \\ \bottomrule
\end{tabular}
\end{table}
\end{ejemplo}

\subsection{Método óptimo de agregación de rankings}

\todo[backgroundcolor=red, inline]{Añadir información}