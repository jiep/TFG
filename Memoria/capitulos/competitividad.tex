\chapter{Competitividad en rankings}

Cuando se dispone de una gran cantidad de rankings, es interasante ver cómo se van intercambiando las posiciones los equipos. Cuando se produzca ésto, diremos que los equipos compiten. Ésto dará lugar a dos tipos de grafos: el grafo de competitividad y el grafo de competitividad evolutivo. Veremos las relaciones de este grafo con otros tipos de grafos conocidos, como los grafos cordales, de comparabilidad, de permutaciones y f-grafos. También veremos una generalización de la tau de Kendall para una familia de rankings y varias medidas de competitividad a partir del grafo de competitividad.\\

Empezamos definiendo la competitividad entre una familia de rankings.

\begin{defi}
Dada una familia $\mathcal{R} = \{c_1, c_2, \dots, c_r\}$ de rankings, decimos que los nodos $(i,j) \in \mathcal{N} \times \mathcal{N}$ compiten si existe $t \in \{1,2,\dots, r-1\}$ tal que $i$ y $j$ intercambian sus posiciones relativas entre rankings consecutivos $c_t$ y $c_{t+1}$.
\end{defi}

\begin{ejemplo}
Si consideramos la familia de rankings $\mathcal{R} = \{c_1, c_2\}$ donde $c_1$ y $c_2$ son, respectivamente

\begin{equation*}
c_1 = \left( \begin{array}{c}
1\\
3\\
2\\
4\\
5
\end{array} \right), \quad
c_2 = \left( \begin{array}{c}
1\\
2\\
4\\
3\\
5
\end{array} \right)
\end{equation*}

Los pares $(1,2)$ y $(1,3)$ no compiten, puesto que los nodos no intercambian sus posiciones relativas entre rankings consecutivos. Sin embargo, los pares $(3,4)$ y $(2,3)$ sí compiten puesto que intercambian sus posiciones relativas entre rankings consecutivos, $c_1$ y $c_2$.
\end{ejemplo}

\section{Grafo de competitividad}
Los pares de nodos que compiten pueden dar lugar a un grafo, que llamaremos grafo de competitividad. Este grafo tiene propiedades interesantes como veremos a continuación.

\begin{defi}[Grafo de competitividad]
Definimos grafo de competitividad de la familia de rankings $\mathcal{R}$, y lo denotaremos como $G_c(\mathcal{R}) = (\mathcal{N}, E_\mathcal{R})$, donde $\mathcal{N}$ es el conjunto de nodos $E_\mathcal{R}$ denota el conjunto de arcos dados por la siguiente regla: existe un arco entre $i$ y $j$ si el par $(i,j)$ compite.
\end{defi}

\begin{ejemplo} \label{ej:grafo_competitividad}
Consideremos cuatro clasificaciones ficticias correspondientes a la División Noroeste de la NBA, compuesta por Porland Trail Blazers (POR), Ockahoma City Thunder (OCK), Denver Nuggets (DEN), Utah Jazz (UTA) y Minnesota Timberwolves~(MIN).\\

Las clasificaciones $\mathcal{R} = \{c_1, c_2, c_3, c_4\}$ son las siguientes:

\begin{equation*}
c_1 = \left( \begin{array}{c}
\text{DEN}\\
\text{POR}\\
\text{OCK}\\
\text{UTA}\\
\text{MIN}
\end{array} \right), \quad
c_2 = \left( \begin{array}{c}
\text{POR}\\
\text{OCK}\\
\text{UTA}\\
\text{DEN}\\
\text{MIN}
\end{array} \right), \quad
c_3 = \left( \begin{array}{c}
\text{POR}\\
\text{OCK}\\
\text{DEN}\\
\text{UTA}\\
\text{MIN}
\end{array} \right), \quad
c_4 = \left( \begin{array}{c}
\text{OCK}\\
\text{UTA}\\
\text{POR}\\
\text{DEN}\\
\text{MIN}
\end{array} \right)
\end{equation*}

El grafo de competitividad $G_c(\mathcal{R})$ de la familia de rankings $\mathcal{R}$ se muestra en la Figura~\ref{fig:grafo_competitividad}. El grafo tiene dos componentes conexas: $\{\text{POR}, \text{OCK}, \text{DEN}, \text{UTA}\}$ y $\{\text{MIN}\}$. Vemos que hay pares de equipos que compiten más de una vez, por ejemplo, el par $(\text{UTA}, \text{DEN})$ cambian sus posiciones relativas tres veces en los rankings $c_1$, $c_2$, $c_3$ y $c_4$. Esta información no la recoge el grafo de competitividad. Esta información la recogerá el grafo de competitividad evolutivo.

\begin{figure}[htb]
\centering
\ejemplografocompetitividad
\caption[Grafo de competitividad]{Grafo de competitividad de $\mathcal{R}$}
\label{fig:grafo_competitividad}
\end{figure} 

\end{ejemplo}

\begin{defi}
Decimos que los nodos $i$, $j$ compiten $k$ veces si $k$ es el número de veces máximo de rankings donde $i$ y $j$ compiten.
\end{defi}

\begin{defi}
El grafo de competitividad evolutivo de $\mathcal{R}$, denotado por $G_c^e(\mathcal{R}) = (\mathcal{N}, E_\mathcal{R}^e)$ es el grafo ponderado con el conjunto $\mathcal{N}$ como nodos y aristas dadas por la siguiente regla: hay una arista entre $i$ y $j$ con peso $k$ si $(i,j)$ compiten $k$ veces.
\end{defi}

\begin{ejemplo}
Consideramos de nuevo la familia $\mathcal{R}$ de rankings del Ejemplo \ref{ej:grafo_competitividad}. El grafo de competitividad de $\mathcal{R}$ se muestra en la Figura~\ref{fig:grafo_competitividad_evolutivo}.

\begin{figure}[htb]
\centering
\ejemplografocompetitividadevolutivo
\caption[Grafo de competitividad evolutivo]{Grafo de competitividad evolutivo de $\mathcal{R}$}
\label{fig:grafo_competitividad_evolutivo}
\end{figure}

\end{ejemplo}

Notar que si eliminamos los pesos del grafo de competitividad evolutivo, obtenemos el grafo de competitividad.

\section{Medidas de competitividad}

Cuando se dispone de más de una familia de rankings, es necesario disponer una serie de medidas que permitan medir la competitividad en cada uno de los grafos de competitividad. Algunas de estas medidas son grado medio normalizado, fuerza media normalizada, coeficiente de clustering y una generalización de la tau de Kendall para una familia de rankings.

\subsection*{Grado medio normalizado}

\begin{defi}
Se define grado medio normalizado de una familia de rankings $\mathcal{R}$, como la suma de todos los grados de los nodos en el grafo de competitividad $G_c(\mathcal{R})$ dividido por la suma sobre todos los nodos de sus grados más altos posibles. Esto es,

\begin{equation}
\mathrm{ND}(\mathcal{R}) = \dfrac{1}{n(n-1)} \sum_{i \in \mathcal{N}} d_i
\end{equation}

donde $d_i$ es el número de nodos adyacentes al nodo  $i$.
\end{defi}

\begin{defi}
Se dice que la familia de rankings $\mathcal{R}$ es más competitiva que la familia de rankings $\mathcal{S}$ con respecto al grado medio normalizado si $\mathrm{ND}(\mathcal{R}) > \mathrm{ND}(\mathcal{S})$.
\end{defi}

\begin{ejemplo}
Consideremos la familia de rankings $\mathcal{R}$ y el grafo de competitividad del Ejemplo~ \ref{ej:grafo_competitividad} y el grafo de competitividad $G_c(\mathcal{S})$ de la Figura~\ref{fig:grado_medio}.\\


\begin{figure}[htb]
\centering
\ejemplogradomedio
\caption{Grafo de competitividad de $\mathcal{S}$}
\label{fig:grado_medio}
\end{figure}

El grado medio de las dos familias son:

\begin{eqnarray*}
\mathrm{ND}(\mathcal{R}) = \dfrac{1}{5\cdot 4} (3 + 1 + 1 + 1 + 0) = \dfrac{6}{20} = \dfrac{3}{10}\\
\mathrm{ND}(\mathcal{S}) = \dfrac{1}{5\cdot 4} (2 + 4 + 2 + 3 + 3) = \dfrac{14}{20} = \dfrac{7}{10}
\end{eqnarray*}

Puesto que $\mathrm{ND}(\mathcal{S}) > \mathrm{ND}(\mathcal{R})$, la familia $\mathcal{S}$ es más competitiva que la familia $\mathcal{R}$.
\end{ejemplo}

\subsection*{Fuerza media normalizada}

\begin{defi}
Se define la fuerza media normalizada de una familia de rankings $\mathcal{R}$ como la suma de todos pesos de las aristas del grafo de competitividad evolutivo $G_c^e(\mathcal{R})$ dividido entre la suma de todas las posibles aristas con sus posibles pesos. Esto es,

\begin{equation}
\mathrm{NS}(\mathcal{R}) = \dfrac{w(E_\mathcal{R}^e)}{\binom{n}{2} (r-1)}
\end{equation} 

donde $w(E_\mathcal{R}^e)$ denota la suma de todos los pesos de las aristas del grafo de competitividad evolutivo $G_c^e(\mathcal{R})$ y $r$ es el número de rankings de la familia $\mathcal{R}$.
\end{defi}

\begin{defi}
Se dice que $\mathcal{R}$ es más competitivo que $\mathcal{S}$ con respecto a la fuerza media normalizada si $\mathrm{NS}(\mathcal{R}) > \mathrm{NS}(\mathcal{S})$.
\end{defi}

\begin{ejemplo}
Consideremos la familia $\mathcal{R}$ del Ejemplo~\ref{ej:grafo_competitividad} y el grafo de competitividad evolutivo de la familia $\mathcal{S}$ compuesto por $5$ rankings de la Figura~\ref{fig:fuerza_media}.\\


\begin{figure}[htb]
\centering
\ejemplofuerzamedia
\caption{Grafo de competitividad evolutivo de $\mathcal{S}$}
\label{fig:fuerza_media}
\end{figure}

Calculamos la fuerza media de cada una de las familias de rankings,

\begin{eqnarray*}
\mathrm{NS}(\mathcal{R}) = \dfrac{7}{10 \cdot 3} = \dfrac{7}{30}\\
\mathrm{NS}(\mathcal{S}) = \dfrac{11}{10\cdot 4} = \dfrac{11}{40}
\end{eqnarray*}

Como $\mathrm{NS}(\mathcal{S}) > \mathrm{NS}(\mathcal{R})$, la familia de rankings $\mathcal{S}$ es más competitiva con respecto a la fuerza media normalizada que la familia $\mathcal{R}$.

\end{ejemplo}

\subsection*{Coeficiente de clustering}

\begin{defi}
El coeficiente de clustering de un nodo $i$ se define como

\begin{equation}
C_i = \dfrac{e_i}{\binom{d_i}{2}}
\end{equation}

donde $d_i$ es el número de vértices adyacentes al nodo $i$ y $e_i$ es el número de pares conectados entre los adyacentes a $i$.
\end{defi}

\begin{defi}
Dada una familia de rankings $\mathcal{R}$, se define el coeficiente de clustering de $\mathcal{R}$ como la media aritmética de los coeficientes de clustering de los nodos del grafo de competitividad $G_c(\mathcal{R})$. Esto es,

\begin{equation}
C(\mathcal{R}) = \dfrac{1}{n} \sum_{i \in \mathcal{N}} C_i
\end{equation}
\end{defi}

\begin{defi}
Decimos que la familia de rankings $\mathcal{R}$ es más competitiva con respecto al coeficiente de clustering que la familia de rankings $\mathcal{S}$ si $C(\mathcal{R}) > C(\mathcal{S})$.
\end{defi}

\begin{ejemplo}
Consideramos los grafos de competitividad de las Figuras~\ref{fig:grafo_competitividad}~y~\ref{fig:grado_medio}. Los coeficientes de clustering de cada uno de los nodos de la familia $\mathcal{S}$ son los siguientes:

\begin{align*}
\mathrm{C}_\text{MIN} & =  0\\
\mathrm{C}_\text{POR} & = \dfrac{4}{6} = \dfrac{2}{3}\\
\mathrm{C}_\text{OCK} & = \dfrac{2}{2} = 1\\
\mathrm{C}_\text{DEN} & = \dfrac{4}{6} = \dfrac{2}{3}\\
\mathrm{C}_\text{UTA} & = \dfrac{2}{2} = 1
\end{align*}

El coeficiente de clustering de la familia $\mathcal{R}$ es 

\begin{equation*}
C(\mathcal{R}) = \dfrac{1}{5}\left( 0 + \dfrac{2}{3} + 1 + \dfrac{2}{3} + 1 \right) = \dfrac{2}{3}
\end{equation*}

De la misma forma los coeficientes de clustering de cada uno de los nodos de la familia $\mathcal{S}$ y de la familia $\mathcal{S}$ son los siguientes:

\begin{eqnarray*}
\mathrm{C}_\text{1} = \dfrac{2}{2} = 1\\
\mathrm{C}_\text{2} = \dfrac{2}{4} = \dfrac{1}{2}\\
\mathrm{C}_\text{3} = \dfrac{2}{2} = 1\\
\mathrm{C}_\text{4} = \dfrac{4}{6} = \dfrac{2}{3}\\
\mathrm{C}_\text{5} = \dfrac{4}{6} = \dfrac{2}{3}
\end{eqnarray*}

\begin{equation*}
C(\mathcal{S}) = \dfrac{1}{5}\left( 1 + \dfrac{1}{2} + 1 + \dfrac{2}{3} + \dfrac{2}{3} \right) = \dfrac{23}{30}
\end{equation*}

Como $C(\mathcal{S}) > C(\mathcal{R})$, la familia $\mathcal{S}$ de rankings es más competitiva que la familia $\mathcal{R}$.
 
\end{ejemplo}

\subsection*{Tau de Kendall generalizada}

Se puede modificar la tau de Kendall de la Definición~\ref{def:tau_kendall} para adaptarla como medida de competitividad de la siguiente manera.

\begin{defi}
Llamamos tau de Kendall entre dos rankings $c_1$ y $c_2$ con $n$ elementos cada uno al siguiente valor

\begin{equation}
\tau(c_1, c_2) = \dfrac{\tilde{K}(c_1, c_2) - K(c_1, c_2)}{\binom{n}{2}}
\end{equation} 

donde $\tilde{K}(c_1, c_2)$ es el número de pares $(i,j)$ que no compiten con respecto a $\mathcal{R} = \{c_1, c_2\}$, y $K(c_1, c_2)$ denota el número de pares que compiten en $\mathcal{R}$.
\end{defi}

Esta medida de dos rankings se puede generalizar a una familia con $r$ rankings de la siguiente forma.

\begin{defi}
Se define tau de Kendall generalizada a la familia de rankings $\mathcal{R}$, con $r \geq 2$ de la siguiente manera:

\begin{equation}
\tau(\mathcal{R}) = \dfrac{\tilde{K}(c_1, c_2) - K(c_1, c_2)}{\binom{n}{2}} = 1 - \dfrac{4 |E_\mathcal{R}|}{n(n-1)}
\end{equation}

donde $|E_\mathcal{R}|$ es el número de aristas del grafo de competitividad $G_c(\mathcal{R})$.
\end{defi}

\begin{defi}
Se defiene tau de Kendall evolutiva de la familia de rankings $\mathcal{R}$ con $r \geq 2$ rankings de la siguiente manera:

\begin{equation}
\tau_e(\mathcal{R}) = 1 - \dfrac{2 w(E_\mathcal{R}^e)}{\binom{n}{2}(r-1)}
\end{equation} 

donde $w(E_\mathcal{R}^e)$ denota la suma de todos los pesos de las aristas del grafo de competitividad evolutivo.
\end{defi}

\begin{defi}
Decimos que la familia de rankings $\mathcal{R}$ es más competitiva con respecto a la tau de Kendall evolutiva que la familia de rankings $\mathcal{S}$ si $\tau_e(\mathcal{R}) < \tau_e(\mathcal{S})$.
\end{defi}

\begin{ejemplo}
Consideremos los grafos evolutivos de las Figuras~\ref{fig:grafo_competitividad_evolutivo}~y~\ref{fig:fuerza_media}. Calculamos la tau de Kendall para la familia de rankings $\mathcal{R}$.

\begin{eqnarray}
\tau(\mathcal{R}) = 1 - \dfrac{4 \cdot 5}{5 \cdot 4} = 0\\
\tau_e(\mathcal{R}) = 1 - \dfrac{4 \cdot 7}{5 \cdot 4} = -\dfrac{2}{5}
\end{eqnarray}

De la misma forma para la familia $\mathcal{S}$,

\begin{eqnarray}
\tau(\mathcal{S}) = 1 - \dfrac{2 \cdot 7}{10 \cdot 3} = \dfrac{8}{15}\\
\tau_e(\mathcal{S}) = 1 - \dfrac{11 \cdot 2}{10 \cdot 4} = -\dfrac{9}{20}
\end{eqnarray}

Como $\tau_e(\mathcal{S}) < \tau_e(\mathcal{R})$ se tiene que la familia $\mathcal{S}$ es más competitiva que la familia $\mathcal{S}$.

\end{ejemplo}

Tanto la fuerza media como la tau de Kendall evolutiva están relacionadas con la siguiente forma.

\begin{prop}
La tau de Kendall y la fuerza media normalizada están relacionadas por la siguiente fórmula:

\begin{equation}
\tau_e(\mathcal{R}) = 1 - 2 \mathrm{NS}(\mathcal{R})
\end{equation}
\end{prop}

\begin{ejemplo}
Volvemos a considerar los grafos evolutivos de las Figuras~\ref{fig:grafo_competitividad_evolutivo}~y~\ref{fig:fuerza_media}. Sabiendo $\tau_e(\mathcal{R})$ y $\tau_e(\mathcal{S})$, podemos calcular $\mathrm{NS}(\mathcal{R})$ y $\mathrm{NS}(\mathcal{S})$ con la fórmula anterior. Si sustituimos en la fórmula, obtenemos los resultados calculados anteriormente de la fuerza media normalizada.
\end{ejemplo}

\section{Relaciones con otros grafos}

Una vez vistas una serie de medidas para medir la competitividad entre familias de rankings, es interesante ver si este tipo de grafo está relacionado con algún otro conocido, como los grafos de comparabilidad, grafos de permutaciones, grafos cordales y los f-grafos y ver las inclusiones que se dan entre cada tipo de grafo. Para empezar, definiremos estos tipos de grafos.

\begin{defi}
Un grafo $G = (\mathcal{N}, E)$ es cordal si cada sus ciclos de cuatro o más vértices tiene una cuerda, que es un arco que une dos nodos que no son adyacentes en el ciclo.
\end{defi} 

\begin{ejemplo}
La Figura \ref{fig:grafo_cordal} muestra un grafo cordal. Recibe el nombre de 4-1 lollipop.

\begin{figure}[htb]
\centering
\ejemplografocordal
\caption{Grafo cordal}
\label{fig:grafo_cordal}
\end{figure}
\end{ejemplo}

\begin{defi}
Un grafo $G = (\mathcal{N}, E)$ es un grafo de permutación si sus vértices representan los elementos de una permutación y cada uno de sus arcos corresponden a un par de elementos que están invertidos por la permutación.
\end{defi}

\begin{ejemplo}
La Figura \ref{fig:grafo_permutación} muestra un grafo de permutación. Este grafo viene de la permutación $(4,3,5,1,2)$.  

\begin{figure}[htb]
\centering
\ejemplografopermutacion
\caption{Grafo de permutaciones}
\label{fig:grafo_permutación}
\end{figure}
\end{ejemplo}

\begin{defi}
Un grafo $G = (\mathcal{N}, E)$ es un grafo de comparabilidad si es el grafo no dirigido obtenido después de eliminar la orientación en $G_\preceq$ para algún orden parcial $\preceq$ de $\mathcal{N}$.  
\end{defi}

\begin{ejemplo}
La Figura \ref{fig:grafo_comparabilidad} muestra un grafo de comparabilidad.

\begin{figure}[htb]
\centering
\ejemplografocomparabilidad
\caption{Grafo de comparabilidad}
\label{fig:grafo_comparabilidad}
\end{figure}
\end{ejemplo}

\begin{defi}
Un f-diagrama en $\R^2$ es un conjunto de curvas $D = \{\gamma_1,\dots, \gamma_n \}$, donde $\gamma_i : [0,1] \to \R^2$ está dado por $\gamma_i(t) = (t, f_i(t))$ con $f_i : [0,1] \to \R$ funciones continuas para todo $1 \leq i \leq n$.
\end{defi}

\begin{ejemplo} \label{ej:f-diagrama}
Consideremos el conjunto $D = \{\gamma_1, \gamma_2, \gamma_3\}$ compuesto por las curvas $\gamma_i : [0,1] \to \R^2$ definidas de la siguiente manera:

\begin{eqnarray*}
\gamma_1(t) = (t,1)\\
\gamma_2(t) = (t,2t)\\
\gamma_3(t) = (t, t^2)
\end{eqnarray*}

Este conjunto $D$ es un f-diagrama ya que cumple todas y cada una de las condiciones de la definición anterior.

\end{ejemplo}

\begin{defi}
Un f-grafo $G(D) = (V,E)$ que representa el diagrama $D = \{\gamma_1,\dots, \gamma_n \}$ está dado por $E = \{1,\dots, n\}$ y $i \neq j \in V$ son adyacentes en $G(D)$ si $f_i(a) = f_j(a)$ para algún $a \in [0,1]$.
\end{defi}

\begin{ejemplo}
La Figura \ref{fig:f-grafo} muestra el f-grafo del f-diagrama $D$ del Ejemplo \ref{ej:f-diagrama}. En efecto, hay una arista entre el nodo $1$ y $2$ porque existe un punto $a \in [0,1]$ tal que $\gamma_1(a) = \gamma_2(a)$. En este caso, $a = 1/2$. Del mismo modo, hay una arista entre los nodos $1$ y $3$ porque $\gamma_1(1) = \gamma_3(1)$. Para los nodos $2$ y $3$, $\gamma_2(0) = \gamma_3(0)$. 

\begin{figure}[htb]
\centering
\ejemplogfgrafo
\caption{f-grafo}
\label{fig:f-grafo}
\end{figure}
\end{ejemplo}

\begin{prop}
Sea $G = (V,E)$ un grafo no dirigido con $n$ nodos. Las siguientes afirmaciones son equivalentes:

\begin{enumerate}
\item $G$ es un f-grafo que representa un f-diagrama de $n$ curvas en $\R^2$.
\item El grafo complementario de $G$ es un grafo de comparabilidad.
\item Existen permutaciones $c_1,c_2, \dots, c_r$ de $\mathcal{N} = \{1,\dots,n\}$ tales que

\begin{equation*}
G = G_1 \cup \dots \cup G_r
\end{equation*}

donde $\cup$ es la unión clásica de grafos y $G_i$ es el grafo de permutaciones asociado a la permutación $c_i$ para cada $1 \leq i \leq r$

\item $G$ es un grafo de competitividad.

\end{enumerate}
\end{prop}

\begin{proof} \label{thm:relaciones_entre_grafos}
Ver Proposición 3.13 de \cite{ongraphsassociatedtosetsofrankings}.
\end{proof}

\begin{teo}
Si $\mathcal{G}$ es el conjunto de todos los grafos finitos, $CH\mathcal{G}$, $P\mathcal{G}$, $C\mathcal{G}$ y $CP\mathcal{G}$ son los conjuntos de los grafos cordales, de permutaciones, de comparabilidad y de competitividad respectivamente, entonces
\begin{enumerate}
\item Hay grafos de comparabilidad que no son ni grafos de competitividad ni grafos cordales, es decir, $C\mathcal{G} \setminus (CP\mathcal{G} \cup CH\mathcal{G}) \neq \emptyset$.

\item Hay grafos de competitividad que no son grafos de comparabilidad ni grafos cordales, es decir, $CP\mathcal{G} \setminus (C\mathcal{G}\cup CH\mathcal{G}) \neq \emptyset$.

\item Hay grafos que no son ni de comparabilidad, ni de competitividad ni cordales, es decir, $\mathcal{G} \setminus (C\mathcal{G} \cup CP\mathcal{G} \cup C\mathcal{G}) \neq \emptyset$.

\item Los grafos de permutaciones coinciden con la intersección de los grafos de competitividad y de comparabilidad, pero no cada grafo de permutación es cordal, es decir, $P\mathcal{G} = C\mathcal{G}\cap CP\mathcal{G}$ y $P\mathcal{G} \not \subseteq CH\mathcal{G}$.

\item Hay grafos cordales que no son grafos de comparabilidad ni de competitividad, es decir, $CH\mathcal{G} \setminus (C\mathcal{G} \cup CP\mathcal{G}) \neq \emptyset$.

\item Hay grafos que son de competitividad y cordales, pero no son de comparabilidad, es decir, $CP\mathcal{G} \cap CH\mathcal{G} \cap (\mathcal{G} \setminus C\mathcal{G}) \neq \emptyset$.

\item Hay grafos que son de competitividad y cordales, pero no de competitividad, es decir, $C\mathcal{G} \cap CH\mathcal{G} \cap (\mathcal{G} \ C\mathcal{G}) \neq \emptyset$.

\item Hay grafos que son cordales, de competitividad y de comparabilidad, es decir, $CH\mathcal{G} \cap CP\mathcal{G} C\mathcal{G} \neq \emptyset$.
\end{enumerate}
\end{teo}

\begin{proof}
Ver Teorema 3.16 de \cite{journals/corr/CriadoGPR13a}.
\end{proof}

%El resumen del Teorema~\ref{thm:relaciones_entre_grafos} se puede ver en la Figura~\ref

\section{Cálculo de las componentes conexas}

En el Ejemplo \ref{ej:grafo_competitividad} vimos que el grafo de competitividad puede tener más de una componente conexa. Estas componentes conexas indican los equipos que han competido entre sí. Veremos un algoritmo que permite calcular las componentes conexas del grafo de competitividad sin calcularlo explíctitamente a partir de los llamamos competidores eventuales.

\begin{defi}
Si tenemos una familia de rankings $\mathcal{R} = \{c_1,\dots,c_r\}$ de nodos $\mathcal{N} = \{1,\dots,n\}$, decimos que un par de nodos $(i,j) \in \mathcal{N} \times \mathcal{N}$ compiten eventualmente si existe $k \in \N$ y nodos $i_i,\dots,i_k \in \mathcal{N}$ tal que $(i,i_1)$ compiten, $(i_1,i_2)$ compiten,\dots, y $(i_k,j)$ compiten. 
\end{defi}

\begin{defi}
Un conjunto $D \subseteq \mathcal{N}$ se llama conjunto de competidores eventuales si es un conjunto maximal con respecto a la propiedad de competición eventual entre sus elementos.
\end{defi}

\begin{teo}
Dado un conjunto finito $\mathcal{R} = \{ c_1, \dots, c_r \}$ de rankings de nodos $\mathcal{N} = \{1,\dots, n\}$ el conjunto de competidores eventuales puede ser identificado con los intervalos cerrados de números naturales $[p,q]$ en el siguiente sentido:

\begin{equation*}
D_{[p,q]} = \{ x \in \mathcal{N} \mid c_s^{-1} \in [p,q] \text{ para algún } c_s \in \mathcal{R} \}
\end{equation*}

Además, $p$ y $q$ son la primera y la última posición de los elementos de $D_{[p,q]}$ con respecto a todos los rankings.
\end{teo} 

\begin{proof}
Ver el Teorema 4.3 de \cite{journals/corr/CriadoGPR13a}.
\end{proof}

La prueba de este teorema da un algoritmo para calcular los conjuntos de competidores eventuales del grafo de competitividad, que coinciden con las componentes conexas de dicho grafo.

\begin{algorithm}\label{algoritmo}
\SetKwInOut{Input}{Entrada}\SetKwInOut{Output}{Salida}
 \Input{}
 \begin{itemize}
 	\item Un conjunto finito de nodos $\mathcal{N} = \{1,\dots, n\} (n \in \N)$
 	\item Un conjunto finito de rankings $\mathcal{R} = \{c_1,\dots,c_r\}$ de $\mathcal{N}$ $(r \in \N)$
 \end{itemize}
 
 \Begin{
 	$j := 1$\;
 	$q_0 := 0$\;
 	$q_j := 1$\;
 	\While{$|\mathcal{N}| > 0$}{
 		$D_j := \emptyset$\;
 		$p_0 := q_{j-1}$\;
 		$p_1 := q_j$\;
		$i := 0$\;
 		\While{$p_i \neq p_{i+1}$}{
 			$i := i + 1$\;
 			Construir $D_j := D_{[q_j,p_i]}$\;
 			$p_{i+1} := \displaystyle \max_{x \in D, c \in \mathcal{R}} c^{-1}(x)$\;
 		}
 		$\mathcal{N} := \mathcal{N} \setminus D_j$\;
 		$j := j + 1$\;
 		$q_j := p_i$\;
 	}
 
 }
 
 \Output{ \ Conjuntos de competidores eventuales $D_1, \dots, D_k$}
 
 \end{algorithm}

\todo[backgroundcolor=red, inline]{Añadir ejemplo}