\chapter{Cadenas de Markov}

\begin{defi}
Un vector fila $\mathbf{p} = (p_1,p_2\dots,p_n)$ es un vector de probabilidad si todas sus componentes son no negativas y su suma es $1$, es decir,

\[ \sum\limits_{i=1}^{n} p_i = 1 \]
\end{defi}

\begin{ejemplo}
El vector $\mathbf{p} = (1/2, 1/4, 1/5, 19/30)$ es un vector de probabilidad porque $\frac{1}{2} + \frac{1}{4} + \frac{1}{5} + \frac{19}{30} = 1$, y todas sus componentes son mayores o iguales que $1$. \\
Sin embargo, el vector $\mathbf{q} = (1, 1/2, -2, 1/2)$ no es un vector de probabilidad porque tiene una componente negativa, $-2$.
\end{ejemplo}

\begin{defi}
Una matriz cuadrada $\mathbf{A} = (a_{ij})$ de tamaño $n \times n$ se dice estocástica si cada una de sus filas es un vector de probabilidad, es decir, si cada cada entrada de $\mathbf{A}$ es no negativa y la suma de las entradas de cada fila es $1$.
\end{defi}

\begin{ejemplo}
La siguiente matriz es una matriz estocástica 

\[\mathbf{A} = \left(\begin{array}{ccc}
1/2 & 1/2 & 0\\
0   & 2/3 & 1/3\\
1   &  0  & 0
\end{array}\right) \]

porque todas sus filas son vectores de probabilidad.
\end{ejemplo}

\begin{defi}
Una matriz estocástica $\mathbf{A}$ se dice regular si todas las entradas de $\mathbf{A}^n$ son positivas, para algún $n \in \N$.
\end{defi}

\begin{ejemplo}
La matriz estocástica

\[ \mathbf{A} = \left(\begin{array}{cc}
0 & 1\\
1/2 & 1/2
\end{array}\right) \]

es regular porque 

\[ \mathbf{A}^2 = \left(\begin{array}{cc}
1/2 & 1/2\\
1/4 & 3/4
\end{array}\right) \]

tiene cada entrada positiva.\\
Sin embargo, la matriz estocástica

\[ \mathbf{B} = \left(\begin{array}{cc}
1 & 0\\
1/2 & 1/2
\end{array}\right) \]

no es regular porque

\[ \mathbf{B}^2 = \left(\begin{array}{cc}
1 & 0\\
3/4 & 1/4
\end{array}\right) \]

tiene una entrada que no es positiva.
\end{ejemplo}

\begin{defi}
Un punto fijo $\mathbf{p}^*$ de una matriz estocástica $\mathbf{A}$ se define como la solución de la ecuación

\[ \mathbf{p}^* \mathbf{A} = \mathbf{p}^* \]
\end{defi}

\begin{ejemplo}
Si consideramos las matriz estocástica regular

\[ \mathbf{A} = \left(\begin{array}{ccc}
0 & 1 & 0\\
0 & 0 & 1\\
1/2 & 1/2 & 0
\end{array}\right) \]

El punto fijo $\mathbf{p}^* = (x,y,z)$ tiene que cumplir que $x+y+z = 1$, y $x,y,z \in [0,1]$. Resolviendo el sistema $\mathbf{p}^* \mathbf{A} = \mathbf{p}^*$, resulta que el punto fijo $p^* = (1/5, 2/5, 2/5)$.
\end{ejemplo}

\begin{teo}
Sea $\mathbf{A}$ una matriz estocástica regular. Entonces

\begin{enumerate}
\item $\mathbf{A}$ tiene un único vector de probabilidad $\mathbf{p}$, y todas las componentes de $\mathbf{p}$ son positivas.
\item La secuencias de matrices de potencias de $\mathbf{A}$, $\mathbf{A},\mathbf{A}^2,\mathbf{A}^3,\dots$ aproxima la matriz $\mathbf{B}$ cuyas filas están compuestas por el punto fijo $\mathbf{p}^*$.
\item Si $\mathbf{p}$ es un vector de probabilidad, entonces la secuencia $\mathbf{p A}, \mathbf{p} \mathbf{A}^2,\mathbf{p} \mathbf{A}^3,\dots$ aproxima el punto fijo $\mathbf{p}^*$.
\end{enumerate}
\end{teo}

\todo[backgroundcolor=red, inline]{Añadir cadenas de Markov}