\chapter{Álgebra lineal}

\section{Sistemas lineales}

\begin{defi}
Llamamos sistema $m$ de ecuaciones lineales con $n$ incógnitas al siguiente conjunto de ecuaciones:

\begin{align*}
a_{11}x_1 + a_{12}x_2 + \dots + a_{1n}x_n & = b_1\\
a_{21}x_1 + a_{22}x_2 + \dots + a_{2n}x_n & = b_2\\
 &\vdots\\
a_{m1}x_1 + a_{m2}x_2 + \dots + a_{mn}x_n & = b_m\\
\end{align*} 

donde $x_i$ son las incógnitas y $a_{ij}$, $b_i$ son constantes conocidas.\\

Las constantes $a_{ij}$ reciben el nombre de coeficientes del sistema y las constantes $b_i$ reciben el nombre de lado derecho del sistema.
\end{defi}

Para cualquier sistema existen tres posibilidades respecto al conjunto de soluciones:

\begin{itemize}
\item Solución única: existe uno  sólo un conjunto de valores para $x_i$ que satisface todas las ecuaiones simultáneamente.
\item Sin solución: no existe un conjunto de valores para $x_i$ que satisfaga todas las ecuaciones simultaneámente.
\item Infinitas soluciones: hay infinitos conjuntos de valores que satisfagan todas las soluciones simultáneamente.
\end{itemize}

\begin{defi}
Dados dos sistemas de ecuaciones lineales, se dicen que son equivalentes si tienen el mismo conjunto de soluciones.
\end{defi}

\begin{defi}
Un sistema lineal es consistente si al menos tiene una solución.
\end{defi}

\begin{defi}
Un sistema lineal es homogéneo si el lado derecho del sistema está compuesto exclusivamente por ceros. En caso contrario, se dice que el sistema es no homogéneo.
\end{defi}

\section{Álgebra matricial}

\begin{defi}
Una matriz es una colección de escalares ordenados en filas y columnas. Si la matriz $\mathbf{A}$ tiene $m$ filas y $n$ columnas, entonces

\[ \mathbf{A} = \left(\begin{array}{cccc}
a_{11} & a_{21} & \dots & a_{1n}\\
\vdots &        &       &       \\
a_{m1} & a_{m2} & \dots & a_{mn} 
\end{array}\right) \]

Si $n = m$, se dice que $\mathbf{A}$ es una matriz cuadrada.\\

Las matrices con una sola columna o con una sola fila se denominan vector columna y vector fila, respectivamente.\\

El conjunto de todas las matrices de tamaño $m \times n$ con coeficientes reales de denotan con $\R^{m\times n}$.
\end{defi}

\subsection*{Operaciones con matrices}

\begin{defi}
Si $\mathbf{A}, \mathbf{B} \in \R^{m \times n}$, la suma de $\mathbf{A}$ y $\mathbf{B}$ se define como la matriz $\mathbf{A} + \mathbf{B}$ que se obtiene de sumar sus respectivas entradas. Esto es, $(a+b)_{ij} = a_{ij} + b_{ij}$.
\end{defi}

\begin{prop}
Sean $\mathbf{A}, \mathbf{B}, \mathbf{C} \in \R^{m\times n}$. Las siguientes propiedades se cumplen:

\begin{itemize}
\item $\mathbf{A} + \mathbf{B} \in \R^{m \times n}$.
\item $(\mathbf{A} + \mathbf{B}) + \mathbf{C} = \mathbf{A} + (\mathbf{B} + \mathbf{C})$.
\item $\mathbf{A} + \mathbf{B} = \mathbf{B} + \mathbf{A}$.
\item La matriz $\mathbf{0}$ cuyas entradas son todos ceros cumple que $\mathbf{A} + \mathbf{0} = \mathbf{A}$.
\item La matriz $-\mathbf{A}$ cuyas entradas son las de $\mathbf{A}$ cambiadas de signo cumple que $\mathbf{A} + (-\mathbf{A}) = \mathbf{0}$.
\end{itemize}
\end{prop}

\begin{defi}
El producto de un escalar $\alpha$ por una matriz $\mathbf{A}$, denotado como $\alpha \mathbf{A}$, se define como la matriz obtenida de multiplicar cada entrada de $\mathbf{A}$ por $\alpha$. Esto es $(\alpha a_{ij}) = \alpha a_{ij}$.
\end{defi}

\begin{prop}
Sean $\mathbf{A}, \mathbf{B} \in \R^{m \times n}$ y $\alpha, \beta \in \R$. Las siguientes propiedades se cumplen:

\begin{itemize}
\item $\alpha \mathbf{A} \in \R^{m \times n}$.
\item $(\alpha \beta) \mathbf{A} = \alpha (\beta \mathbf{A})$.
\item $\alpha(\mathbf{A} + \mathbf{B}) = \alpha \mathbf{A} + \beta \mathbf{B}$.
\item $(\alpha + \beta)\mathbf{A} = \alpha \mathbf{A} + \beta \mathbf{A}$.
\item $1 \mathbf{A} = \mathbf{A}$.
\end{itemize}
\end{prop}

\begin{defi}
La matriz transpuesta de $A \in \R^{m \times n}$ se define como la matriz $\mathbf{A}^T$ obtenida de intercambiar filas por columnas en $\mathbf{A}$. Esto es, $a_{ij}^T = a_{ji}$.
\end{defi}

\begin{prop}
Sean $\mathbf{A}, \mathbf{B} \in \R^{m \times n}$ y $\alpha \in \R$. Las siguientes propiedades se cumplen:

\begin{itemize}
\item $(\mathbf{A} + \mathbf{B})^T = \mathbf{A}^T + \mathbf{B}^T$.
\item $(\alpha \mathbf{A})^T = \alpha \mathbf{A}^T$.
\end{itemize}
\end{prop}

\begin{defi}
Sea $\mathbf{A} \in \R^{n\times n}$.

\begin{itemize}
\item Se dice que $\mathbf{A}$ es una matriz simétrica si $\mathbf{A} = \mathbf{A}^T$.
\item Se dice que $\mathbf{A}$ es una matriz antisimétrica si $\mathbf{A} = -\mathbf{A}^T$.
\end{itemize}
\end{defi}

\begin{defi}
Sean $\mathbf{A} \in \R^{m \times p}$ y $\mathbf{B} \in \R^{p \times n}$. El producto matricial $\mathbf{A}\mathbf{B}$ se define como la matriz de tamaño $m \times n$ tal que

\[ (ab)_{ij} = \sum_{k=1}^{p} a_{ik}b_{kj} \] 
\end{defi}

\begin{defi}
Sean $\mathbf{A} \in \R^{m \times p}$ y $\mathbf{B}, \mathbf{C} \in \R^{p \times n}$. Las siguientes propiedades se cumplen:

\begin{itemize}
\item $\mathbf{A}(\mathbf{B} + \mathbf{C}) = \mathbf{A}\mathbf{B} + \mathbf{A}\mathbf{C}$.
\item $\mathbf{A}(\mathbf{B} \mathbf{C}) = (\mathbf{A}\mathbf{B})\mathbf{C}$.
\end{itemize}
\end{defi}

\begin{defi}
La matriz $\mathbf{I} \in \R^{n \times n}$ definida como 

\[ i_{ij} = \delta_{ij} = \begin{cases}
1 \text{ si } i = j\\
0 \text{ si } i \neq j
\end{cases} \]

recibe el nombre de matriz identidad de orden $n$. Además cumple que 

\[\mathbf{A} \mathbf{I} = \mathbf{A}\quad \text{y} \quad \mathbf{I} \mathbf{A} = \mathbf{A}\]

si $\mathbf{A} \in \R^{n \times n}$.
\end{defi}

\begin{defi}
Dadas $\mathbf{A}, \mathbf{B} \in \R^{n \times n}$ que satisfacen 

\[ \mathbf{A} \mathbf{B} = \mathbf{I} \quad \text{y} \quad \mathbf{B} \mathbf{A} = \mathbf{I} \]

se llama matriz inversa de $\mathbf{A}$ y se denota como $\mathbf{B} = \mathbf{A}^{-1}$.\\

Una matriz que no tiene matriz inversa recibe el nombre de matriz singular. En caso contrario, se dice matriz no singular. 
\end{defi}

\begin{defi}
Sean $\mathbf{A}$, $\mathbf{B}$ matrices no singulares. Las siguientes propiedades se cumplen:

\begin{itemize}
\item $(\mathbf{A}^{-1})^{-1} = \mathbf{A}$.
\item $\mathbf{A} \mathbf{B}$ es no singular.
\item $(\mathbf{A} \mathbf{B})^{-1} = \mathbf{B}^{-1} \mathbf{A}^{-1}$.
\item $(\mathbf{A}^{-1})^T = (\mathbf{A}^T)^{-1}$.
\end{itemize}
\end{defi}

\begin{defi}\label{def:diagonalmente_dominante}
Una matriz $\mathbf{A} \in \R^{n \times n}$ se dice diagonalmente dominante si 

\[ |a_{ii}| > \sum_{\substack{j=1 \\ j \neq i}}^{n} |a_{ij}| \quad \forall i = 1,\dots, n \]
\end{defi}



\section{Autovalores y autovectores}
\begin{defi}
Dada una matriz $\mathbf{A} \in \R^{n \times n}$, unos escalares $\lambda \in \R$ y un vector $\mathbf{0} \neq \mathbf{x} \in \R^{n \times 1}$ que satisfacen $\mathbf{A} \lambda = \lambda \mathbf{x}$ reciben el nombre de autovalores y autivectores de $\mathbf{A}$, respectivamente.
\end{defi}

\begin{defi}
El conjunto de todos los autovalores de $A \in \R^{n \times n}$, denotado por $\sigma(\mathbf{A})$, se llama espectro de $\mathbf{A}$.
\end{defi}

\begin{prop}
Sea $\mathbf{A} \in \R^{n \times n}$. Entonces,

\[ \lambda \in \sigma(\mathbf{A}) \iff \mathbf{A} - \lambda \mathbf{I} \text{ es singular } \iff \det(A - \lambda I) = 0 \] 
\end{prop}

\begin{defi}
El polinomio característico de $\mathbf{A} \in \R^{n \times n}$ es $p(\lambda) = \det(\mathbf{A} - \lambda \mathbf{I})$.\\

La ecuación característica de $\mathbf{A}$ es $p(\lambda) = 0$.
\end{defi}

\begin{defi}
Los autovalores son las soluciones de la ecuación característica.
\end{defi}

\begin{defi}
Dos matrices $\mathbf{A}, \mathbf{B} \in \R^{n \times n}$ se dicen semejantes si existe una matriz no singular $\mathbf{P}$ tal que $\mathbf{P}^{-1}\mathbf{A} \mathbf{P} = \mathbf{B}$.
\end{defi}

\begin{defi}
Sean $\lambda \in \sigma(\mathbf{A}) = \{\lambda_1, \lambda_2, \dots, \lambda_s\}$.\\

La multiplicidad algebraica de $\lambda$ es el número de veces que se repite como raíz del polinomio característico. En otras palabras, $\mathrm{ma}_\mathbf{A}(\lambda_i) = a_i \iff (x-\lambda_1)^{\alpha_1} \cdots (x - \lambda_s)^{a_s}  = 0$ es la ecuación característica de $\mathbf{A}$.\\

La multiplicidad geométrica de $\lambda$, $\mathrm{mg}_\mathbf{A}(\lambda)$ es el número máximo de autovectores linealmente independientes asociados a $\lambda$.
\end{defi}

\begin{defi}
Para toda $\mathbf{A} \in \C^{n \times n}$ y todo $\lambda \in \sigma(\mathbf{A})$ se cumple que

\[ \mathrm{mg}_\mathbf{A}(\lambda) \leq \mathrm{ma}_\mathbf{A}(\lambda) \]
\end{defi}

\begin{teo}
Una matriz $A \in \C^{n \times n}$ es diagonalizable si y sólo si $\mathrm{mg}_\mathbf{A}(\lambda) = \mathrm{ma}_\mathbf{A}(\lambda)$ para todo $\lambda \in \sigma(\mathbf{A})$.
\end{teo}

\subsection{M-matrices} \label{sec:m_matrices}

\begin{defi}
Una M-matriz $\mathbf{A}$ es una matriz tal que $a_{ij} \leq 0 \ \ \forall i \neq j$ y $\mathbf{A}^{-1} \geq \mathbf{0}$.
\end{defi}

\begin{defi}
Si $\mathbf{A} \in \R^{n\times n}$, el número 

\[\rho(\mathbf{A}) = \max_{\lambda \in \sigma(\mathbf{A})} |\lambda|\]
\end{defi}

\begin{prop}
Las siguientes condiciones se cumplen si $\mathbf{A}$ es una M-matriz:

\begin{itemize}
\item $\mathbf{A}$ es una M-matriz si y sólo si existe una matriz $\mathbf{A} \geq \mathbf{0}$ y un número $r > \rho(\mathbf{B})$ tal que $\mathbf{A} = r\mathbf{I} - \mathbf{B}$.
\item Si $\mathbf{A}$ es una M-matriz, entonces $\Re(\lambda) \ \forall \lambda \in \sigma(\mathbf{A})$.
\item Si $\mathbf{A}$ es una M-matriz, entonces todos los menores principales de $\mathbf{A}$ son positivos.
\end{itemize}
\end{prop}

\subsection{Forma canónica de Jordan}

\begin{defi}
El índice de un autovalor $\lambda$ de una matriz $\mathbf{A} \in \C^{n \times n}$ se define como el entero más pequeño que cumple que

\[ \rang((\mathbf{A} - \lambda \mathbf{I})^k) = \rang((\mathbf{A} - \lambda \mathbf{I})^{k+1}) \] 

Se denota como $\mathrm{indice}(\lambda)$.\\

Se entiende que $\mathrm{indice}(\mu) = 0 \iff \mu \not \in \sigma(\mathbf{A})$.
\end{defi}

\begin{teo}
Para cada $\mathbf{A} \in \C^{n \times n}$ con distintos autovalores $\sigma(\mathbf{A}) = \{\lambda_1, \lambda_2,\dots, \lambda_s \}$, existe una matriz no singular $\mathbf{P}$ tal que

\[ \mathbf{P}^{-1} \mathbf{A} \mathbf{P} = \mathbf{J} = \left(\begin{array}{cccc}
\mathbf{J}(\lambda_1) & \mathbf{0}            & \dots  & \mathbf{0}\\
\mathbf{0}            & \mathbf{J}(\lambda_2) & \dots  & \mathbf{0}\\
\vdots                & \vdots                & \ddots & \vdots\\
\mathbf{0}            & \mathbf{0}            & \dots  & \mathbf{J}(\lambda_s)
\end{array}\right) \]

donde 

\[ \mathbf{J}(\lambda_j) = \left(\begin{array}{cccc}
\mathbf{J}_1(\lambda_j) & \mathbf{0}              & \dots  & \mathbf{0}\\
\mathbf{0}              & \mathbf{J}_2(\lambda_j) & \dots  & \mathbf{0}\\
\vdots                  & \vdots                  & \ddots & \vdots\\
\mathbf{0}              & \mathbf{0}              & \dots  & \mathbf{J}_{t_j}(\lambda_j)
\end{array}\right) \quad \text{ con } \mathbf{J}_\cdot(\lambda_j) = 
\left(\begin{array}{cccc} 
\lambda_j & 1      &         & \\
          & \ddots & \ddots     & \\
          &        & \ddots     & 1\\
          &        &            & \lambda_j
\end{array}\right) \]

y $t_i = \dim(\mathbf{A} - \lambda \mathbf{I})$.\\

El mayor bloque de Jordan de $\mathbf{J}_{\lambda_j} = \mathrm{indice}(\lambda_j) \times \mathrm{indice}(\lambda_j)$.
\end{teo}

\todo[backgroundcolor=red, inline]{Añadir sección de mínimos cuadrados}